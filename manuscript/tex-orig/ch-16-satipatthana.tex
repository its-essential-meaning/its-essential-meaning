\emph{(Cattāro Satipaṭṭhānā)}


\label{start}The actual way of living out the Noble Eightfold Path for the
development of Wisdom and therewith gaining deliverance from Suffering,
or for attaining Arahatship, is the practising of the Fourfold
Applications of Mindfulness (\emph{cattāro satipaṭṭhānā}). How one
practises this Way of Mindfulness is given in the Discourse called the
\emph{Mahā Satipaṭṭhānā Sutta}.\footnote{\href{https://suttacentral.net/dn22/en/sujato}{DN 22} and \href{https://suttacentral.net/mn10/en/sujato}{MN 10}}


It is sometimes thought that the practice of this Way of Mindfulness can
be undertaken without any prior understanding of the Buddha’s Teaching.
This is wrong. To the one who examines the \emph{Satipaṭṭhānā Sutta} carefully
it is quite clear that there must be a good understanding of the
Teaching if one is to embark on the practice of the four \emph{satipaṭṭhānas}
so to obtain any beneficial results. Repeatedly the \emph{Satipaṭṭhānā Sutta}
says "abides seeing the nature of things in things" (\emph{dhammesu
dhammānupassī viharati}),
and this abiding is defined as understanding or knowing as
it really is (\emph{yathābhūtaṁ pajānāti}), This means that the individual
practising it is one who is \textbf{seeing}.


Further, the \emph{Sutta} says that if the \emph{Satipaṭṭhānā} is practised for
between seven years to seven days the individual so practising it can
expect either Arahatship or \emph{anāgāmi}-ship. This therefore indicates
that, if such great results are to be expected, its practice has to be a
full-time pursuit which cannot in any way be taken lightly. For
instance, \emph{kamesu micchācārā vāyāmo} will not be a mere avoidance of
"wrongful" sex conduct as it is sometimes supposed to be, but a
\textbf{complete cutting away} from \textbf{all} pleasures of the senses. Such a
thorough practice is very difficult for a householder. Therefore it
would be incorrect to expect one to become a \emph{sotāpanna}, \emph{sakadāgāmi}
or \emph{anāgāmi}, or even to reach the Path, by a repetition of the
\emph{Satipaṭṭhānā Sutta} however often and regularly that be.


It is also sometimes thought that the fruits mentioned in the
\emph{Satipaṭṭhānā Sutta} can be achieved quickly and in a comfortable manner
without sufficient renunciation. Such individuals sooner or later find
themselves disillusioned. And then the worst of it all happens. Having
been so disillusioned, they begin to wonder whether the Buddha has been
right or wrong; and to add to the bargain they imagine that they are now
in a better position to wonder.


The Buddha says that the \emph{Satipaṭṭhānā} is the "one and only way"
(\emph{ekāyano maggo}) to the full comprehension of the Four Noble Truths
and therefore to Arahatship. In order to see this one should examine in
detail what is meant by "abides seeing the nature of things in things"
(\emph{dhammesu dhammānupassī viharati}).


\emph{Viharati} means abides or lives. That means one is having living
experience. In other words one is \textbf{conscious} of something Categorizing
broadly, one is conscious of the four Groups of Form, Feeling,
Perception and \emph{Determinations}. That any of these Four Groups is
\textbf{present} means one is conscious of it.


Consciousness is \textbf{always} entailed. That is why Consciousness is not one
of the four \emph{satipaṭṭhāna} — the four \emph{satipaṭṭhānas} being on the Body
(i.e. the most important Form to one), Feeling, Mentality (\emph{citta})
and dhammas (\emph{things}). Not doubt Consciousness is included in the
list of the \textbf{dhammas} which are to be contemplated on under the fourth
\emph{satipaṭṭhāna} called \emph{dhammesu dhammānupassī viharati}. But that is
different.


Now, all living experience can be classified under two categories:


\begin{enumerate}

\item{Experiencing something and having \textbf{right} knowledge about the experience,}

\item{Experiencing something and having \textbf{wrong} knowledge about the experience.}

\end{enumerate}


I can see a rope and recognize it as a rope, or I can see a rope and
take it for a snake. Whilst seeing the sun shining upon the sand I can
take it to be 'water' or to be the sun shining upon the sand. The
seeing, together with the wrong understanding, is as much a living
experience as the seeing together with the right understanding is.
Likewise one experiences a certain thing. One feels a feeling (\emph{vedanaṁ
vediyāmī}). That is, there is \emph{vedanāsu} …​ \emph{viharati}. One
experiences a lustful thought (\emph{sarāgaṁ cittaṁ}). That is, there is
\emph{citte} …​ \emph{viharati}. Likewise there is the experience of the
various \emph{dhammas} That is, there is \emph{dhammesu} …​ \emph{viharati}. But -
and this is the important thing — one can see the true nature of that
which is being experienced \textbf{or} not see it. Seeing the true nature of
the feeling that is being experienced is the \emph{vedanānupassī}.
Likewise, seeing the true nature of the thought is the \emph{cittānupassī}.
Seeing the true nature of the \emph{dhamma} is the \emph{dhammānupassī}. So,
together we get \emph{vedanāsu vedanānupassī viharati}, \emph{citte cittānupassī
viharati} and \emph{dhammesu dhammānupassi viharati}.


The position with regard to the \emph{satipaṭṭhāna} on the body is slightly
different towards the latter part, in that one does not and cannot
experience in oneself all the states of the body described therein, such
as the dead body in the charnel-field, though of course one sees that
the same fate will befall one’s own body. In this particular case one
sees the phenomenon externally (i.e. as of another) but as applicable
internally (i.e. to oneself) too. A matter worthy of note in this
\emph{satipaṭṭhāna} concerning the body is the use of the word
\emph{kāyasankhāra}. Having spoken of the in-breathing and out-breathing,
the word \emph{kāyasaṅkhāra} is brought in. \emph{Kāyasankhāra}, we have seen,
has been defined as in-breathing and out-breathing. Diverting the mind
to \emph{kāyasaṅkhāra} is to indicate that the in-breathing and out-breathing
is the \emph{saṅkhāra} upon which the body stands supported When the thing
(body, in this case) is seen to depend on a \emph{saṅkhāra} (breathing, in
this case) that is subject to arising and passing away, then it is seen
that the thing (body) is also subject to arising and passing away, and
is therefore Not-self. Therefore to translate \emph{kāyasankhāra} as
"activity of the body" or as "bodily formation" is not only wrong but
also misleading and misses the entire purpose.


It is quite clear that there can be \textbf{no other way} for one to fully
comprehend things. The \emph{dhammesu} …​ \emph{viharati} part is necessary for
\textbf{full} comprehension, since full comprehension comes only with actual
experience. That is why, though the \emph{sekha} sees the cessation of
Suffering, he is described as not having fully comprehended it. To fully
comprehend it or \textbf{penetratively} see it \textbf{through and through} he must
also \textbf{experience} it. The Arahat is at all times experiencing the
cessation of Suffering. He therefore fully comprehends it and sees it
penetratively through and through.


\label{truth-for-him}The \emph{Satipatthānā Sutta} assumes a prior understanding of the Buddha’s
Teaching. Obviously, this understanding cannot be obtained from this
\emph{Sutta}. It has to be obtained from the other Suttas. Therefore,
before embarking on the actual practice of the \emph{Satipatthāna} one has to
go through the other Suttas and devote a great deal of time to trying
to obtain sufficient understanding of the Buddha’s Teaching. And the
most certain way of obtaining a proper understanding of it is to build
one’s understanding on the very fundamentals that the Buddha has taught
in the \emph{Mūlapariyāya Sutta}. But very hard work is needed. In
conclusion one can only repeat what has already been said in the preface
- that is, that though these fundamentals and their resultant
implications are very difficult to \textbf{see}, they edify him who sees
them. They are truth \textbf{for him}.


