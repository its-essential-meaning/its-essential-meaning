Then the Venerable Sāriputta approached the Blessed One, paid homage to him, and sat down to one side. The Blessed One then said to him:


"Sāriputta, I can teach the Dhamma briefly; I can teach the Dhamma in detail; I can teach the Dhamma both briefly and in detail. It is those who can understand that are rare."


"It is the time for this, Blessed One. It is the time for this, Fortunate One. The Blessed One should teach the Dhamma briefly; he should teach the Dhamma in detail; he should teach the Dhamma both briefly and in detail. There will be those who can understand the Dhamma."


"Therefore, Sāriputta, you should train yourselves thus: (1) 'There will be no I-making, mine-making, and underlying tendency to conceit in regard to this conscious body; (2) there will be no I-making, mine-making, and underlying tendency to conceit in regard to all external objects; and (3) we will enter and dwell in that liberation of mind, liberation by wisdom, through which there is no more I-making, mine-making, and underlying tendency to conceit for one who enters and dwells in it.' It is in this way, Sāriputta, that you should train yourselves.


"When, Sāriputta, a bhikkhu has no I-making, mine-making, and underlying tendency to conceit in regard to this conscious body; when he has no I-making, mine-making, and underlying tendency to conceit in regard to all external objects; and when he enters and dwells in that liberation of mind, liberation by wisdom, through which there is no more I-making, mine-making, and underlying tendency to conceit for one who enters and dwells in it, he is called a bhikkhu who has cut off craving, stripped off the fetter, and, by completely breaking through conceit, has made an end of suffering. And it was with reference to this that I said in the Pārāyana, in 'The Questions of Udaya':


\begin{quotation}
"The abandoning of both \\
sensual perceptions and dejection; \\
the dispelling of dullness, \\
the warding off of remorse;


"purified equanimity and mindfulness \\
preceded by reflection on the Dhamma: \\
this, I say, is emancipation by final knowledge, \\
the breaking up of ignorance."


\end{quotation}


\vfill\eject

Atha kho āyasmā sāriputto yena bhagavā tenupasaṅkami; upasaṅkamitvā bhagavantaṁ abhivādetvā ekamantaṁ nisīdi. Ekamantaṁ nisinnaṁ kho āyasmantaṁ sāriputtaṁ bhagavā etadavoca: "saṅkhittenapi kho ahaṁ, sāriputta, dhammaṁ deseyyaṁ; vitthārenapi kho ahaṁ, sāriputta, dhammaṁ deseyyaṁ; saṅkhittavitthārenapi kho ahaṁ, sāriputta, dhammaṁ deseyyaṁ; aññātāro ca dullabhā"ti.


"Etassa, bhagavā, kālo, etassa, sugata, kālo yaṁ bhagavā saṅkhittenapi dhammaṁ deseyya, vitthārenapi dhammaṁ deseyya, saṅkhittavitthārenapi dhammaṁ deseyya. Bhavissanti dhammassa aññātāro"ti.


"Tasmātiha, sāriputta, evaṁ sikkhitabbaṁ: 'imasmiñca saviññāṇake kāye ahaṅkāramamaṅkāramānānusayā na bhavissanti, bahiddhā ca sabbanimittesu ahaṅkāramamaṅkāramānānusayā na bhavissanti, yañca cetovimuttiṁ paññāvimuttiṁ upasampajja viharato ahaṅkāramamaṅkāramānānusayā na honti tañca cetovimuttiṁ paññāvimuttiṁ upasampajja viharissāmā’ti. Evañhi kho, sāriputta, sikkhitabbaṁ.


Yato ca kho, sāriputta, bhikkhuno imasmiñca saviññāṇake kāye ahaṅkāramamaṅkāramānānusayā na honti, bahiddhā ca sabbanimittesu ahaṅkāramamaṅkāramānānusayā na honti, yañca cetovimuttiṁ paññāvimuttiṁ upasampajja viharato ahaṅkāramamaṅkāramānānusayā na honti tañca cetovimuttiṁ paññāvimuttiṁ upasampajja viharati; ayaṁ vuccati, sāriputta: 'bhikkhu acchecchi taṇhaṁ, vivattayi saṁyojanaṁ, sammā mānābhisamayā antamakāsi dukkhassa'.


Idañca pana metaṁ, sāriputta, sandhāya bhāsitaṁ pārāyane udayapañhe:


\begin{quotation}
'Pahānaṁ kāmasaññānaṁ, \\
domanassāna cūbhayaṁ; \\
Thinassa ca panūdanaṁ, \\
kukkuccānaṁ nivāraṇaṁ.


Upekkhāsatisaṁsuddhaṁ, \\
dhammatakkapurejavaṁ; \\
Aññāvimokkhaṁ pabrūmi, \\
avijjāya pabhedanan'"ti.


\end{quotation}

