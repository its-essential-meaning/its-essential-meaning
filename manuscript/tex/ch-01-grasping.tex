\chapter{Grasping}

\begin{quote}
``\,`Person! Person!' (\textit{sakkāya}), Venerable One, it is said. But what is it that the Exalted One has called the `person'?''

``These Five Grasping Groups (\textit{upādāna-kkhandhā}), friend Visākha, has the Exalted One called the `person', namely: the Grasping Group of Form, the Grasping Group of Feeling, the Grasping Group of Perception, the Grasping Group of Determinations, the Grasping Group of Consciousness. These Five Grasping Groups, friend Visākha, has the Exalted One called the `person'.''

 -- \href{https://suttacentral.net/mn44/en/sujato}{MN 44}, The Shorter Classification
\end{quote}

Thus the Buddha teaches me that I comprise five groups or aggregates of Grasping. The Pali word \textit{upādāna} has been translated here as \textit{Grasping}. It may also be translated as \textit{Holding}.

\begin{quote}
And what, monks, is Form (\textit{rūpa})? The Four Primary Modes (\textit{dhātu}), and the Form that is present by grasping (\textit{upādāya}) the Four Primary Modes -- this, monks, is called Form.

 -- \href{https://suttacentral.net/sn22.56/en/bodhi}{SN 22.56}, Perspectives
\end{quote}

The Four Primary Modes mentioned here are the Earth-Mode, Water-Mode, Fire-Mode and Air-Mode. They are often referred to as `elements'. But in relation to `matter', which is what Form refers to `elements' gives the idea of indivisible fundamental ingredients, and a wrong impression can be created that Buddhism splits the world-mass into four distinct fundamental ingredients. Form refers to what we call `matter'. But the Four Primary Modes do not refer to four elements or ingredients which constitute this `matter'. They refer to four distinguishable general \emph{modes of behaviour}, according to which `matter' makes itself known. The most important group of behaviours to me is that which I refer to as `my body' -- `this material body made up of the Four Primary Modes' (\textit{kāyo rūpī catummahābhūtiko}, \href{https://suttacentral.net/mn74/en/sujato}{MN 74}).

\begin{quote}
And what, monks, is Feeling (\textit{vedanā})? It is these six feeling-groups, namely: feeling sprung from Contact with the eye, feeling sprung from Contact with the ear, feeling sprung from Contact with the nose, feeling sprung from Contact with the tongue, feeling sprung from Contact with the body, feeling sprung from Contact with the mind. This, monks, is called Feeling.
\end{quote}

In the above passage, by Contact (\textit{phasso}) is not meant what is commonly referred to as `sense-impression'. As we shall see later on, Contact is the \emph{coming together} of three things, the sense-base (eye, ear, etc.), the corresponding percept (sight, sound, etc.), and the kind of Consciousness involved (eye-consciousness, ear-consciousness, etc.).

\begin{quote}
And what, monks, is Perception (\textit{saññā})? It is these six perception-groups namely: sight-perception, sound-perception, smell-perception, taste-perception, touch-perception, idea-perception (\textit{dhammasaññā}). This is called Perception.

And what, monks, are Determinations (\textit{sankhārā})? It is these six intention-groups (\textit{cetanākāya}), namely: intention with regard to sight, intention with regard to sound, intention with regard to smell, intention with regard to taste, intention with regard to touch, intention with regard to ideas. These, monks are called Determinations.

And what, monks, is Consciousness (\textit{viññāna})? It is these six consciousness-groups, namely: eye-consciousness, ear-consciousness, nose-consciousness, tongue-consciousness, body-consciousness, mind-consciousness. This, monks, is called Consciousness.

 -- \href{https://suttacentral.net/sn22.56/en/bodhi}{SN 22.56}, Phases of the Clinging Aggregates
\end{quote}

The personality is thus analysed and broken up into its constituent parts. My entire being is composed of them. Beyond them there is naught else for me. \emph{My world} is the totality of these Five Grasping Groups. They constitute \emph{my world}.

None of these Groups can however exist by itself separated from the others. They are inseparable, and of their inseparability the Venerable Sāriputta says:

\begin{quote}
Whatever, friend, there exists of Feeling, of Perception, and of Consciousness, these things are associated and not dissociated, and it is impossible to dissociate one from the other and show their differences. For, whatever one feels, one perceives and whatever one perceives, of that one is conscious.

 -- \href{https://suttacentral.net/mn43/en/sujato}{MN 43}, The Great Classification
\end{quote}

Then again we have the Buddha teaching:

\begin{quote}
Were one, monks, to declare thus: `Apart from Form, apart from Feeling, apart from Perception, apart from Determinations, I will show the coming, or the going, or the disappearance, or the appearance, or the growth, or the increase, or the abundance of Consciousness' -- that is not possible.

 -- \href{https://suttacentral.net/sn22.53/en/bodhi}{SN 22.53}, Engagement
\end{quote}

Before proceeding any further it is extremely important to understand clearly what the Buddha defines as Grasping (\textit{upādāna}), or as Holding.

The difference between life and inanimate things is that in the former there is \emph{intention}. All conscious action is \emph{intentional}, whilst action pertaining to inanimate things is devoid of intention.

Now, in the context of the Five Grasping Groups, the Buddha defines the Group of Determinations as the Group of Intention (\textit{cetanā}). But why does he describe it is a \emph{Grasping} Group? The rest of mankind has seen Intention as either good or bad intention, moral or immoral intention, and so on. Nevertheless the Buddha appears to see something far more fundamental and deep-rooted in it. All these intentions, whether they be good or bad, moral or immoral, or anything else, he groups together and describes as a Grasping. To him they all appear to be basically of one and the same character. They are all forms of Grasping.

What then is Grasping?

And what precisely is the difference between Grasping (\textit{upādāna}) and Intention (\textit{cetanā})?

This, more than any other, is the fundamental question posed by the Buddha's Teaching.

Now, it is easy to \emph{state} the answer, but it is extremely difficult to \emph{see} it.

The answer is: essentially, all notions of subjectivity, all notions of a `self' or a `person' or a `somebody', all thoughts of `I' and `mine', are Grasping. Thus the Grasping Group of Determinations (here Intention) means, the \emph{Group of Intention based an notions of `self' and thoughts of `I' and `mine'}. Intentions which are \emph{not} based on any notion of `self' or any thought of `I' or `mine' whatever are merely a Group of Intention.

We shall consider the relationship between the three notions: `self', `I' and `mine' in more detail later on. Of these three notions, the most fundamental one is `mine'. To \emph{grasp} something (or \emph{hold} something) means to \emph{consider it as `mine'}. It is not easy to \emph{see} this. But it is extremely important that it is seen.

To grasp Form means: to consider Form as `mine'. The Grasping Group of Form means \emph{Group of Form considered as `mine'}. So it is with the other Groups.

Further, if there be anything that is grasped or can be grasped, then that is the Five Groups or a part thereof. When I say I grasp a certain external material object, what I really mean is that I grasp those feelings, perceptions, etc., which arise when I become conscious of that object. If I do not want those particular feelings, perceptions etc., then I do not want the object also, and hence will not grasp it.

Now, just as much as one grasps Form, one grasps Feeling, Perception, Determinations and Consciousness also. One considers them all as `mine'.

My world is the Five Grasping Groups that go to make me up. If there is anything that I must comprehend, then it must be these.

\begin{quote}
Monks, I will show you things that are to be comprehended, and what comprehending is \ldots\hspace{0pt} Do ye listen to it. And what, monks, are the things to be comprehended? Form, monks, is a thing to be comprehended; Feeling is a thing to be comprehended; Perception is a thing to be comprehended; Determinations are a thing to be comprehended; Consciousness is a thing to be comprehended. These, monks, are the things to be comprehended. And what, monks, is comprehending \ldots\hspace{0pt}

 -- \href{https://suttacentral.net/sn22.106/en/sujato}{SN 22.106}, Should Be Completely Understood
\end{quote}

The immediate question that arises is: Could there be a Group of Form, a Group of Feeling, a Group of Perception, a Group of Determinations, a Group of Consciousness which is wholly and entirely \emph{devoid} of notions of `self' and thoughts of `I' and `mine'? Or, as against the Five Grasping Groups could there be just the Five Groups? Particularly, with regard to intentional action, could there be any such action which is \emph{unaccompanied} by any notions of `self' and thoughts of `I' and `mine'?

For the present we leave this question unanswered and proceed.
