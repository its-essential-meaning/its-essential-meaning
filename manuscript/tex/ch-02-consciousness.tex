\chapter{Consciousness}

In the elucidation of the Five Grasping Groups, Consciousness takes the pride of place. The reason for that is that any \textbf{experience} means \textbf{being conscious of one or more of the other four Groups}. I am conscious of Form (i.e. I am conscious either of my body or of an external object or of both); I am conscious of Feeling; I am conscious of Perception; and I am conscious of Determinations.

\begin{quote}
It is as with the choir-master of a five-member choir who himself, as the chief, takes up his part and in the performance of the whole piece takes in himself along with it.
\end{quote}

What now is Consciousness (\emph{viññāna})?

When I say I am conscious of something it means that that something is \textbf{present} to me. A sight, a sound, a smell, a taste, a touch, or an idea is present. I am \textbf{aware} of a certain perception, or the perception is \textbf{present} to me. I am conscious of it.

Sometimes Consciousness is seen equated to the subject to whom the phenomenon is present. This is not correct. Consciousness does not refer to the subject. Neither does it refer to the phenomenon nor to a part of the phenomenon. It is not \textbf{what} is present or a part of what is present. It is only the \textbf{presence} of the phenomenon. It is the presence of that which is present. A feeling is present to me. Consciousness is not the feeling. It is only the presence of the feeling. It is the being conscious of the feeling Presence that is `mine' or presence `for me' is Grasping-Consciousness (\emph{upādāna-viññāna}).

Without sufficient reason, the Buddha says, no Consciousness arises.

\begin{quote}
Consciousness, monks, is named after that in dependence on which it comes into being.

\begin{itemize}
\item
  The Consciousness which comes into being in respect of sights in dependence on the eye is called eye-consciousness;
\item
  the Consciousness which comes into being in respect of sounds in dependence on the ear is called ear-consciousness;
\item
  the Consciousness which comes into being in respect of odours in dependence on the nose is called nose-consciousness;
\item
  the Consciousness which comes into being in respect of tastes in dependence on the tongue is called tongue-consciousness;
\item
  the Consciousness which comes into being in respect of touch in dependence on the body is called body-consciousness;
\item
  the Consciousness which comes into being in respect of ideas in dependence on the mind is called mind-consciousness.
\end{itemize}

Just as, monks, fire is named after that independence on which it burns. The fire that burns in dependence on logs of wood is called a log-fire; the fire that burns in dependence on chips is called a chip-fire; the fire that burns in dependence on grass is called a grass-fire; the fire that burns in dependence on cow-dung is called a cow-dung fire; the fire that burns in dependence on husks is called a husk-fire; the fire that burns in dependence on rubbish is called a rubbish-fire. In the same way, monks, Consciousness is named after that in dependence on which it comes into being. The Consciousness that comes into being in respect of sights in dependence on the eye is called eye-consciousness \ldots{}

 -- \href{https://suttacentral.net/mn38/en/bodhi}{MN 38}, The Greater Discourse on the Destruction of Craving
\end{quote}

The four Groups, Form, Feeling, Perception, Determinations, are called the supports, or the footholds, or the base, for Consciousness.

\begin{quote}
``There are these five kinds of seed, monks. What five? Seed from root, seed from trunk, seed from joints, seed from shoots, and seed from grain.

``If, monks, these five kinds of seed were present undamaged, not rotten, unspoiled by wind and heat, capable of sprouting, well preserved, but there is no earth and water, would, monks, these five kinds of seed come to growth, spread, and increase?''

``No, Lord.''

``If, monks, these five kinds of seed were damaged, rotten, spoilt by wind and heat, incapable of sprouting, not well preserved, but there is earth and water, would, monks, these five kinds of seed come to growth, spread, and increase?''

``No, Lord.''

``If, monks, these five kinds of seed were undamaged, not rotten, unspoiled by wind and heat, capable of sprouting, well preserved, and there is earth and water, would, monks, these five kinds of seed come to growth, spread and increase?''

``Yes, Lord.''

``As the earth, monks, should the four supports for the persistence of Consciousness be regarded. As the water, monks, should delight and attachment be regarded. As the five kinds of seed, monks, should the nutritive Consciousness be regarded.

``If Consciousness persists, monks, it is by holding to Form that it persists. With Form as object, with Form as support, in association with delight, it attains to growth, spread and increase.

``If Consciousness persists, monks, it is by holding to Feeling \ldots{} Perception \ldots{} Determinations \ldots{} that it attains to growth, spread and increase.''

 -- \href{https://suttacentral.net/sn22.54/en/bodhi}{SN 22.54}, Seeds
\end{quote}

The footholds for Consciousness can be viewed from a second angle. That is through a dual classification of \textbf{internal} and \textbf{external} bases. The six sense-bases, viz., the eye, the ear, the nose, the tongue, the body, and the mind, are called the internal bases or the internal supports for Consciousness, whilst those phenomena corresponding to these six sense-bases, viz., sight, sound, smell, taste, touch and idea are called the external bases or the external supports for Consciousness. The latter are called external bases because they are largely dependent on objects external to the corresponding internal base.

Consciousness and the other four Groups Cannot therefore be comprehended from a standpoint outside of them by any method of objective synthesis induction, and so on. Through themselves, and only through themselves, i.e., by one's own experience \textbf{only} can they be understood.
