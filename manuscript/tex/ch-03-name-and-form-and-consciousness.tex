\chapter{Name-and-Form and Consciousness}

When Consciousness is explained as something that arises and ceases, the question follows: What are the conditions necessary for the arising of Consciousness, and its ceasing?

To this, the Buddha gives the answer: Name-and-Form (\emph{nāma-rūpa}) is the basis, the genesis, the condition for Consciousness.

`What being present is Consciousness present? Dependent on what does Consciousness exist?' The answer is: `Name-and-Form being present, there is Consciousness. Dependent on Name-and-Form, Consciousness exists.' (\href{https://suttacentral.net/dn14/en/sujato}{DN 14})

Thus the condition necessary for the arising of Consciousness is Name-and-Form.

Again, `What being present is Name-and-Form present? Dependent on what does Name-and-Form exist?' The answer is: `Consciousness being present, there is Name-and-Form. Dependent on Consciousness, Name-and-Form exists.' (\href{https://suttacentral.net/dn14/en/sujato}{DN 14})

Thus the condition for Consciousness is Name-and-Form, and the condition for Name-and-Form is Consciousness.

\begin{quote}
Consciousness turns back from Name-and-Form; it goes not beyond.

 --- \href{https://suttacentral.net/dn14/en/sujato}{DN 14}, The Great Discourse on the Harvest of Deeds
\end{quote}

All this needs explaining.

To start with, there must be a clear understanding of what is referred to as Name-and-Form (\emph{nāma-rūpa}). It is where there is no such understanding that one find this phenomenon called \emph{nāma-rūpa} referred to as `mind-and-matter'. \emph{Rūpa} is certainly `matter', but as we shall see \emph{nāma} is not `mind'.

Firstly, what is \emph{rūpa}, which has been translated as Form?

Form, as just stated, refers to `matter'.

Now, any Form or lump of `matter' can be regarded as a particular \textbf{group of behaviours}. Since a particular lump of `matter' or a particular group of behaviours is always \textbf{present} in the same fashion, I come to the \textbf{conclusion} that that `matter' exists independent of my senses. Since I always note with regard to that `matter' the same sights, sounds, smells, etc. I conclude that that `matter' exists independent of myself. Further, since the same `matter' exhibits almost the very same sights, sounds, smells, etc. to every individual, we conclude that there is a `material world' existing quite independent of us individuals.

The various modes of behaviour are not dependent on Consciousness. But to distinguish one mode of behaviour from another they have to be \textbf{cognized} or they must be made \textbf{present}. When so cognized these behaviours \textbf{appear} in a certain fashion, or, when they are made to be present they are then present in a certain fashion. That means, there is an \textbf{appearance} of these behaviours (the word `appearance' being taken in a rather wide sense) --- an appearance which takes the form of sights, sounds, smells, etc. Further, this appearance behaves in a certain fashion. Thus there is both an \textbf{appearance of behaviour} and a \textbf{behaviour of appearance}. And the set of behaviours defining the particular lump of `matter' or object is \textbf{inferred} from the behaviour of its \textbf{appearance}.\footnote{See \href{ch-13-nibbana.xml\#the-four-primary}{Chapter 13, Nibbāna}: `But their {[}the Four Primary Modes} \textbf{appearance} is a matter for Consciousness, and their `existence' is \textbf{inferred} through the behaviour of this \textbf{appearance}.'{]}

All modes of behaviour can be categorized under four main modes called the Four Primary Modes (\emph{catunnaṁ mahābhūtānaṁ}). They are Earth-Mode, Water-Mode, Fire-Mode and Air-Mode. They may also be called the Solid-Mode, the Fluid-Mode, the Ripening-Mode and the Motion-Mode.

\begin{quote}
And what, monks, is the Earth-Mode (\emph{paṭhavīdhātu})? The Earth-Mode may be internal, may be external. And what, monks, is the internal Earth-Mode? Whatever is hard, solid, is internal, \textbf{grasped by oneself} (\emph{paccattaṁ \ldots{} upādinnaṁ}), that is to say: the hair of the head, the hair of the body, nails, teeth, skin, flesh, sinews, bones, marrow of the bones, kidneys, heart, liver, pleura, spleen, lungs, intestines, mesentery, stomach, excrement, or whatever other thing is hard, solid, is internal, grasped by oneself --- this, monks, is called the internal Earth-Mode. Whatever is the internal Earth-Mode and whatever is the external Earth-Mode, just these are the Earth-Mode \ldots{}

And what, monks, is the Water-Mode (\emph{āpodhātu})? The Water-Mode may be internal, may be external. And what, monks, is the internal Water-Mode? Whatever is liquid, become liquid, is internal, grasped by oneself, that is to say: bile, phlegm, pus, blood, sweat, fat, tears, serum, saliva, mucus, synovial fluid, urine, or whatever other thing is liquid, become liquid, is internal, grasped by oneself --- this, monks, is called the internal Water-Mode. Whatever is the internal Water-Mode and whatever is the external Water-Mode, just these are the Water Mode \ldots{}

And what, monks, is the Fire-Mode (\emph{tejodhātu})? The Fire-Mode may be internal, may be external. And what, monks, is the internal Fire-Mode? Whatever is heat, become heat, is internal, grasped by oneself, that is to say: that by which one is vitalized, that by which one is consumed, that by which one is scorched, that by which what has been munched, drunk, eaten and tasted is fully digested, or whatever other thing is heat, become heat, is internal, grasped by oneself --- this, monks, is called the internal Fire-Mode. Whatever is the internal Fire-Mode and whatever is the external Fire-Mode, just these are the Fire-Mode \ldots{}

And what, monks, is the Air-Mode (\emph{vāyodhātu})? The Air-Mode may be internal, may be external. And what, monks, is the internal Air-Mode? Whatever is air, become airy, is internal, grasped by oneself, that is to say: winds going upwards, winds going downwards, winds in the abdomen, winds in the belly, winds permeating the limbs, in-breathing, out-breathing, or whatever other thing is air, become airy, is internal, grasped by oneself -- this, monks, is called the internal Air-Mode. Whatever is the internal Air-Mode and whatever is the external Air-Mode, just these are the Air-Mode \ldots{}

 --- \href{https://suttacentral.net/mn140/en/bodhi}{MN 140}, The Exposition of the Elements
\end{quote}

In the above definitions the Buddha refers to the Four Primary Modes as `grasped by oneself' (\emph{paccattaṁ upādinnaṁ}). In other words, he is referring to the \textbf{Grasping} Group of Form (to \emph{rūpa-upādāna-kkhandha}).

Beyond the above there is only one important thing (according to the Suttas) the Buddha has thought about Form: that is, the question of the Four Primary Modes `getting no footing' (\emph{na gādhati}). We shall come to this later on. One might therefore wonder why the Buddha has taught so little about Form or `matter'. But the Buddha has a distinct purpose in his Teaching. And elucidations are made by him only in as far as such are necessary for that purpose. He seeks no intellectual approval of what he teaches. His Teaching is designed for a purpose. It is designed to lead one on (\emph{opanayika}) towards a particular goal.

The analysis of Form or `matter' given above is sufficient. No further analysis of it are essential, as they would not help me to solve the problem of the Five Grasping Groups, the problem of `my world'. What is essential is to realise that the analysis given by the Buddha is \textbf{sufficient}.

Form, the Buddha teaches, indicates a certain characteristic. That is, the characteristic of persisting (\emph{paṭigha}). It is similar to the idea of \textbf{inertia} taught in physical science. A tendency for a body to \textbf{maintain} its characteristics is demonstrated. By that what we really mean is: there is seen a tendency for the body to maintain its \textbf{appearance}. This is a very important characteristic of Form. It is really because of this characteristic that we can distinguish various objects from one another. If the table does not remain the table and the book does not remain the book when I am continuing to be conscious of them I cannot then distinguish them from each other.

There is a material object.\footnote{`Material object' is not quite the same as `matter'. The former is a particular `lump of ``matter``\,'.} This object is \textbf{not} dependent on Consciousness. But this object can be \textbf{present} or \textbf{not present}. Its presence is a matter of Consciousness. The object being present to the individual means that he is conscious of it. It has been `discovered' by his Consciousness as it were. Though the object does not depend on Consciousness there is no \textbf{presence} of the object if there is no Consciousness. Consciousness means this \textbf{presence}. \textbf{I am conscious} of an object means that that object is \textbf{present to me}.

Now, an object is always present in \textbf{some fashion}. It is present \textbf{as} shape, colour, smell, sound, etc. Its presence is therefore known by these. Or it is present in terms of these. Those things called shape, colour, sound, smell, etc., which are brought about when Consciousness `discovers' the object are called Name (\emph{nāma}). It is as if the particular Name is \textbf{how} the particular object is present (Consciousness). It is the \textbf{appearance} of the object (the word `appearance' being again taken in a rather wide sense). Therefore, we can define Name (\emph{nāma}) as `how Form (\emph{rūpa}) is present (\emph{viññāna})'. The \textbf{how} or the \textbf{manner} is Name, and the presence is Consciousness.

This appearance or `how it is present' is always given a designation (\emph{adhivacana}). This designation therefore actually belongs to \emph{nāma}. But we refer to the object by this designation.

It must be noted that this `how it is present' includes a number of things. The shape, colour, smell, sound, etc., are the perceptions. Then there are certain feelings which are either pleasant, unpleasant or neutral. Further, there is Intention, Attention and Contact in relation to the object. All these go to make up Name (\emph{nāma}). `Feeling, Perception, Intention, Contact, Attention --- this is called Name.' (\href{https://suttacentral.net/sn12.2/en/bodhi}{SN 12.2})

It should be quite clear from the above that \emph{nāma} is not `mind'. \emph{Rūpa} is `matter', but \emph{nāma} is not `mind'. `Mind', as a sense-base, is \emph{mano}; as mentality, it is \emph{citta}. Thus it is wrong to translate \emph{nāma-rūpa} as `mind-and-matter'.

It is not an uncommon thing to find Name (\emph{nāma}) being taken to include Consciousness (\emph{viññāna}). This is wrong. Name does \textbf{not} include Consciousness. It only entails Consciousness.

If we examine this further we shall find that:

\textbf{(1.)} Since `matter' has the characteristic of inertia or persistence, its appearance is seen to persist or remain the same. That is since Form (\emph{rūpa}) has the characteristic of persistence (\emph{paṭigha}), we discern in Name (\emph{nāma}) a persistence.

\textbf{(2.)} Since appearance has some particular designation, its `substance' (i.e. the `matter' which gives this appearance) is seen to have a designation. That is, since Name (\emph{nāma}) has designation --- (\emph{adhivacana}), we discern in Form (\emph{rūpa}) a designation.

It is important to see this since the Buddha refers to it when he teaches the relationship between Name-and-Form (\emph{nāmarūpa}) and Contact (\emph{phasso}), a relationship which we shall presently come to. We shall then be taking a particular experience in order to make the matter more clear.

What now are Intention (\emph{cetanā}), Attention (\emph{manasikāra}), and Contact (\emph{phasso}) which are included in Name (\emph{nāma})?

At this moment I am sitting. The \textbf{present} phenomenon is a sitting position. This present phenomenon, the sitting position, now brings to mind certain other phenomena such as a standing position, a lying position, etc. From the present sitting position, which is now the actual, it is possible to \textbf{make} actual one of these new positions or states which are now \textbf{not present}. Thus there is one actual state and many possible ones.

There is a relation between the present sitting position and the possible standing position. Likewise, there is a relation between the present sitting position and the possible lying position. This relation in one case is \textbf{that which is necessary to bring about the standing position from the sitting position}, and in the other case \textbf{that which is necessary to bring about the lying position from the sitting position}. Both these relations are \textbf{actions}. The \textbf{type} of action varies slightly. But basically they are both \textbf{actions}.

When the action is completed, and let us say, the standing position is present, then the sitting position has vanished, and the sitting position has become a \textbf{possible} present. The present actual has disappeared giving way to a possible becoming the present actual. The disappeared actual present is now only a possible present.

Adopting the standing position involves \textbf{selecting} or \textbf{choosing} the standing position from all the possible positions. And so I \textbf{exercise my choice}. There comes about an \textbf{opted action}. Thus the action involved in the change from sitting to standing is the \textbf{exercise of choice}. All other positions are sacrificed and this one position is consciously held to. This action, or this exercise of choice, is called \textbf{Intentional Action}. From the intentional action there comes to be \textbf{present} the new position. `Thus, Ānanda, intentional action is the field, Consciousness is the seed.'\footnote{\href{https://suttacentral.net/an3.76/en/thanissaro}{AN 3.76}, Continued Existence} Just as the seed springs up out of the field the new position becomes present (Consciousness) resulting from the intentional action (\emph{kamma}).

\textbf{All conscious action is intentional}. Conscious action is the exercise of preference for one available mode of behaviour or action at the expense of others. And it is this action, namely, the exercise of choice, that distinguishes life-action from material-action.

In the exercise of choice, or in intentional action, there is Attention (\emph{manasikāra}) towards that particular action. The attention on the action keeps the action going. The state of affairs is being preserved as it were. And intention cannot be present unless attention is present.

Contact (\emph{phasso}) now remains to be considered.

This word represents a very important phenomenon and so should be clearly understood. If this phenomenon called Contact is absent, there can be no experience. Examination of it also throws some light on how Name-and-Form is dependent on Consciousness and Consciousness is dependent on Name-and- Form.

\begin{quote}
In dependence on eye and sights springs up eye-consciousness. The \textbf{coming together} of the three is called Contact \ldots{} In dependence on ear and sounds \ldots{} In dependence on nose and odours \ldots{} In dependence on tongue and taste \ldots{} In dependence on body and touch \ldots{} In dependence on mind and ideas springs up mind-consciousness. The \textbf{coming together} of the three is called Contact.

 --- \href{https://suttacentral.net/sn12.43/en/bodhi}{SN 12.43}, Suffering
\end{quote}

There is something important to be noted here. Broadly, by Contact is meant the coming together of the percept, the sense-base and that particular sense-consciousness. But with regard to the \emph{puthujjana} (commoner)\footnote{\emph{Puthujjana} refers to the common or ordinary person, to the commoner.} what arises is Grasping-Consciousness (\emph{upādāna-viññāna}). Therefore, with the \emph{puthujjana}, Contact is \textbf{inclusive} of thoughts of `I' and `mine'. That is, there is contact between a subject who says `I' and `mine' and the object.

Contact (\emph{phasso}) is a particular form of coming together. It is a particular form of \textbf{union}. Perception, Feeling and Determinations come about because there is such a coming together. In other words, Perception, Feeling and Determinations are dependent on Contact.

Yet, though Perception, Feeling and Determinations are dependent on Contact, Form is not dependent on Contact. Form is dependent on the Four Primary Modes.

\begin{quote}
\protect\hypertarget{dependent}{}{}Monk, it is to be seen that the Group of Form (or `matter') is dependent on the Four Primary Modes, is conditioned by the Four Primary Modes. The Group of Feeling is dependent on, is conditioned by Contact. The Group of Perception is dependent on, is conditioned by Contact. The Group of Determinations is dependent on, is conditioned by Contact.

 --- \href{https://suttacentral.net/mn109/en/sujato}{MN 109}, The Longer Discourse on the Full-Moon Night
\end{quote}

Now, Contact is dependent on Name-and-Form. The Buddha teaches that this should be understood thus:

\begin{quote}
``Ānanda, those modes, features, characteristics, exponents, by which Name-body is to be seen --- if all those modes, features, characteristics exponents, were absent would a coming together of designation be evident in the Form-body (\emph{rūpakāye adhivacanasamphasso})?''

``It would not, Lord.''

``Ānanda, those modes, features, characteristics, exponents, by which Form-body is to be seen --- if all those modes, features, characteristics, exponents, were absent, would a coming together of inertia be evident in the Name-body (\emph{nāmakāye paṭighasamphasso})?''

``It would not, Lord.''

``Ānanda, those modes, features, characteristics, exponents, by which Form-body and Name-body are to be seen --- if all those modes, features, characteristics, exponents, were absent, would a coming together of designation and a coming together of inertia be evident?''

``They would not, Lord.''

``Ānanda, those modes, features, characteristics, exponents, by which Name-and-Form is to be seen --- if all those modes, features, characteristics, exponents, were absent, would there be Contact (that particular coming together)?''

``There would not, Lord.''

``Ānanda, those modes, features, characteristics, exponents, by which Name-and-Form is to be seen --- if all those modes, features, characteristics, exponents, were absent, would there be Contact (that particular coming together)?''

``There would not, Lord.''

``Wherefore, Ānanda, just that is the reason, the ground the arising, the condition for Contact, to wit, Name-and-Form.''

 --- \href{https://suttacentral.net/dn15/en/bodhi}{DN 15}, The Great Discourse on Causation
\end{quote}

Since it is important to understand this rather difficult teaching let us analyse a particular experience to make it clear.

There is a bottle of ink, or I am conscious of a bottle of ink. That is the experience.

This means that a Form (\emph{rūpa}) which appears as a `bottle of ink' (Name, \emph{nāma}) is present (Consciousness, \emph{viññāna}).

Now, if Feeling, Perception, etc., were absent would there be present a `bottle of ink'?

This question expanded would run thus: If the black colour, the shape, the smell, the neutral feeling, the intention to dip the pen in it, etc., were absent would a designation `bottle of ink' pertain to that Form (to that lump of `matter')?

The shape, smell, etc., are the features of the Name-body, and `bottle of ink' is the \textbf{designation}. Therefore, generalizing, the question would run thus: If those features, modes, characteristics exponents, by which the Name-body is discerned were absent, would there be a coming together of a designation in the Form-body?\footnote{`Designation in Form-body' (\emph{rūpakāye adhivacana}) corresponds to `appearance of behaviour'.}

The answer is: No.

Again, if the characteristics (like inertia) of the Form (of that lump of `matter') were absent, would the appearance designated `bottle of ink' remain so, or be inert?

Generalizing, the question would run thus: If those features, etc., by which Form-body is discerned were absent would there be a coming together of inertia in the Name-body?\footnote{`Inertia in Name-body' (\emph{nāmakāye paṭigha}) corresponds to `behaviour of appearance'.}

The answer is: No.

Thus, this particular coming together called Contact is possible only because Name has its own characteristics and Form has its own characteristics, which means that Contact is possible only because Name-and-Form are just what they are. Hence Contact is dependent on Name-and-Form.

That Consciousness is also dependent on Name-and-Form is now not so difficult to see. If Consciousness is to be there, Form must be there either as one's own or external to one; Intention must be there to determine what one should be conscious of; and, of course, where there is Intention there is Attention. But this alone is insufficient. Perception, Feeling, and Contact must also be there. Thus the sum total of Name-and-Form \textbf{must} be present for Consciousness to be present. Hence Consciousness is dependent on Name-and-Form.

Earlier we saw that there must be Consciousness for Name-and-Form to be there, Name being the manner in which Form appears when one is conscious of it. Without Consciousness there can be no Name-and-Form. Thus we have the triad: Name-and-Form depends on Consciousness, Consciousness depends on Name-and-Form, and Contact depends on Name-and-Form.

Name-and-Form and Consciousness arise \textbf{simultaneously}. One does not arise and wait for the other in time to arise in dependence upon it. They both arise in dependence on each other, and therefore \textbf{together}. Likewise they cease together. If one is there, so is the other. There is a total-either-way-simultaneity.

There are things which, however, do not have a total-either-way-simultaneity as Name-and-Form and Consciousness have. For example perception and knowledge. `Perception arises first, knowledge arises thereafter (in dependence on Perception)' (\href{https://suttacentral.net/dn9/en/thanissaro}{DN 9}) But the case with Name-and-Form and Consciousness is different. Since they depend on each other they arise together and cease together. One neither precedes nor follows the other in time. The relationship that Name-and-Form and Consciousness bear towards each other is therefore one that is `not involving time' or `timeless' (\emph{akālika}). As against this type of relationship, the relationship between in-breathing and out-breathing is one that is `involving time' (\emph{kālika}), since one follows or precedes the other in time. Incidentally, \emph{akālika} is to be given no other meaning than the one just given, and it is important to note that this is the actual meaning of this word. Various other meanings seem to be given to this word, resulting in confusion particularly when it comes to the Doctrine of Dependent Arising (\emph{paṭicca-samuppāda}).

The three Groups --- Feeling, Perception and Determinations --- taken together can also be called Name (\emph{nāma}). Since Name has been defined as the totality of Feeling, Perception, Intention, Contact and Attention, it means that, in this context, Determinations is the totality of Intention, Contact and Attention. That is possible because Perception directly involves the pair of bases for Consciousness and the kind of Consciousness involved (e.g., eye, sights, and eye-consciousness), which means that Contact (which is the coming together of these --- three is included, and the Fourth Group Determinations (as Intention) includes Attention, since in the exercise of choice there is always attention on the particular thing chosen. Thus the Five Groups --- Form, Feeling, Perception, Determinations and Consciousness --- can also be called Name-and-Form and Consciousness.
