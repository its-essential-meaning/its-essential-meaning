\chapter{Taṇhā and Bhava}

We have seen that any actual present points to many possibilities. From these possibilities conscious life makes a choice and exercises it. In exercising the choice Consciousness finds its new footing.

The question now is: \textbf{What} determines that particular choice and no other? Why is it that at any given instant I choose to do this and not any other?

The answer is simply that I \textbf{want} that thing towards which \textbf{that} particular action will lead me. Keeping the wanted thing in mind, or wanting that thing, I take the action that will lead me to it. Of all the courses of action available I select and pursue that particular course of action which leads me to the wanted thing. Throughout the action, the wanting lasts.

Now, a \emph{puthujjana}'s want can be categorized under three main headings called \emph{kāma}, \emph{bhava} and \emph{vibhava}. Wanting any one or more of these things is called \emph{taṇhā}.

The word \emph{taṇhā} is usually translated as Craving or Thirst. This however, tends to give an inaccurate picture, since either of the words Craving or Thirst gives the impression of an acute wanting. But wanting \emph{kāma}, \emph{bhava} or \emph{vibhava} in the \textbf{slightest} degree is \emph{taṇhā}. On the other hand the word \textbf{wanting} is too wide to be used for \emph{taṇhā}, since though the Arahat has \textbf{no taṇhā} he certainly has other wants like wanting to eat food when hungry or wanting to rest when tired. There seems to be no exact English equivalent to \emph{taṇhā}, so we shall use it as it is.

The \emph{puthujjana} has \emph{kāma-taṇhā}, \emph{bhava-taṇhā} and \emph{vibhava-taṇhā}.

What now are \emph{kāma}, \emph{bhava} and \emph{vibhava}?

\emph{Kāma} is sense-pleasure,\footnote{Pleasure, it should be noted, is not the feeling born of the senses. One can take pleasure in a feeling or not take pleasure in it. Thus pleasure is a matter of one's mental attitude. The Buddha said that his mind was freed from the Taint of sense-pleasure (\emph{kāmāsavāpi cittaṁ vimuccitva}). Sight, sound, smell, taste and touch are the strands of sense-pleasure (\emph{kāmagunā}).} i.e., the pleasure connected with the senses. Wanting pleasure that arises in connection with one or more of the senses is called \emph{kāma-taṇhā}.

What is \emph{bhava}?

It is not possible to answer this question and indicate the meaning of the answer effectively and with sufficient clarity unless the question that was left unanswered in the first chapter is answered -- the question: As against the Five Grasping Groups could there be \textbf{just} the Five Groups?

The answer is: Yes.

The unique discovery that the Buddha made was just this: there could be Five Groups \textbf{without} Grasping. In other words, there could be an individual (i.e., as distinct from other individuals) who has no notion whatever of `self', `I' and `mine'. This individual is called Arahat (\emph{arahaṁ}). \textbf{The Arahat does not consider anything whatever as `mine'.} The Buddha experienced this state of affairs in himself. Thus he was the first Arahat to have appeared in the world in our time.

\textbf{Bhava} literally means `existence' or `being'. But existence of what? Being what? It refers to the \textbf{existence of `self'}, or to the notion \textbf{`I exist'}. Or we may say it refers to \textbf{being a `self'}, to \textbf{being a subject (`I')}. Or yet, to the \textbf{existence of subjectivity}. A life-mode completely devoid of all notions of `self' and of thoughts of `I' and `mine' will \textbf{not} be a \textbf{bhava}.

The \emph{puthujjana} looks upon his existence as `\textbf{my} existence'. He thinks `my self exists' or `I exist'. This looking upon one's existence as `my existence' or `existence of my self' is called having \emph{bhavadiṭṭhi}. When there are no notions of `self' no thoughts of `I' and `mine' there can be no such thing as `my existence' and so on.

To speak very precisely, \emph{bhava} is the existence of the notions `self' and `I'. Something is, of course, always identified as `self' and `I', which again, is one or more of the Five Grasping Groups. But \emph{bhava} is really a matter of one's thinking just as much as Grasping (\emph{upādāna}) is.\footnote{This should not lead the reader to think that since \emph{bhava} and \emph{upādāna} are really a matter of one's thinking they can be easily got rid of if necessary. If one completely gets rid of the thought `mine' so that it will never arise again, then one has become Arahat.}

It is the mind that is freed from the taint of \emph{bhava}, just as it is the mind that is freed from the taints of sense-pleasure and Ignorance. The Buddha said that his mind was freed from the taint of \emph{bhava}: `Thus knowing, thus seeing, my mind was freed from the taint of sense-pleasure, freed from the taint of \emph{bhava}, freed from the taint of Ignorance.' (\href{https://suttacentral.net/mn36/en/bodhi}{MN 36}) He described himself as `gone to the end of \emph{bhava}' -- \emph{bhavassa pāragu}. (\href{https://suttacentral.net/iti100/en/sujato}{Iti 100}) He did not say he was \textbf{going} to the end of \emph{bhava}. He said that he had already \textbf{gone} to the end of \emph{bhava}; which means he lives free from \emph{bhava}. In him \emph{bhava} has ceased. He is \emph{bhavanirodha}, because he is completely free from notions of `self' and from thoughts of `I' and `mine'. He does not look upon his existence as `\textbf{my} existence'. He does not think `\textbf{I} exist'. Certainly he uses the \textbf{words} `I' and `mine' for purposes of conversation. But that is all. They are expressions current in the world of which he makes use, but he is not at all affected by them.

The tendency to the conceit `I' and `mine' (\emph{ahaṅkāramamaṅkāramānānusaya}) is not always apparent, for it is not always on the surface. It lies deep-rooted and latent. It is something that one perceives only when one reflects upon it. One does not perceive it at all times even though it is present at all times lying behind one's thoughts and actions. In the Buddha, and in the Arahats, this tendency is completely uprooted, never to arise again. When all thoughts of `I' and `mine' are extinct and do not arise again, `my existence' or `I exist' are also extinct and do not arise again.

The Arahat `has gone beyond all \emph{bhava}' -- \emph{upaccagā sabbabhavāni}. (\href{https://suttacentral.net/ud3.10/en/sujato}{Uda 3.10}) The Arahat Maha Kassapa declared that he had `escaped from \emph{bhava}' -- \emph{bhavābhinissato}. (\href{https://suttacentral.net/thag18.1/en/sujato}{Thag 1089}) \emph{Bhava} ceases with the attainment of Arahatship. Thereafter the Arahat lives freed from \emph{bhava}, delivered from \emph{bhava}. Before the ascetic Gotama attained Arahatship all life that existed was \emph{bhava}. With his attaining Arahatship there came to be for the first time in the world a life free of \emph{bhava}.

\emph{Bhava} is also called a fetter. This fetter is completely cut off and destroyed in the Arahat. He is \emph{parikkhīnabhavasaññojano}. (\href{https://suttacentral.net/sn22.110/en/sujato}{SN 22.110})

If \emph{bhava} means `existence' pure and simple, then the living Arahat cannot have completely destroyed \emph{bhava}, for as a living Arahat there is an `existence'.\footnote{See \href{ch-08-impermanence.xml\#living-experience}{Chapter 8, Impermanence}: `The Not-Determined therefore is the \textbf{living experience of the Arahat}.'} The living Arahat has `existence', but no \emph{bhava}. He is \emph{bhava}-ceased. \emph{Bhava} refers to the existence of \textbf{all} modes of life \textbf{other than} that of the Arahats, simply because all such modes of life, save the Arahat's, are a case of `\textbf{my} existence' or `self'-existence to some degree or other.

If the precise meaning of \emph{bhava} is not understood there can be much confusion with regard to the Buddha's Teaching, particularly when it comes to the Doctrine of Dependent Arising (\emph{paṭiccasamuppāda}). The Teaching will then become either a matter of faith, which will remain beyond reach here, or a hypothesis left for future verification. But the Buddha's Teaching is neither a matter of faith nor a matter of hypothesis. It is a teaching to be experienced here and now, all of it, from beginning to end.

As with the word \emph{taṇhā}, there does not seem to be an exact equivalent in English for the word \emph{bhava}. This word is usually translated as `becoming' or `existence' -- all of which miss the point.\footnote{Sometimes \emph{bhava} is seen translated as rebirth! The extent to which the meaning of the Suttas (Discourses) is hidden from the reader by such inaccuracies can thus be seen.} We shall therefore keep to the Pali word \emph{bhava}.

Wanting \emph{bhava}, i.e., wanting `self'-existence or `my existence' is \emph{bhava-taṇhā}.

\emph{Vibhava} is not so straightforward as \emph{bhava}.

Some thinkers, in their search for truth, want to be quite certain of everything they take to be true. They begin by doubting everything, including their very existence. But by doubting their own existence they very cleverly deceive themselves for the simple reason that their very doubting proves their existence. To doubt also one must exist. \emph{Vibhava} stems out of this type of thinking. It is a denial of existence. Existence denies itself. But this denial of existence only leads to a confirmation of existence. It does \textbf{not} lead to \textbf{cessation} of existence.

Since, perhaps, it appears insane to deny existence whilst being existent, this tendency to denial is pushed back (\emph{atidhāvati}) to `after death', and in the following manner a denial of existence is made: `To the extent, revered Sir, that this self is of Form, is made up of the Four Primary Modes, is from the union of the parents, is cut off and destroyed on the dissolution of the body, and does not exist after death, to that extent is there a complete cutting off of the self.' (\href{https://suttacentral.net/dn1/en/bodhi}{DN 1})

This thinking, in the final analysis, only confirms `self'-existence. Thus \emph{vibhava} -- denial of `self'-existence -- only confirms \emph{bhava}, only confirms `self'-existence. It does not lead to \textbf{cessation} of `self'-existence. Just as \emph{bhava} is, \emph{vibhava} is also based on `self' and on thoughts of `I' and `mine'.

\begin{quote}
Those worthy recluses and brahmins who lay down for beings the cutting off, the destruction, the denial of \emph{bhava} (of `self'-existence), these, afraid of the `person' (\emph{sakkāya}), loathing the `person', simply keep running and circling round the `person'. Just as a dog that is tied by a leash to a strong post keeps running and circling round the post, so do these worthy recluses and brahmins, afraid of the `person', loathing the `person', simply keep running and circling round the `person'.

 -- \href{https://suttacentral.net/mn102/en/sujato}{MN 102}, The Five and Three
\end{quote}

Wanting the cutting off of `self'-existence and the destruction of `self'-existence at death is called \emph{vibhava-taṇhā}.\footnote{See \href{ch-99-appendix.xml\#vibhava-tanha}{Appendix on Vibhava-taṇhā}.} (It is an undesirability because such a destruction of `self'-existence cannot be got. `Self'-existence, or \emph{bhava}, can be destroyed only by following a particular training, i.e., by treading the Noble Eightfold Path.)

Just as much as grasping belief in `self' is the most fundamental of the four kinds of Grasping, wanting \emph{bhava} is the most fundamental of the three kinds of \emph{taṇhā}.

Thus the \emph{puthujjana}'s intentional action is determined by \emph{taṇhā}. `So, Ānanda, action is the field, Consciousness the seed, and \emph{taṇhā} the moisture.' (\href{https://suttacentral.net/an3.76/en/thanissaro}{AN 3.76}) Just as moisture must be present for the seed to sprout up out of the field, so must \emph{taṇhā} be there for the \emph{puthujjana}'s Consciousness to arise from his intentional action. \emph{Taṇhā} is one of the most powerful factors that go into the fashioning of one's life, yet it is one factor that can be brought under immediate control. The necessity to control \emph{taṇhā} cannot be overstressed if one is to progress. Hence the reason for the Buddha laying so much stress on it.

In the texts the words \emph{taṇhā}, \emph{chanda}, \emph{rāga} and \emph{nandi} often come together. There seems to be a tendency to consider the words \emph{chanda}, \emph{rāga} and \emph{nandi} as being almost identical with \emph{taṇhā}. They are not `various shades of \emph{taṇhā}'. They have their own meaning.

\emph{Chanda} means desire, \emph{rāga} means attachment, and \emph{nandi} means delight. Desire, attachment and delight are things dependent on \emph{taṇhā}. Were there no kind of wanting sense-pleasures or `self'-existence there can be no desire or attachment or delight.

\begin{quote}
Thus it is, Ānanda, that \emph{taṇhā} arises dependent on feeling, pursuit dependent on \emph{taṇhā}, gain dependent on pursuit decision dependent on gain, \textbf{desire} and \textbf{attachment} dependent on decision, tenacity dependent on desire and attachment, possession dependent on tenacity, avarice dependent on possession, watch and ward dependent on avarice, and many a bad and unskilled state of things such as blows and wounds, strife, contradiction and retort, quarrelling, slander and lies arise from keeping watch and ward.

 -- \href{https://suttacentral.net/dn15/en/bodhi}{DN 15}, The Great Discourse on Causation
\end{quote}

Desire (\emph{chanda}), attachment (\emph{rāga}) and delight (\emph{nandi}) have also been referred to as Grasping (\emph{upādāna}). `Friend, Visākha, that desire and attachment there is in the Five Grasping Groups, that there, is the Grasping.' (\href{https://suttacentral.net/mn44/en/sujato}{MN 44}) And, `Whatsoever there is delight in Feeling, that is Grasping.' (\href{https://suttacentral.net/mn38/en/bodhi}{MN 38}) This means to say that grasping something also means desiring of it, or being attached to it, or delighting in it. This is so because desiring, or being attached, or delighting, is \textbf{in effect} the same as regarding as `mine'. It is a matter of direct experience that when desire, attachment or delight exist `I' and `mine' also exist. It is only an `I' that can desire something or be attached to it or delight in it.

\emph{Taṇhā}, desire, attachment, delight, are all supports for \emph{bhava}. `\textbf{I} exist' or `\textbf{my} existence' stands supported by these. \emph{Bhava} hangs on these as its `cord'. They are called the `cord of \emph{bhava}' (\emph{bhavanetti}).

\begin{quote}
Whatever desire, attachment, delight, \emph{taṇhā}, whatever tendencies to determinations, attachments, and to the grasping of various means there are in the mind, Radha, towards Form \ldots{} Feeling \ldots{} Perception \ldots{} Determinations \ldots{} Consciousness, that is called the cord of \emph{bhava}. The cessation of these is the cessation of the cord of \emph{bhava}.

 -- \href{https://suttacentral.net/sn23.3/en/sujato}{SN 23.3}, The Conduit To Rebirth
\end{quote}

Just as a bunch of mangoes hanging by a stalk will fall down when the stalk is cut, so will \emph{bhava} disappear when the cord of \emph{bhava} is cut. The Buddha said that he stood with the cord of \emph{bhava} cut. Thus he stood freed from \emph{bhava}.

\begin{quote}
Just, monks, as when the stalk of a bunch of mangoes has been cut, all the mangoes that were hanging on that stalk go with it, just so, monks, the body of the Tathāgata stands with the cord that binds it to \emph{bhava} cut (\emph{ucchinnabhavanettiko}).

 -- \href{https://suttacentral.net/dn1/en/bodhi}{DN 1}, The All-embracing Net of Views
\end{quote}
