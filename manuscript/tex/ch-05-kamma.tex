\chapter{Kamma}

It is useful to discuss briefly the subject of intentional action once again.

\begin{quote}
Intention, monks, I declare is \emph{kamma}. Having intended, one does \emph{kamma} through body, speech, and mind.

 --- \href{https://suttacentral.net/an6.63/en/thanissaro}{AN 6.63}, A Penetrative Discourse
\end{quote}

This statement of the Buddha is not quite as simple as it is usually reckoned to be. Firstly, from the Sutta itself, it is clear that this statement was made in relation to the \textbf{non}-Arahat. The literal meaning of \emph{kamma} is `action'. With the \emph{puthujjana} it therefore refers to `\textbf{my} action' or `\textbf{I} act'. The word \emph{kamma} is used in this sense. Further expanded, \emph{kamma} means `\textbf{my} intentional action' or `the action \textbf{I} intentionally take'. And \textbf{all} action that is consciously done is intentional. This intentional action can be by means of body, speech, or mind.

Intentional action \textbf{unaccompanied} by thoughts of `I' or `mine' is \textbf{not kamma}. The Arahat has no thoughts of `I' and `mine'. Therefore the Arahat's intentional action is not \emph{kamma}. The Arahat has intentional action, but no \emph{kamma}. \textbf{Kamma is the non-Arahat's intentional action.} Of the Arahat the Buddha says: `He does not commit new \emph{kamma}.'\footnote{Of all the many things about \emph{kamma} the Buddha has taught, such as the various types of \emph{kamma} and their various fruits (\emph{vipāka}), this is the most important. It is also the most fundamental thing about it.}

Ethics is concerned with the question of `what should \textbf{I} do'. Whether that `what should be done by \textbf{me}' is good or bad, moral or immoral, etc., it is necessarily something which \textbf{I} should do.

Ethics accepts that `I' and `mine' must exist. It builds itself on the basis that `I' is a necessity. Ethics may or may not be conscious of its own position here. Nevertheless that remains its basic position. Actually, ethics is a searching after the most comfortable or the best way in which `I' can exist. But, as we shall see later on, `I' exists only in so far as Ignorance of the Four Noble Truths exists. Ethics does not know this fact. Thus, in the final analysis, ethics is a searching after the most comfortable and best way in which Ignorance can exist. It is therefore no wonder that no two schools of ethics are in agreement. Wherever there is Ignorance, there is conflict.

Where `I' and `mine' are wholly and entirely extinct, there the question of what should \textbf{I} do does not arise. Arahatship is \textbf{the extinction of ethics} also. Whilst all religions, in the end, teach an ethics of some kind or other. Buddha teaches the extinction of \textbf{all} ethics also.

The Buddha teaches the arising and ceasing of \emph{kamma} thus:

\begin{quote}
Monks, were there \emph{kamma} performed in lust, born of lust, conditioned by lust, arising from lust --- that \emph{kamma} is unskilful (\emph{akusala}), that \emph{kamma} is blameworthy, that \emph{kamma} has pain as fruit, that \emph{kamma} leads to the arising of (further) \emph{kamma}. That \emph{kamma} does not lead to the cessation of \emph{kamma}. Monks, were there \emph{kamma} performed in hatred \ldots{} performed in delusion \ldots{} that \emph{kamma} is unskilful, that \emph{kamma} is blameworthy, that \emph{kamma} has pain as fruit, that \emph{kamma} leads to the arising of \emph{kamma}. That \emph{kamma} does not lead to the ceasing of \emph{kamma}. These, monks, are the three conditions for the arising of \emph{kamma}.

Monks, were there \emph{kamma} performed with non-lust, conditioned by non-lust, arising from non-lust --- that \emph{kamma} is skilful (\emph{kusala}), that \emph{kamma} is praiseworthy, that \emph{kamma} has happiness as fruit, that \emph{kamma} leads to the ceasing of \emph{kamma}. That \emph{kamma} does not lead to the arising of \emph{kamma}. Monks, were there \emph{kamma} performed with non-hatred \ldots{} performed with non-delusion \ldots{} that \emph{kamma} is skilful, that \emph{kamma} is praiseworthy, that \emph{kamma} has happiness as fruit, that \emph{kamma} leads to the ceasing of \emph{kamma}. That \emph{kamma} does not lead to the arising of \emph{kamma}. These, monks, are the three conditions for the ceasing of \emph{kamma}.

 --- \href{https://suttacentral.net/an3.111/en/sujato}{AN 3.111}, Sources (1st)
\end{quote}

Summarized, the above means: unskilful \emph{kamma} leads to the arising of \emph{kamma}, and skilful \emph{kamma} leads to the cessation of \emph{kamma}. Or, unskilful intentional action accompanied by thoughts of `I' and `mine' leads to further intentional action accompanied by thoughts of `I' and `mine', and skilful intentional action accompanied by thoughts of `I' and `mine' leads to the cessation of intentional action accompanied by thoughts of `I' and `mine'. The Arahat not having any thoughts of `I' and `mine' whatsoever does \textbf{not} perform \textbf{either} skilful \textbf{or} unskilful \emph{kamma}.

The Buddha further teaches how \emph{kamma} rooted in lust, hatred and delusion leads to further \emph{kamma} whilst \emph{kamma} rooted in non-lust, non-hatred and non-delusion leads to the cessation of \emph{kamma}, thus:

\begin{quote}
Monks, there are these three conditions for the arising of \emph{kamma}. What three? Monks, for things which in the past were based on desire and attachment \ldots{} in the future will be based on desire and attachment \ldots{} in the present are based on desire and attachment, desire is born.

How, monks, is desire born for things which in the past were based on desire and attachment \ldots{} in the future will be based on desire and attachment \ldots{} in the present are based on desire and attachment? Monks, things which in the past \ldots{} in the future \ldots{} in the present are based on desire and attachment, one turns over in his mind. Thus turning over in his mind things which in the present are based on desire, desire is born. Desire being born, he is fettered by those things. I call it a fetter, monks --- that mind full of attachment. Thus, monks, is desire born for things which in the present are based on desire and attachment. These are the three conditions, monks, for the arising of \emph{kamma}.
\end{quote}

On the other hand:

\begin{quote}
How, monks, is desire not born for things which in the past were based on desire and attachment \ldots{} in the future will be based on desire and attachment \ldots{} in the present are based on desire and attachment? Monks, one understands the future result of things which in the present are based on desire. Seeing this result one turns away from them. Turning away from them, the mind getting detached from them, one penetrates them with wisdom and sees them plain. Thus, monks, is desire not born for those things which in the present are based on desire.

 --- \href{https://suttacentral.net/an3.112/en/sujato}{AN 3.112}, Sources (2nd)
\end{quote}
