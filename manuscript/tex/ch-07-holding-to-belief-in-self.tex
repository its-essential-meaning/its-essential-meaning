\chapter{Holding to Belief in Self- and Person-view}

With this we come to the problem of `self'. It is in fact the basic problem. It is also more difficult a problem than it is generally supposed to be. And if one is speaking of the Buddha's Teaching then fundamentally one is speaking of \textit{attavādupādāna} (holding to belief in `self') and \textit{asmimāna} (the conceit `I am'). For, a Teaching that is meant to lead on to a cutting off at the root that which is called \textit{upādāna} must necessarily have a great deal to do with the most fundamental of \textit{upādāna}, viz., \textit{attavādupādāna}. The Buddha's Teaching sets out to destroy Suffering, and this, as we shall see later on, is to destroy and uproot beliefs in `self' and thoughts of `I' and `mine', which again is to uproot \textit{upādāna}. `Uproot false view of seIf.'\footnote{\href{https://suttacentral.net/snp5.15/en/sujato}{Snp 5.15}, The Question of Posāla} With the uprooting and destruction of these false views other things follow.

The notion of `self-hood' is fundamentally a notion of \emph{mastery}\footnote{The Pali word is \textit{vasa}. See \href{https://suttacentral.net/mn35/en/sujato}{MN 35}.} over things, a notion of being able to wield power over things, which in the final analysis is a notion of mastery over Form, Feeling, Perception, Determinations and Consciousness. `I am master over this body, it is mine.' Or else, `I am master over my intentions, they are mine.' To own or appropriate a thing means to become master over it, to wield power over it. Moreover, I think that `I am' is something pleasant and pleasurable only when I feel or think I am permanently \emph{master} over my Form, Feeling, Perception, Determinations and Consciousness, and I have \emph{power} over them so as to make them become just what I want them to be. This feeling is the feeling of `self-hood', and with it I lull myself into a false sense of security.

Though in times of right mindfulness the \textit{puthujjana} may tend towards seeing impermanence, the \textit{puthujjana}'s reaction towards things is as if they were permanent. His actions are based upon such wrong view. Basing himself on this wrong view he intends and acts. Whatever mastery he possesses is very \emph{temporary} and very \emph{partial}. Impermanence undermines the mastery. And a mastery that is undermined by impermanence is certainly no mastery. In short, the assumed `self-hood' is no `self- hood' at all. `Self-hood' is therefore a \emph{deception}.

I really do not wield power or possess mastery over the Five Grasping Groups which I regard as \emph{my own}. I cannot say to my Consciousness: `Let my Consciousness be thus, let my Consciousness be not thus.' I cannot say to my body which is suffering from an ulcer, `Let my body be relieved of the ulcer' and so have my body relieved of the ulcer. I certainly wish that from this body which I regard as my own the ulcer would vanish. In fact I think that it should never have come at all. But however much I wish the ulcer to vanish it does not. Nor can I wield any such power over the other Groups which I regard as `mine'. I cannot say to the feelings which I regard as `mine', `Let my feelings be thus, let my feelings be not thus', and so have my feelings as I want them to be. Thus this `self-hood' which is adhered to is a deception, ever and again leading to betrayal, to disappointment. Betrayal and disappointment are the inevitable outcome of adhering to a deception.

Now, the \textit{puthujjana} has \textit{attavādupādāna}. That is to say, the \textit{puthujjana} holds (\textit{upādāna}) to belief (\textit{vāda}) in `self' (\textit{attā}).\footnote{Holding to belief in `self' essentially means holding to belief in a master.} Because he holds to this belief in `self' he keeps looking for something which he can \emph{identify} as this his `self'.\footnote{To `identify something as his ``self``\,' essentially means to identify something as that thing over which he is master.} If he is to keep believing there is a `self' then he \emph{must} at the same time regard something or other as this `self'. And if there is anything that he is led to identify as this his `self' it must pertain to the Five Grasping Groups. It must be one or more of these Groups. It is impossible for him to identify it with anything else. He therefore views one or more of the Groups as `self'.\footnote{To view the Groups as `self' essentially means to regard that `I am master over the Groups'. `The Groups are my self' means `I am master over my Groups'.} This means he has gone to wrong view. He has gone to the wrong view that one or more of his Groups is `self'. Having thus gone to a wrong view he elaborates on the view and formulates a distinction between himself and the rest thus: `The self, the world' (\textit{attā ca loko ca}). And he further keeps deliberating about himself (that is about the Five Grasping Groups regarded as `self') thus:

\begin{quote}
\begin{itemize}
\item
  `Was I in the past',
\item
  `was I not in the past',
\item
  `who was I in the past',
\item
  `how was I in the past',
\item
  `having been who in the past who have I come to be now',
\item
  `will I be in the future',
\item
  `will I not be in the future',
\item
  `who will I be in the future',
\item
  `how will I be in the future',
\item
  `being who in the future whom will I be again in the future'.
\end{itemize}

About his present existence too he begins to doubt:

\begin{itemize}
\item
  `Am I',
\item
  `am I not',
\item
  `who am I',
\item
  `how am I',
\item
  `from where has this being come',
\item
  `where is he going'.
\end{itemize}

Further, one or other of the following views arises in him as though it were real and true:

\begin{itemize}
\item
  `There is self for me',
\item
  `there is not self for me',
\item
  `by self I recognize self',
\item
  `by self I recognize not-self',
\item
  `by not-self I recognize self',
\item
  or `this my self which feels pleasant and unpleasant feelings, reaps the fruit of good and bad action, is permanent, steadfast, eternal, not transitory, stands unchanging as an eternal thing'.
\end{itemize}

 -- \href{https://suttacentral.net/mn2/en/bodhi}{MN 2}, All the Taints
\end{quote}

Thus he is gone to (wrong) view, or he is of (wrong) view (\textit{diṭṭhigata}).

All this deliberating about `self' is because he is attached to a belief in `self', because he has desire and passion towards `self'. If he does not hold to a belief in `self', these deliberations do not arise.

This regarding or viewing the Grasping Groups as `self' in some way or other is called the `person'-view (\textit{sakkāyadiṭṭhi}).

\begin{quote}
``But how, noble lady, is there the `person'-view?''

``Here, friend Visākha, the uninstructed \textit{puthujjana} not discerning the Noble Ones, not skilled in the Noble Doctrine, untrained in the Noble Doctrine, not discerning the Worthy Ones, not skilled in the Doctrine of the Worthy Ones, untrained in the Doctrine of the Worthy Ones, regards

\begin{itemize}
\item
  Form \ldots{} Feeling \ldots{} Perception \ldots{} Determinations \ldots{} Consciousness as `self',
\item
  or regards `self' as having Consciousness,
\item
  or regards Consciousness as being in `self',
\item
  or regards `self' as being in Consciousness.
\end{itemize}

Thus, friend Visākha, it is said there is the `person'-view.''

 -- \href{https://suttacentral.net/mn44/en/sujato}{MN 44}, The Shorter Classification
\end{quote}

But why say `\emph{person}'-view (\textit{sakkāyadiṭṭhi})?

\textit{Sakkāya} means `person', `somebody', a `self-existing being'.\footnote{It does not matter very much what word we use as the English equivalent of the Pali word \textit{sakkāya}. The fact is that whatever word we use to denote \textit{sakkāya} will equally baffle the individual who does not understand its meaning. What is needed is not so much a precise English equivalent for the word \textit{sakkāya} as much as understanding what it refers to.} To the \textit{puthujjana} he is himself a \textit{sakkāya}, i.e., a `person', a `somebody', a collection of Five Grasping Groups which regards itself as master over itself. To be precise the Five Grasping Groups takes itself to be a \textit{sakkāya}.

Another word is \textit{satta}. The \textit{puthujjana} takes himself to be a \textit{satta}. \textit{Satta} or \textit{sakkāya} refers to the sentient being regarded in some way or other as `self'. That is, it refers to the Five Grasping Groups taken to be `self'. It is the \textit{puthujjana}'s concept of the sentient being. It is his concept of himself. That is why the Five Grasping Groups are called \textit{sakkāya}.\footnote{`What, monks, is the \textit{sakkāya}? The Five Grasping Groups are to be so called.' (\href{https://suttacentral.net/sn22.105/en/sujato}{SN 22.105})} To have this concept means to be gone to `person'-view.

Again, the Five Grasping Groups looks upon itself as `self' so long as it contains \textit{attavādupādāna}, i.e., so long as it holds to belief in `self'. \textit{Sakkāya} incorporates \textit{sakkāya-diṭṭhi}; that is to say, `person' contains `person'-view.\footnote{This statement is not fully applicable to the \textit{sotāpanna}, and higher \textit{sekhas}. To the extent that thoughts of `I' and `mine' and the deception `self' arise in them, they are still \textit{sakkāya}. But they know that regarding anything as `I am' or as `mine' or as `self' is wrong. Therefore they do not hold to any belief in `self'. Thus they have no \textit{attavādupādāna}, and to that extent have no \textit{sakkāyadiṭṭhi} also. See also \href{ch-16-satipatthana.xml\#truth-for-him}{Chapter 16, Four Applications of Mindfulness}: `The \textit{Satipatthānā Sutta} assumes a prior understanding of the Buddha's Teaching. {[}\ldots\hspace{0pt}} though these fundamentals and their resultant implications are very difficult to \emph{see}, they edify him who sees them. They are truth \emph{for him}.'{]} It should be noted that \textit{sakkāyadiṭṭhi} is not a question of just passively viewing oneself as a \textit{sakkāya} in a rather detached manner. It is much more dynamic and intense a matter, deeply rooted. Hence the difficulty in getting rid of it.

\textit{Sakkāyadiṭṭhi} should not be identified purely and simply with `the view that \emph{in} the Five Grasping Groups there is a self' or with `the belief in a self or soul'. Regarding one or more of the Five Grasping Groups as `self' in some way or other is different to purely and simply regarding the Five Grasping Groups as having a `self' \emph{in} them somewhere or other. The person who mistakes \textit{sakkāyadiṭṭhi} to mean purely and simply `the view that in the Five Grasping Groups there is a self' can very effectively impede his own progress and even think he is an \textit{ariya} (Noble One) whilst he is not.

After a masterly analysis of the Five Grasping Groups, perhaps with the assistance of modern science, he finds no self-existing thing in it. Thus quite honestly he comes to the conclusion that there is no self in the Five Grasping Groups, and so he thinks he has no \textit{sakkāyadiṭṭhi}, which means he now thinks he is a \textit{sotāpanna},\footnote{See Chapter XV for definition of the \textit{sotāpanna}. At this stage it would be sufficient to know that the \textit{sotāpanna} is not a \textit{puthujjana} and that he is therefore an \textit{ariya}, i.e. he is a Noble.} whilst in truth he really is not.

The Five Grasping Groups constantly recognizes itself as `self'. It is its very nature. And the apparent `self', or that which appears as `self' is taken as it appears and is identified as `self'.

\textit{Sakkāyadiṭṭhi} is a determined thing (\textit{saṅkhata dhamma}), because it has come about with \textit{attavādupādāna} as necessary condition. Here, \textit{attavādupādana} is a \textit{saṅkhāra}. As a \textit{saṅkhāra} it is the necessary condition for \textit{sakkāyadiṭṭhi}. Without \textit{attavādupādāna} there can be no \textit{sakkāyadiṭṭhi}. Because the \textit{puthujjana} holds to belief in `self' he views the Five Grasping Groups (or one or more of them) as this `self' which he believes in.

On the other hand, if there is no holding to belief in `self', then there can be no \textit{sakkāyadiṭṭhi}, because then no identification or regarding of anything as `self' will arise. The \textit{puthujjana} does not see this. He does not see that his \textit{sakkāyadiṭṭhi} is dependent on a \textit{saṅkhāra} and that all \textit{saṅkhāras} are \emph{impermanent}. But if he sees that the \textit{saṅkhāra} called holding to belief in `self' (\textit{attavādupādāna}) is impermanent then the \textit{saṅkhāra} will cease, and he will no longer be deceived into believing in any `self'. When \textit{attavādupādāna} ceases his identification of the sentient being as self ceases, which means \textit{sakkāyadiṭṭhi} ceases and he ceases to be a \textit{puthujjana}. He has then crossed from the plane of the \textit{puthujjana} (\textit{puthujjana bhūmi}) to the plane of the Noble (\textit{ariya bhūmi}).

Of the three notions `This is mine, this am I, this is my self', the most fundamental one is `\emph{this is mine}'. In the Discourse on \textit{The Fundamentals of All Things} (Majjhima Nikāya 1) the Buddha narrates at length the many things that the \textit{puthujjana} takes to be `mine'. He does \emph{not} include the other two notions of `I' and `self' at all in this Discourse.

Further, in the \textit{Ānanda Sutta} we have the following:

\begin{quote}
By grasping Form is there `I am', not by not-grasping (\textit{rūpam upādāya asmīti hoti no anupādāya}). By grasping Feeling \ldots{} Perception \ldots{} Determinations \ldots{} Consciousness is there `I am', not by not-grasping.

 -- \href{https://suttacentral.net/sn22.83/en/bodhi}{SN 22.83}, Ānanda
\end{quote}

This too indicates that `mine' (which is essentially the same as what has been referred to in the Sutta as grasping) is more fundamental than `I', and that for `I' to be present `mine' must be present.

It is of great \emph{practical} importance to see that `mine' is the most fundamental of these three notions `mine', `I' and `self'. The \textit{puthujjana}'s constant thinking is a thinking that something is \emph{his}. In fact there is nothing more fundamental than this about his experience. And he must seek to understand this state of affairs in his own experience itself. The notions `I' and `self' do not take the same stature as the notion `mine'. When he, the \textit{puthujjana}, is conscious of a feeling, he is always conscious of it as \emph{my} feeling. It is this consideration `mine' that leads the \textit{puthujjana} on.

The \textit{puthujjana}, however, works with the assumption that the fundamental is `I' and not `mine'. Since \emph{he} exists, he thinks things are \emph{his}. '\emph{Since ``I``} exist, things are \emph{mine}.' But the fundamental condition, the Buddha points out, is `mine'. The \textit{puthujjana} having Grasping Consciousness, things \emph{present} themselves to him as \emph{`mine'}. And this state of affairs further \emph{points} to a subject to \emph{whom} they are present. That is, they point to an `I'. The correct position is therefore: \emph{Since things are `mine', `I' exist.}

The \textit{puthujjana} then begins to wonder what precisely this `I' is. He begins to reflect upon the `I'. And when he so reflects he sees a `self'; that is to say, he sees a mastery over things. A `self' appears before him as he reflects, just as `water' appears to the deer when it gazes upon the sun shining on the sand. `Mine' being present all the time, this `self' also appears as `\emph{my} self'.

Finally, the \textit{puthujjana} -- holding to belief in `self' all the time -- tries to identify this `self'. But he can identify it with nothing else other than one or more of the Five Grasping Groups. He therefore proceeds to regard or view one or more of the Groups as `self' -- more precisely, as `my self'. He thinks `The Groups are myself', meaning fundamentally, `I am master over my Groups'. Thus he has \textit{sakkāyadiṭṭhi}.

The notion of `self' is secondary to `mine' and `I'. It is like a coarse layer that lies over the conceit `I am'. Before getting rid of the conceit `I am' (\textit{asmimāna}), holding to belief in `self' is got rid of. The Ariyan disciple (who is a \textit{sotāpanna}), seeing fully well how \textit{sakkāyadiṭṭhi} arises, has got rid of it. That is to say, he no longer regards anything as `self'. But until he becomes Arahat the subtle conceit `I' still remains in him. It is only the Arahat who is utterly freed of `I' and `mine' too.
