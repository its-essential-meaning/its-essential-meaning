\chapter{Impermanence}

Invariably, one imagines all too soon that one understands and perceives the Buddha's doctrine of Impermanence \emph{(aniccatā)}. But the impermanence that the Buddha teaches is not the impermanence that one sees around oneself. It is something far more subtle than that.

The meaning of the word \emph{anicca (a-nicca)} is not-permanent. It says nothing more.

In reflection, the thinker is not averse to accepting that things are not-permanent or not-eternal or not-everlasting. He sees most things passing away, at least after some time. And if he thinks he will not live long enough to see a particular thing pass away he contents himself by inferring that it will pass away some time in the future, somehow. Thus to a large extent he avoids falling into the one extreme of \authoremph{eternal} existence. And if at all there be anything that shall remain permanent it may be his own `self'!

But herein lies the difficulty. For, as he moves himself away from the extreme of eternal-existence he falls into the other extreme of \authoremph{no}-existence without realising that he is actually falling there. He falls from the extreme of \authoremph{eternal} duration to the extreme of \authoremph{no} duration. The usual way in which this happens is by assuming that things are \authoremph{becoming from moment to moment}. If he is questioned as to what a moment is, he will reply that it is the shortest possible time.

But the shortest possible time is \authoremph{no} time. Thus, his thinking that a thing exists for only a moment when critically analysed means that the thing exists for \authoremph{no} time, which only means that the thing \authoremph{does not exist at all}.

But if his definition of moment means \authoremph{some} duration of time however small it be, then what he really means is that the thing exists or persists \authoremph{without change for some time or other}. That means there is a \authoremph{temporary persistence}.

Temporary persistence is not rejected by not-permanent. A thing can be not-permanent but yet \authoremph{exist without change for some time}. It lasts \authoremph{for some time} though \authoremph{not for ever}. Therefore, \authoremph{`nothing endures absolutely for ever, and nothing is absolutely without duration.'} In other words, it means that `between its appearance and disappearance a thing endures.'

Actually changes go on at various levels of generality. A table, for instance, remains a table even though its components are changing. Though a little part of it may have changed and even disappeared, yet the table remains a table. And it will remain a table until changes have developed to the point at which the table is no more. `A thing \authoremph{remains the same} means it has become the \authoremph{invariant of a transformation}.'

\clearpage

Now, the Buddha teaches:

\begin{quote}
Monks, there are these three Determined-characteristics of the Determined \emph{(saṅkhata)}. What three? Arising is to be discerned, passing away is to be discerned, otherwise-ness in persistence is to be discerned. These are the three Determined-characteristics of the \emph{Determined}.

Monks, there are these three Not-Determined-characteristics of the Not-Determined \emph{(asaṅkhata)}. What three? Arising is not to be discerned, passing away is not to be discerned, otherwise-ness in persistence is not to be discerned. These are the three Not-Determined-characteristics of the Not-Determined.

 -- \href{https://suttacentral.net/an3.47/en/bodhi}{AN 3.47}, Conditioned
\end{quote}

In understanding the above we must remember what the Buddha refers to as the Determined \emph{(saṅkhata)} and the Not-Determined \emph{(asaṅkhata)}. It is very easy for us to assume that what the Buddha taught can be applied in full to each and every thing that lies outside the domain of his Teaching. But it is very dangerous. We must know always precisely what he is referring to, and avoid stretching the limits in our own imagination. He specifically said that he teaches only one thing always, i.e., Suffering and the cessation of Suffering. `Formerly, and now also, Anurādha, it is just Suffering and the cessation of Suffering that I proclaim.' (\href{https://suttacentral.net/sn44.2/en/sujato}{SN~44.2}) He also said that his Teaching has the taste of Deliverance right through. `Just as the great ocean, Paharada, has but one taste, the taste of salt, even so, Paharada, this Dhamma and Discipline has but one taste, the taste of Deliverance.' (\href{https://suttacentral.net/an8.19/en/bodhi}{AN~8.19}) The words Determined and Not-Determined are also used by him in relation to that \authoremph{one} thing he teaches -- Suffering and the cessation of Suffering.

What now is the Determined?

As the \emph{Khajjanīya Sutta} tells us (\href{https://suttacentral.net/sn22.79/en/bodhi}{SN 22.79}), it is the Five Grasping Groups. This means that it refers to the `person', or to `self', or to the subject `I'. These are all determined things. Hence they are called the Determined \emph{(saṅkhata)}. Of all the things that have been determined the most important thing is `my self', and it is precisely in \authoremph{this} that the problem of Suffering lies. The Buddha is not teaching anything other than about this problem either.

Let us now see how the three characteristics of the Determined as taught by the Buddha apply to the Determined, particularly the characteristic which he refers to as `otherwise-ness in persistence' \emph{(ṭhitassa aññathattaṁ)}.

\emph{Ṭhitassa aññathattaṁ} means otherwise-ness \emph{(aññathattaṁ)} in persistence \emph{(ṭhitassa)}. It is commonly thought that this refers to decay \emph{(jarā)}. But it does not refer to decay. It refers to something much more fundamental than decay. Appearance \emph{(uppāda)}, disappearance \emph{(vayo)}, and otherwise-ness in persistence \emph{(ṭhitassa aññathattaṁ)} are three characteristics that are fundamental to all Grasping Groups at \authoremph{all} times and not merely at time of decay. The essential idea in the word `persistence' is really permanence or unchange. A thing persists for some time means it exists without change for some time.

Firstly, what is it that is seen to persist?

It is `self'. It is `I'. An apparent `self' is seen to persist. There is a persistence of this apparent `self'.

`I' am thinking, or `I' am eating, or `I' am writing. Or, `my self' is thinking, or `my self' is eating, or `my self' is writing. Now, whether `my self' is writing or doing something else, the `my self' is seen to persist. Though the actions of the `I' and the appropriations of the `mine' are varying and changing, the `self'-ness and the `I'-ness is seen to persist. In other words, the subjectivity is seen to persist. Now, this `self' is always identified as something. There is something that is always taken to be this `self'. And that is one or more of the Five Grasping Groups. But whilst this `self' is persisting, that which is taken as this `self' is becoming otherwise, is under-going transformation or change all the time.

The persisting `self' is equated to the Grasping Groups which are becoming otherwise. Thus we get an `otherwise-ness in persistence', a characteristic which is discernible from the time of appearance up to the time of disappearance. And for the Five Grasping Groups appearance is synonymous with birth whilst disappearance is synonymous with death. This is somewhat similar to the principle called `Invariance under Transformation' which occurs in Quantum Mechanics and Relativity Theory. The invariant with regard to the Five Grasping Groups is `self'.

In this persistence of an apparent `self' lies the reason for Religion to assume the actual existence of an immortal self called a `soul'. This is the connection between `self' and `soul' that is of any worthwhile interest. Religion makes this assumption because though it sees a persistence in `self' it does not see the conditions that keep this `self' persisting and therewith also does not see its destruction.

\clearpage

What now is the Not-Determined? The Not-Determined is defined as follows:

\begin{quote}
The destruction of lust, the destruction of hatred, the destruction of delusion -- this, monks, is called the Not-Determined \emph{(asaṅkhata)}.

 -- \href{https://suttacentral.net/sn43.12/en/bodhi}{SN 43.12}, The Unconditioned
\end{quote}

The Path leading to the Not-Determined is further defined as the Noble Eightfold Path.

Then we have Arahatship also defined as the destruction of lust and of hatred and of delusion:

\begin{quote}
The destruction of lust, the destruction of hatred, the destruction of delusion -- this, friend, is called Arahatship.

 -- \href{https://suttacentral.net/sn38.2/en/sujato}{SN 38.2}, A Question About Perfection
\end{quote}

\protect\hypertarget{living-experience}{}{}Thus, the Not-Determined is synonymous with Arahatship. The Not-Determined therefore is the \authoremph{living experience of the Arahat}. He has trod the Path leading to the Not-Determined, has arrived at the Not-Determined, and is now living experiencing the Not-Determined.

Now, actually and in truth, there is no `Arahat' to be found.

\enlargethispage*{\baselineskip}

\begin{quote}
Anurādha, the \emph{Tathāgata},\footnote{Tathāgata refers to the Buddha.} actually and in truth, is here not to be found.

\emph{Anurādha, diṭṭheva dhamme saccato thetato Tathāgate anupalabbhyamāne.}

 -- \href{https://suttacentral.net/sn44.2/en/sujato}{SN 44.2}, Anurādha Sutta
\end{quote}

The reason for this is that there is no `self' or `I' and `mine' existing with the Buddha. As with the Buddha, so with the Arahats. Though we use the word `Arahat' for purposes of conversation there is no `person' called an Arahat. No `person' who says `I am Arahat' or `this Arahatness is mine.' We can distinguish an Arahat from another Arahat as two different individuals. But with regard to the Arahat there is no `person' or `somebody' or `self' who says `I' and `mine'.

\begin{quote}
The Arahat intentionally acts, but the acting is quite unaccompanied by any thought of a subject who is acting. For all non-Arahats such thoughts (in varying degrees, of course) do arise. The Arahat remains an individual (i.e., distinct from other individuals), but is no longer a \authoremph{person} (i.e. a somebody, a `self', a subject). This is not, as one might perhaps be tempted to think, a distinction without a difference. It is a genuine distinction, a very difficult distinction, but a distinction that \authoremph{must} be made.

 -- Ñāṇavīra Thera, in a letter to the author
\end{quote}

It is \authoremph{the} distinction that has to be seen.\footnote{The ordinary man cannot distinguish between individuality and `person'-ality. To him, there is always only a `person'-ality, and individuality is identical with it. The Arahat is an individual \emph{(puggala)} in that there is distinct set of Five Groups as separate from another set, but there being no Grasping, he is not a `person' \emph{(sakkāya)}.}

The difference between life-action and the action of inanimate things is the presence of intentionality in life-action. Intention is present only in life, and it is present in \authoremph{all} life whether Arahat or non-Arahat. The Buddha teaches that all life, save that of the Arahat, has Grasping also. Thus for the non-Arahat there is both intention and Grasping, whilst for the Arahat there is intention but \authoremph{no} Grasping.

Grasping, as mentioned earlier, is essentially subjectivity (`self', `I' and `mine'). The subjectivity, to some degree or other, is present in all life except that of the Arahat. Thus again, all non-Arahats have both intention and subjectivity, whilst the Arahat has intention but no subjectivity. All life before the advent of the Buddha (i.e. before the ascetic Gotama became Arahat) was a case of intention together with subjectivity. The Buddha, in his own being, discovered that there could be intention but no subjectivity -- a difficult thing indeed to see. It is also so difficult a thing to achieve that nothing short of the Noble Eightfold Path can take one there.

If the ordinary man is told there can be intentionality without subjectivity, i.e., that there can be intentional action completely unaccompanied by any thoughts of `I', he will invariably say that this is impossible. But it is precisely this `impossibility' that the Buddha discovered and made a possibility. It is essentially in this that he stands unique.

There is an Arahat-ness that is being experienced which we refer to as the `Arahat's life' or the `living experience of the Arahat'. That is all. But no `person' or `self' with regard to the Arahat is to be found. And that means no `person' or `self' is determined. That is why Arahat-ness is referred to as the Not-Determined, i.e., as \emph{asaṅkhata}. Being Not-Determined, there can be no appearance, no disappearance, and no otherwise-ness in persistence.

In teaching Suffering and the cessation of Suffering, the Buddha teaches the \emph{saṅkhata} and the \emph{asaṅkhata}. \emph{Saṅkhata} refers to the `person' \emph{(sakkāya)} which is a Suffering, and \emph{asaṅkhata} refers to the Arahat, which is the cessation of the `person' \emph{(sakkāyanirodha)} or the cessation of Suffering.

\sectionBreak

Be it again noted that the problem of `self' \emph{(attā)} is of considerably greater difficulty than it is generally supposed to be. So are the problems of Impermanence \emph{(anicca)} and Suffering \emph{(dukkha)}.

`Self' is not an indefiniteness. It is a \authoremph{deception}, and a deception (a mirage, for example) can be as definite as one pleases. The only thing is, that it is \authoremph{not} what one takes it for. When the sun shines on the sand there is the \authoremph{appearance} of water. I am thus \authoremph{deceived} to take the phenomenon as water. The \authoremph{deception} of water \authoremph{is} there all right, though the phenomenon is \authoremph{not}-water. I am only \authoremph{deceived} in thinking that it is water. To understand the phenomenon of the sun shining on the sand I must realize that it is not-water. So is it with `self'. The deception of `self' is there. I must understand that the phenomenon I take to be `self' is Not-self \emph{(anattā)}. The Five Grasping Groups are taken to be `self' though in truth they are not. I must therefore see that the Five Grasping Groups are Not-self.

To make an assertion, positive or negative, about `water' with regard to the sun shining on the sand is to work accepting falsity at face value. To say `the water exists' or `the water does not exist' is to base one's statement on the wrong premise `water'. Likewise to make an assertion, positive or negative, about `self' is to work accepting falsity at face value. For this reason the Buddha refrains \authoremph{both} from asserting \authoremph{and} from denying the existence of `self' when Vacchagotta questioned him as to whether `self' exists or does not exist.

To have answered Vacchagotta categorically that `self' does exist or that `self' does not exist would have been unwise. For the fact is that whilst no actual self is to be found there yet \authoremph{is} a \authoremph{deception} of a `self' to be found. What a person who asks such direct questions about a deception should be given are not direct answers of `yes' or `no', but \authoremph{proper instruction}.

`Self' is always something very ambiguous to the \emph{puthujjana}. He always feels there is a self, but whenever he tries to get hold of it or spot it he fails. The deer thinks there is water when the sun shines on the sand and produces the \authoremph{mirage} of water. But when the deer runs after the `water' the water eludes him.

If the deer is told, `There is water', it will reply, `But I cannot find water however much I run after it.' If on the other hand the deer is told. `There is no water', it will reply, `But I see water however much you say no.' The \emph{puthujjana} is in the same dilemma with regard to his `self'. If he is told, `There is no self for you', he will say, `But I see a self'. On the other hand if he is told, `There is a self for you', he will say, `But I cannot find precisely where or what it is'. And that would have been just the position Vacchagotta would have fallen into had the Buddha given him direct answers to his questions either in the affirmative or in the negative. To the \emph{puthujjana} a `self' always \authoremph{appears}, but never does he find it when he tries to.

What the Buddha said was: `All things are Not-self' (\emph{sabbe dhammā anattā}, \href{https://suttacentral.net/mn35/en/sujato}{MN 35}). It simply means that no thing is self, or that if you look for a self you will not find one. `Self' is a deception, like a mirage. It does not mean that the mirage, as such, does not exist. The mirage \authoremph{does} exist. And it keeps persisting. It keeps persisting as `\authoremph{my} self' which is distinct from all other things. In its persistence there is a distinctiveness to be seen, a being different to all other things -- `the self, the world' \emph{(attā ca loko ca)}.

Impermanence \emph{(aniccatā)} is seen in its essential and effective meaning, and is seen \authoremph{for certain}, only when Not-Self-ness \emph{(anattatā)} is also seen and recognized, simply because one thinks that whatever else in the world is impermanent one's `self' is permanent. Everything to the seer is impermanent except the seer himself! What after all is the significance of Impermanence if it does not apply to the \authoremph{one} thing that matters to me -- my `self'?

It is only when a person sees that this last bastion of permanency, viz., his `self', is nothing but a deception or mirage which will pass away when the conditions that keep it going are removed, that he really and truly gets the impact of Impermanence. It is \authoremph{only then} that he sees that \authoremph{all} (which, for him, is nothing more than his Five Grasping Groups) is impermanent. Then only does he have perception of Impermanence.
