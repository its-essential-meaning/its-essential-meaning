\chapter{All Things Are Not Self}

The Buddha shows that a thing is impermanent by showing that the necessary condition (or the \emph{saṅkhāra}) upon which the thing depends is impermanent.

This can be seen, for instance, in the \emph{Pārileyya Sutta}. The destruction of the taints is spoken of, and the following passage occurs:

\begin{quote}
Monks, how knowing, how seeing, is there without delay the destruction of the taints? Here, monks, the uninstructed \emph{puthujjana} not discerning the Noble Ones, unskilled in the Noble Doctrine, untrained in the Noble Doctrine, not discerning the Worthy Ones, not skilled in the Doctrine of the Worthy Ones, regards Form as `self'. This regarding, monks, is the \emph{saṅkhāra}.

This \emph{saṅkhāra}, how does it result, how does it arise, how is it born, and how is it produced? In the uninstructed \emph{puthujjana}, monks, nourished by feeling that is born from Contact with Ignorance, arises \emph{taṅhā}. Thence is born that \emph{saṅkhāra}. Thus, monks, that \emph{saṅkhāra} is impermanent, determined, dependently arisen. That \emph{taṅhā} is impermanent, determined, dependently arisen. That Contact is impermanent, is determined, is dependently arisen. That Ignorance is impermanent, is determined, is dependently arisen. Thus knowing, monks, thus seeing, is there without delay the destruction of the taints.

-- \href{https://suttacentral.net/sn22.81/en/bodhi}{SN 22.81}, At Pārileyya
\end{quote}

And so with the other Groups, Feeling, Perception, Determinations and Consciousness.

Here, the taints are the things \emph{(dhammā)} considered. The necessary condition for the taints is the regarding of Form (or the other Groups) as `self'. That is, regarding the Groups as `self' is the \emph{saṅkhāra} for the taints. Thus we have: `This regarding, monks, is the \emph{saṅkhāra}.' The Buddha then goes on to show that the \emph{saṅkhāra} on which the taints depend is impermanent by showing that this \emph{saṅkhāra} in turn depends upon certain other conditions for its own existence. So we have, `Thus, monks, that \emph{saṅkhāra} impermanent, is determined, dependently arisen.' When the impermanence of the \emph{saṅkhāra} called `regarding the groups as ``self''\,' is seen, then the impermanence of the taints which depend on this \emph{saṅkhāra} is seen, and hence the possibility of their destruction is seen.

In the above, the Buddha does not directly say that the taints are impermanent. He indicates the fact in an indirect manner. He shows that the taints are impermanent by showing that the \emph{saṅkhāra} which forms the necessary condition for the taints to exist is impermanent. I will stop regarding the house I live in as permanent only if I see that the constituents which form the necessary conditions for the house, i.e. the foundation, the walls, the roof, etc., are impermanent. My being merely told that the house is impermanent does not convince me of its impermanence. That the house is impermanent is seen by me only by my seeing that its constituent factors are impermanent. When the constituent factors that go to make up the house are impermanent then the house must necessarily be impermanent:

\begin{quote}
Whatever cause, whatever condition there be for the arising of Form \ldots\hspace{0pt} Feeling \ldots\hspace{0pt} Perception \ldots\hspace{0pt} Determinations \ldots\hspace{0pt} Consciousness, that is impermanent. How can, monks, consciousness that is so composed of impermanent things be permanent?

 -- \href{https://suttacentral.net/sn22.18/en/bodhi}{SN 22.18}, Impermanent with Cause
\end{quote}

With regard to the other two characteristics of the Grasping Groups, viz., Not-self and Suffering, the same applies. When the \emph{saṅkhāra} are Not-self and Suffering those things determined by the \emph{saṅkhāra} are also Not-self and Suffering.

The Buddha shows the \emph{puthujjana} that whatever thing \emph{(dhamma)} he identities as `self' is something that is dependent upon other things. In other words, he shows the \emph{puthujjana} that the latter's `self' is a determined thing dependent upon Determinations, upon \emph{saṅkhāra}. He further shows that these \emph{saṅkhāra} which form the necessary conditions for that thing identified as `self' are impermanent. `All \emph{saṅkhāra} are impermanent \emph{(sabbe saṅkhārā aniccā)}'. (\href{https://suttacentral.net/mn35/en/sujato}{MN 35})

Now, when he sees that the \emph{saṅkhāra} upon which his `self' depends are impermanent, then he sees that this his `self' must also be necessarily impermanent, and hence not worth holding to. That means he is now left with a `self' that is impermanent and not worth holding to. (He finds that he has no real mastery in the face of this impermanence.) And if it is impermanent and not worth holding to, then it contradicts the very concept of `self'. This means that what he had identified as `self' is now no longer self. The thing \emph{(dhamma)} which he had regarded as `self', he now finds is Not-self \emph{(anattā)}. Thus: `All things are Not-Self' \emph{(sabbe dhammā anattā)}.

Therefore, when `All \emph{saṅkhāra} are impermanent' is seen `All things are Not-self' and `All \emph{saṅkhāra} are Suffering'\footnote{See \href{ch-11-suffering.xml\#impermanent}{Chapter 11: Suffering, `\,``Self'' always implies permanency\ldots\hspace{0pt}'}} are also seen. These three stand together, and fall together. When there is perception of Impermanence there is simultaneously perception of Not-self and perception of Suffering. When perception of Impermanence is not there, there is also no perception of Not-self and no perception of Suffering.
