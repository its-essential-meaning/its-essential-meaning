\chapter{Birth, Decay and Death}

Prince Siddhartha left his palace for no other reason than to find the solution to the problem of birth, decay and death, to determine whether he could get beyond these phenomena. All these were nothing but Suffering. And as Gotama the Buddha he claimed he had won the not-born, the not-decaying and the not-dying.

\begin{quote}
So I, monks, being liable to birth because of `self' (\emph{attanā}), having known the peril in what is liable to birth, seeking the non-born, the uttermost security from the bonds -- Nibbāna -- won the not-born, the uttermost security from the bonds -- Nibbāna. Being liable to decay because of `self' \ldots\hspace{0pt} won the not-decaying, the uttermost security from the bonds -- Nibbāna. Being liable to disease because of `self' \ldots\hspace{0pt} won the not-diseasing, the uttermost security from the bonds -- Nibbāna. Being liable to death because of `self' \ldots\hspace{0pt} won the not-dying, the uttermost security from the bonds -- Nibbāna. Being liable to sorrow because of `self' \ldots\hspace{0pt} won the not-sorrowing, the uttermost security from the bonds -- Nibbāna. Being liable to stain because of `self' \ldots\hspace{0pt} won the stainless, the uttermost security from the bonds -- Nibbāna. Knowledge and vision arose in me: `Unshakeable is my Deliverance; this is the end of birth; there is no \emph{bhava} again now.'

 -- \href{https://suttacentral.net/mn26/en/bodhi}{MN 26}, The Noble Search
\end{quote}

But the \emph{puthujjana} sees the Buddha `decaying' and `dying' in the same manner he sees others. Nevertheless the Buddha claimed he had arrived at and experiences the not-decaying and the not-dying. And the Buddha was the first individual in the world who made this claim. Subsequently, of course, those who followed his instructions to the very end and became Arahats, also made the same claim.

How are we to understand this?

The understanding of this must obviously lie in the understanding of the phenomenon of birth, decay and death.

The definition of birth, decay and death given by the Buddha himself is as follows:

\begin{quote}
And what monks, is birth?

That which, of this and that \textbf{being} in this and that group of \textbf{beings}, is birth, production, descent, arising coming forth, the appearance of the Groups, acquiring of the sense-bases -- this is called birth.

And what, monks is decay and death?

That which, of this and that being in this and that group of beings, is decay, decrepitude, breaking up, hoariness, wrinkling of the skin, shrinkage of life-span, over-ripeness of faculties -- this is called decay. That which, of this and that being in this and that group of beings, is falling, separation, breaking up, disappearance, mortality, death, completion of time, breaking up of the Groups, laying down of the body, cutting off of the living senses -- this is called death. Thus it is, this decay and this dying that is called decay and death.

 -- \href{https://suttacentral.net/sn12.2/en/bodhi}{SN 12.2}, Analysis of Dependent Origination
\end{quote}

Now, birth, decay and death in the above are referred to in relation to `beings'. The Pali word is \emph{satta}. \emph{Satta} (being) is defined for us as follows:

\begin{quote}
``\,'Being! being! (\emph{satta})' it is said. To what extent, Lord, is one called a being?''

``That desire, Radha, that attachment, that delight, that \emph{taṇhā} which is concerned with Form \ldots\hspace{0pt} Feeling \ldots\hspace{0pt} Perception \ldots\hspace{0pt} Determinations \ldots\hspace{0pt} Consciousness -- entangled thereby, fast entangled thereby, \textbf{therefore} is one called a `being'.''

 -- \href{https://suttacentral.net/sn23.2/en/sujato}{SN 23.2}, Sentient Beings
\end{quote}

Very clearly, \emph{satta} (being) refers to the Five \emph{Grasping} Groups. For, as we have seen earlier, desire, attachment, delight and \emph{taṇhā} are present only in the Grasping Groups.

Thus, birth (\emph{jāti}) means \textbf{birth of the Grasping Groups}. Decay (\emph{jarā}) means \textbf{decay of the Grasping Groups}. Death (\emph{marana}) means \textbf{death of the Grasping Groups}. Just as much as \emph{bhava} means the \textbf{existence of the Grasping Groups}. Fundamentally, then: Birth means birth of `self' and the birth of `I' and `mine'; decay means decay of `self' and the decay of `I' and `mine'; and death means death of `self' and the death of `I' and `mine'.

In the Pali passage, the translation of which has been quoted at the beginning of this chapter, \emph{attanā} is a key word. The Buddha says here that before attaining Buddhahood he was subject to birth, decay and death \textbf{because of `self'} (\emph{attanā}). If this word is lost sight of the entire point is missed.

Apart from the Buddha and the Arahats, every other living being has thoughts of `self' and of `I' and `mine' to some degree or other. To the \emph{puthujjana} his existence is just a matter of existence of `self', a matter of `\textbf{I} exist' or `\textbf{my} existence'. That is, it is just a matter of \emph{bhava}. To him there is only a birth of `self', of a `person' or `somebody' who says: `This is mine; this am I; this is my self'. So is it with decay and death. Where there are no thoughts of `I' and `mine' whatever, no thoughts or feelings of subjectivity, no \emph{bhava}, the question of birth, decay and death does not arise. For, there is no `person', no `I' who is born or decays or dies. Thoughts such as `I was in the past', or `I am in the present', or `I will be in the future' are all over. So also are the thoughts `I was born' or `will I be born again' or `I am decaying' or `I will decay' or `I will die'.

Now, the \emph{puthujjana} neither \textbf{experiences} his birth nor even recollects it. He has also no experience of his death. But the Buddha says that birth is Suffering and death is Suffering. If however, the \emph{puthujjana} does not experience his own birth and death, what `birth' and `death' does he then \textbf{experience} as a Suffering? What is this `birth' that \textbf{is} a Suffering to him? Likewise, what is this `death' that is a Suffering to him?

The answer is:

The \emph{puthujjana} sees \textbf{others} being born and dying. For him this is a matter of \textbf{immediate seeing}. So he comes to the conclusion that he also was born and that he also will die. He thinks `I was born' and `I will die'. This is all that birth and death mean to him during his conscious existence. It is this \textbf{thinking} of his own birth and death that is a \textbf{present} Suffering, and not the actual events of his birth and death. This thinking of his own past birth and his own future death goes on right through his life, forming a part of the mass of Suffering that exists for him.

\begin{quote}
Things have mind as forerunner, mind as chief, are mind made.

\emph{manopubbaṅgamā dhammā manoseṭṭhā manomayā}.\footnote{This verse in the \emph{Dhammapada} embraces in its orbit a far wider range than it is generally reckoned to. Quite understandably it has been given first precedence in this collection of verses in as much as the \emph{Mūlapariyāya Sutta} has been given first precedence in the collection of medium length discourses called the \emph{Majjhima Nikāya}.}

 -- \href{https://suttacentral.net/dhp1-20/en/anandajoti}{Dhp 1}
\end{quote}

What drove Prince Siddhartha out of his palace at the age of twenty-nine was not the actual event of his birth or of a death, but the \textbf{thought} of his past birth and a death to come.

\emph{Jāti}, it must be noted, is \textbf{not} rebirth. In the Pali, rebirth is \emph{punabbavābhinibbatti}, which means, \textbf{the coming to be of a renewed bhava.}\footnote{For example: \emph{katam panāvuso āyatim punabbhavābhinibbatti} -- `How, friend, is there the coming to be of a renewed \emph{bhava}?' (\href{https://suttacentral.net/mn43/en/sujato}{MN 43}). In the following Sutta passage both \emph{jāti} and \emph{punabbhavābhinibbatti} appear: \emph{āyatim punabhhavābhinibbattiyā sati āyatiṁ jāti jarāmaranaṁ sokaparideve dukkha domanassupāyāsā sambhavanti} -- `There being in the future a coming to be of a renewed \emph{bhava}, there is in the future birth, decay, death, sorrow, grief, suffering, lamentation and woe produced.' (\href{https://suttacentral.net/sn12.38/en/bodhi}{SN 12.38})}

Another form of \emph{bhava} springs up. This new springing up in the future is re-birth. And if one is free from birth (as the Arahat is), then one is also free from any possibility of rebirth. `He having realised the destruction of birth does not come to rebirth.' (\href{https://suttacentral.net/iti104/en/sujato}{Iti 104})

As the body with all its sense organs changes from youth to old age, it is to the \emph{puthujjana} that this change is \textbf{decay}. `Decay' is the concept that the \emph{puthujjana} has regarding a change in his body which he considers as `\textbf{my} body'. How does he form this concept? To him the body is a means by which he satisfies his \emph{taṇhā}, i.e. his wanting `my existence' and sense-pleasures. This is the significance his body has to him. When the body has changed to what he calls old, it no longer permits him to enjoy the same satisfaction of his \emph{taṇhā}, which \emph{taṇhā} still remains in him as strong as ever. Thus the body is now not as \textbf{desirable} as it was. He laments and grieves at it. And he considers it as having decayed. But the Arahat has no trace of \emph{taṇhā} whatever. In him there is no wanting `my existence' or sense-pleasures whatever. Thus, to the Arahat, the body does not have the same significance as it has to the \emph{puthujjana}. To the Arahat it is just body and no more. Not having \emph{taṇhā}, when the body grows old he does not lament or grieve at it. It is not decay to him. The body has just changed, and that is all. There is no `I' or a `my this' or a `my that' to decay.\footnote{A change in the body is considered or conceived of as a change for the better or for the worse \textbf{only if} it is considered as a change in `\textbf{my} body'. The same applies to Feeling, Perception, Determinations and Consciousness. It is very important that this is seen.}

\begin{quote}
\begin{itemize}
\item
  `I am' -- monk, this is a supposition (\emph{maññitaṁ}).
\item
  `This am I' -- this is a supposition.
\item
  `I will exist' -- this is a supposition.
\item
  `I will not exist' -- this is a supposition.
\item
  `I will be possessed of Form' \ldots\hspace{0pt}
\item
  `I will be possessed of not-Form' \ldots\hspace{0pt}
\item
  `I will be possessed of Perception' \ldots\hspace{0pt}
\item
  `I will be possessed of non-Perception' \ldots\hspace{0pt}
\item
  `I will be possessed of neither Perception nor non-Perception' -- this is a supposition.
\end{itemize}

A supposition, monk, is a disease; a supposition is an imposthume; a supposition is a barb. Monk, when he has gone beyond all suppositions, the sage is said to be at peace. But, monk, a sage who is at peace is not born, does not decay, is not agitated, does not envy. \textbf{As there stands nothing of which can be said `was born'}, not being so born, how, monk, could he decay (\emph{tam pi'ssa bhikkhu na'tthi yena jāyetha, ajāyamāno kiṁ jiyyissati})? Not decaying, how could he die? Not dying how could he be agitated? Not being agitated, how could he envy?

 -- \href{https://suttacentral.net/mn140/en/bodhi}{MN 140}, The Exposition of the Elements
\end{quote}

That of which can be said `was born' is `self' or `I'. But the Arahat is completely free from `self' and `I'. He has no thoughts of `self or of `I' and `mine' whatever. Therefore he has no thoughts of a `was born' or a `decaying' or a `will decay' or a `will die'. With him there is no `self' or `I' to which \textbf{only} these things apply.\footnote{It is not impossible to use the words `decay' and `death' for the Arahat provided the implications are very clearly kept in mind. The change that happens to the body of the non-Arahat is the same as that which happens to the body of the Arahat. In the former case it is a decay, and this implies that the change is unwelcome and is a Suffering. But in the latter case the change is not unwelcome (in fact, it is neither welcome nor unwelcome) and is not a Suffering. If in this latter case we call the change `decay', then we will have to use the word \textbf{purely} as a \textbf{designation} for the change but having no other significance whatsoever. The same applies to the use of the word `death'. Ordinary usage of the words `decay' and `death', however, always imply definite significances such as unwelcome-ness and Suffering. These significances being wholly and entirely absent for the Arahat, the change that goes on in the Arahat's body is not called decay and the laying down of life in the Arahat is not called death. The Arahat is decayless and deathless.}

All this is of course easily \textbf{stated}, though not at all easy to \textbf{see}. But the Buddha's Teaching \textbf{is} not easy to see. In fact, it is a very difficult Teaching to See.

In the \emph{Upasena Sutta} we have the case of a serpent having fallen on the body of Arahat Upasena. Upasena then requests the monks to lift his body on to a couch and take it outside so that it may break up\footnote{The body `breaking up' refers to life ending.} there. Arahat Upasena was then told that no change for the worse in his faculties necessitating such action was evident. The reply the Arahat gave is very illuminating. He said:

\begin{quote}
Friend Sāriputta, he who should think `I am the eye', `the eye is mine', or `I am the tongue', `the tongue is mine', or `I am the mind', `the mind is mine' -- in him there would be an otherwise-ness in his body, there would be a change for the worse (\emph{viparināmo}) in his faculties. But in me, friend Sāriputta, there are no such thoughts as `I am the eye', `the eye is mine', or `I am the tongue', `the tongue is mine', or `I am the mind', `the mind is mine'. How then, friend Sāriputta, could there be to me the existence of an otherwise-ness in the body, or a change for the worse in the faculties?

 -- \href{https://suttacentral.net/sn35.69/en/sujato}{SN 35.69}, Upasena and the Viper
\end{quote}

So the monks put the Venerable Upasena's body on a couch and bore it outside, and the body broke up then and there.

In the Sutta passage, the translation of which has been just given, we get the word \emph{viparināmo}. The literal meaning of this word is `transformation'. To the non-Arahat this transformation is either a `change for the better' or a `change for the worse'. But to the Arahat there is no such thing. For him there is purely and simply a change which bears \textbf{no} significance of either being for the better or for the worse. This is the basic meaning of Arahat Upasena's reply.

The Buddha did not say that he \textbf{will} be experiencing deathlessness after his life is over and the body broken up. He said that he, likewise the Arahats, \textbf{live experiencing} deathlessness. Exhorting the five monks at Benares (whom he first taught) to listen to him, he described himself thus:

\begin{quote}
The Tathāgata, monks, is Arahat, is All Enlightened. Give ear, monks. Deathlessness has been reached (\emph{amatamadhigataṁ}). I will intruct you.

 -- \href{https://suttacentral.net/pli-tv-kd1/en/brahmali}{Vin I. 5-8}, Mahāvagga
\end{quote}

\emph{Amatamadhigataṁ} means `\textbf{gone} to deathlessness' and \textbf{not} `going to deathlessness.' It is something that \textbf{has happened} or \textbf{has been achieved} `Having attained it and realised it' (\emph{sacchikatvā upasampajja}) the Arahat `lives experiencing it in the body' (\emph{kāyena ca phusitvā viharati}).

The Arahat has come to the cessation of birth, decay and death. He is `entirely freed from birth, decay and death' -- \emph{parimutto jātiyā jarā maranena}. (\href{https://suttacentral.net/an3.38/en/bodhi}{AN 3.38})

He `has done away with birth and death' -- \emph{pahīnajātimarano}. (\href{https://suttacentral.net/an3.57/en/bodhi}{AN 3.57})

He `has gone beyond birth and death' -- \emph{jāti marana maccagā}. (\href{https://suttacentral.net/iti77/en/sujato}{Iti 77})

He is one who `has arrived at the destruction of birth' -- \emph{jātikkhayaṁ patto}. (\href{https://suttacentral.net/iti99/en/sujato}{Iti 99})

He `has conquered death' -- \emph{maranābhibhū}. (\href{https://suttacentral.net/thag20.1/en/sujato}{Thag 1180})

To him applies: `Calm and unclouded, peaceful, freed of longing, he hath crossed over birth and decay, I say' -- \emph{santo vidhūmo anīgho nirāso atāri so jātijaranti brūmī'ti}. (\href{https://suttacentral.net/an3.32/en/bodhi}{AN 3.32})

When Ānanda attained at Arahatship he said of himself, `Gone to the end of birth and death he bears the final frame' -- \emph{dhāreti antimaṁ dehaṁ jātimaranapāragu}. (\href{https://suttacentral.net/thag17.3/en/sujato}{Thag 1022})

Again, the Buddha is the first human being in the world who overcame death, though the greatest thinkers in the world have wondered how it could ever be done. And the Buddha did not overcome death in the fashion that everybody would imagine it should be done. That is by living for ever. He did it by \textbf{removing} that to which death \textbf{applies}. The experience of the living Arahat is birthless, decayless and deathless, because all subjectivity (i.e. everything that is to do with `self' and `I' and `mine') to which alone birth, decay and death are applicable, has been completely cut off never to arise again.

After all this subjectivity has been made extinct there yet remains life for a while longer, which is the life of the Arahat. This the Buddha describes as `stuff remaining' (\emph{upādisesa}). This too comes to an end when the Arahat's life span is over and the body breaks up. But the ending of the Arahat's life is not to be called `death'. About \emph{upādisesa} we shall speak more later.

With anybody other than an Arahat questions pertaining to `after death' (\emph{parammaranā}) are relevant. What happens to the being (\emph{satta}) when the body breaks up after death (\emph{kāyassa bhedā parammaranā}) is a relevant question. But such a question is not relevant to the Arahat. With the Arahat there is no question of death, hence no question of after death. For the Arahat there is only a breaking up of the body (\emph{kāyassa bhedā}) which happens with the Arahat's life coming to an end (\emph{jīvita pariyādānā}). That is all. As we have said earlier, with the Arahat there is no `person' existing. There is only a certain experience going on.

Does the Tathāgata exist after death? Does the Tathāgata not exist after death? Does the Tathāgata both exist and not exist after death? Does the Tathāgata neither exist nor not exist after death?

The Buddha does not give replies to these questions either in the affirmative or in the negative. For this reason it must not be thought that there is something very mysterious about them or that there is something unrevealed by the Buddha here. He teaches that these questions \textbf{do not apply} (\emph{na upeti}). Why so? Because, in relation to the Buddha, there is \textbf{no} `person' or `being' or `somebody' who says `I' and `mine' existing \textbf{to whom} they can apply. Thus there is no death applicable to the Buddha. Hence questions pertaining to `after death' do not apply.

The Buddha on one occasion so admonished Vacchagotta when the latter asked these questions. Vacchagotta then proclaimed that he was at a loss on this point, that he was bewildered, and what is more, that that measure of satisfaction he had had from former conversation with the Buddha -- even that he had now lost! At which the Buddha informed Vacchagotta that he \textbf{ought} to be at a loss, that he \textbf{ought} to be bewildered, which only means that the uninstructed \emph{puthujjana} \textbf{ought} to be at a loss in understanding the Buddha's Teaching.

\begin{quote}
You ought to be at a loss, Vaccha, you ought to be bewildered. For, Vaccha, this Dhamma is deep, difficult to see, difficult to understand, peaceful, excellent, beyond dialectic, subtle, intelligible to the wise.

 -- \href{https://suttacentral.net/mn72/en/thanissaro}{MN 72}, To Vacchagotta on Fire
\end{quote}

This particular Discourse to Vacchagotta is well worth a careful study. The burning flame that is brought in as a simile is to denote the `person' (\emph{sakkāya}). Just as the flame burns and exists by taking up dried leaves and sticks (\emph{tiṇakaṭṭhupādānaṁ}), so does the `person' exist by Grasping. And just as the flame will become extinct (\emph{nibbāyeyya}) when there is no more taking up of dried leaves and sticks, so does the `person' become extinct when the Grasping ceases. What would remain is that which we referred to as the `stuff remaining' and designated as Arahat. In as much as there is now no flame to go east, west, north, south or anywhere else, with regard to the Arahat there is no `person' to die, and hence no `person' to arise after death.

The \emph{puthujjana} looks upon the Arahat as he would look upon himself. That is as a \emph{sakkāya}, a `self', a `person' who says `I' and `mine'. Thus viewing he puts these questions. The \emph{puthujjana} being a Five \textbf{Grasping} Groups (which essentially means having thoughts of subjectivity, of `I' and `mine') thinks that the Arahat is also a Five \textbf{Grasping} Groups. He does not know that \textbf{all} Grasping is extinct in the Arahat, that the Arahat `has laid down all Grasping' -- \emph{sabbupādānapariyādāna}, (\href{https://suttacentral.net/sn35.62/en/bodhi}{SN 35.62}) that the Arahat `has destroyed all Grasping' -- \emph{sabbupādānakkhayaṁ}. (\href{https://suttacentral.net/ud3.10/en/anandajoti}{Uda 3.10}) He does not see that the Arahat `by the destruction, dispassion, cessation, giving up, casting out all suppositions, all standpoints, all latent conceits of `I' and `mine', is freed without Grasping'. (\href{https://suttacentral.net/mn72/en/thanissaro}{MN 72})

When the Arahat is asked questions about himself on the basis of things not applicable to him, what other reply can he give than saying that those questions about him do not apply to him?

\begin{quote}
Even so, great king,

\begin{itemize}
\item
  \textbf{that} Form \ldots\hspace{0pt}
\item
  \textbf{that} Feeling \ldots\hspace{0pt}
\item
  \textbf{that} Perception \ldots\hspace{0pt}
\item
  \textbf{those} Determinations \ldots\hspace{0pt}
\item
  \textbf{that} Consciousness
\end{itemize}

by which one discerning the Tathāgata might discern him -- 

\begin{itemize}
\item
  \textbf{that} Form \ldots\hspace{0pt}
\item
  \textbf{that} Feeling \ldots\hspace{0pt}
\item
  \textbf{that} Perception \ldots\hspace{0pt}
\item
  \textbf{those} Determinations \ldots\hspace{0pt}
\item
  \textbf{that} Consciousness
\end{itemize}

has been got rid of, cut off at the root, made like a palm-tree stump that can come to no further existence and is not liable to rise again in the future. Freed from reckoning as Consciousness is the Tathāgata, great king. He is deep, immeasurable, unfathomable as is the great ocean. To say, `The Tathāgata exists after death', does not apply. To say, `The Tathāgata does not exist after death', does not apply. To say, `The Tathāgata does exist and does not exist after death', does not apply. To say, `The Tathāgata neither exists nor does not exist after death', does not apply.

 -- \href{https://suttacentral.net/sn44.1/en/bodhi}{SN 44.1}, Khema
\end{quote}

The Groups of Form, Feeling, Perception, Determinations and Consciousness which have been cut off at the root never to arise again are the \textbf{Grasping} Groups of Form, Feeling, Perception.

Determinations and Consciousness. And birth, decay and death apply only to the Grasping Groups, because an `I' or a `self', to which only birth, decay and death are applicable, is present only if there is Grasping. When Grasping is extinct, all such subjectivity is extinct. What then remains is a residual \textbf{Not-Grasping} Five Groups to which birth, decay and death do not apply. `This is deathlessness, that is to say, the deliverance of the mind from Grasping' -- \emph{etaṁ amataṁ yadidaṁ anupādā cittassa vimokkho}. (\href{https://suttacentral.net/mn106/en/sujato}{MN 106})

\begin{quote}
The King Pasenadi asks the Buddha,

``To the born is there any other than decay and death?''

To which the Buddha replies,

``To the born, great king, there is none other than decay and death.

``Great king, were there eminent nobles, prosperous, owning great treasure, great wealth, large hoards of gold and silver, immense means, abundant supplies of goods and corn -- to them who are born there is none other than decay and death.

``Great king, were there eminent brahmins \ldots\hspace{0pt}

``Great king, were there eminent householders, prosperous, owning great treasure, great wealth, large hoards of gold and silver, immense means, abundant supplies of goods and corn -- to them who are born there is none other than decay and death.

``Great king, were there monks who are Arahat, have destroyed the taints, have finished, done what was to be done, laid down the burden, won the highest good, completely destroyed the fetter of \emph{bhava}, freed by right insight -- to them there is a breaking up of the body, a laying down of it.''

 -- \href{https://suttacentral.net/sn3.3/en/sujato}{SN 3.3}, Old Age and Death
\end{quote}

In the above reply the Buddha teaches that birth, decay and death are applicable to the nobles, brahmins, etc. But when it comes to the Arahat, birth, decay and death do not apply.

If the point that has been discussed in this chapter is missed the uniqueness of the Buddha's Teaching is also missed. The Buddha's Teaching is to be experienced here and now, in this life -- all of it, from beginning to end. Decaylessness and deathlessness are also to be experienced here and now.
