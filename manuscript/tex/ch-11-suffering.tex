\chapter{Suffering}

Just as one does with some other doctrines in Buddhism one imagines all too soon that one has comprehended the doctrine of Suffering also. Such imaginings and coming to conclusions effectively impede one's progress. For one believes one has understood what in truth one has not.

The \emph{puthujjana} meets with sufficient Suffering in his life. He may have also heard about the Buddha having declared that life is Suffering. But yet, in spite of it all, however much Suffering he undergoes, he is not drawn towards the Buddha's Teaching, which offers him the way out of Suffering. He contents himself by feeling that he will in time get over his present state of Suffering, and that his present state of Suffering is just another one of those aspects of life which he must put up with. `Sense pleasures are said by me to be of little satisfaction, of much Suffering, of much tribulation, wherein is more peril.' (\href{https://suttacentral.net/mn22/en/bodhi}{MN 22}) The \emph{puthujjana} does not see this.

\enlargethispage{\baselineskip}

The \emph{puthujjana} delights in the fact of his existence. `I exist' or `my existence' is something desirable to him. He delights in notions of `self'. He delights in seeing things as `mine'. `\,``All is mine,'' he conceives. He delights in All' -- \emph{sabbaṁ meti maññati sabbaṁ abhinandati} (\href{https://suttacentral.net/mn1/en/bodhi}{MN 1}). He delights in `self'-existence. He does not see that `self'-existence is Suffering. He does not see that delighting in `self'-existence is really a delighting in Suffering. He can even have an aversion to destroying his thoughts of `I' and `mine', which means he can have an aversion to the Buddha's Teaching, an aversion to treading the Noble Eightfold Path that leads to the utter destruction of these thoughts. It is therefore no surprise that the Buddha was rather hesitant at the very outset to teach what he had discovered.

The reason for the \emph{puthujjana} acting in this fashion is that he really does \authoremph{not} see Suffering. He may \authoremph{believe} that he sees Suffering, but actually he does not see it. For him to see Suffering he must see Impermanence and Not-self too. The perception of Suffering comes only together with the perception of Impermanence and Not-self. And developing that perception is by no means an easy task.

\protect\hypertarget{impermanent}{}{}'Self' always implies permanency, and hence desirability. The \emph{puthujjana} has to see that what he takes to be his `self' (thereby to be permanent and desirable) is really Not-self. This he can only see by seeing that what he takes as his `self' is impermanent. When that which he takes to be `self' is seen to be impermanent he no longer takes it to be `self', and therewith he loses desire for it. He also sees that by taking it to be `self', whilst in truth it is Not-self, he is always led into betrayal and disappointment, i.e. he is always led into Suffering.

\enlargethispage{\baselineskip}

\begin{quote}
Neither do I, monks, see that holding to a belief in `self' from the holding to which there would not arise sorrow, lamentation, suffering, grief, despair.

 -- \href{https://suttacentral.net/mn22/en/bodhi}{MN 22}, The Simile of the Snake
\end{quote}

When this insight grows in him he sees that he had been working right through with a deception -- the deception of `self'. He sees that he had been working on the basis that a deception was the actual thing, and that therefore it was always a case of betrayal, disappointment, grief, agitation, worry, suffering, to some degree or other. `It is just Suffering that is produced, Suffering that persists and disappears. Nought beside Suffering is produced, nought beside Suffering ceases.' (\href{https://suttacentral.net/sn5.10/en/bodhi}{SN 5.10}) To the extent one sees this, to that extent one has right view.

When the individual no longer regards anything as `self' (which only means that he no longer regards any of the Five Grasping Groups as `self') he has got rid of his \emph{sakkāyadiṭṭhi}. He has then crossed over from the plane of the \emph{puthujjana} to the plane of the \emph{Ariyas} (Noble Ones), and along with it he has left a whole heap of Suffering behind him which otherwise he would have to undergo. Since he may still have thoughts of `I' and `mine' left in him, he will yet have Suffering. But that Suffering will be nothing compared with what it was when he was a \emph{puthujjana}.

The Buddha summarily defines Suffering as the Five Grasping Groups: `Birth is Suffering, decay is Suffering, disease is Suffering, death is Suffering, union with the undesired is Suffering, being sundered from the desired is Suffering, not getting what is wished for is Suffering. In short, the Five Grasping Groups are Suffering.' (\href{https://suttacentral.net/pli-tv-kd1/en/brahmali}{Vin I. 5-8}) This essentially means, all notions of `self', all thoughts of `I' and `mine' are Suffering.

Birth, decay disease, death, etc., are all Suffering because they apply to a `self' or an `I' who is subject to all these. The \emph{puthujjana} thinks `I was born' or `I will be born' or `I am diseased' or `I am decaying' or `I will decay' or `I will die' or `I am not getting what I want', etc. He thus laments and grieves and despairs and sorrows, and so suffers. It is `I' who is lamenting. It is `I' who is disappointed and betrayed by the scheme of things. It is `I' who is grieving, etc. It is `my' body that is altering in a manner that `I' do not wish it to. It is `my' perception that `I' am not satisfied with, and so on. If `my' life or `my' so and so's life is not affected then there is no worry, no care, no Suffering. What is there to care for or get worried about a life that does not concern `me'? For a world that is not `mine'? There is attachment to something only because that something is `mine' or has to do with `mine'. For me to be concerned about something it has to have a connection with \authoremph{me} in \authoremph{some} way or other.

\begin{quote}
I will show you, monks, worry from Grasping, likewise no worry from no Grasping, Do ye listen.

And how, monks, is there worry from Grasping?

Herein, monks, the untaught \emph{puthujjana} regards Form thus: `This is mine; this am I; this is my self'. Of such a one the Form alters and becomes otherwise. Owing to the altering and otherwise-ness of Form, sorrow and grief, woe, lamentation and despair arise in him. Thus, monks, there is worry from Grasping.
\end{quote}

The same applies to the other four Groups, Feeling, Perception, Determinations and Consciousness.

\begin{quote}
And how, monks, is there no worry from no Grasping?

Herein, monks, the well-taught Ariyan disciple (Noble disciple) regards Form thus: `Not, this is mine; not, this am I; not, this is my self.'\footnote{\emph{Na etaṁ mama} is usually translated as `This is not mine'. But this rendering tends to leave in the reader's mind the impression that though \authoremph{this} is not mine, there may be something else that is mine. In fact such an impression is deliberately made to remain in the reader's mind when, for instance, \emph{na eso me attā} is translated by scholars as `this is not the self of me' -- as if to say that \authoremph{this} is not my self, but something else is. Such situations have to be avoided. `Not, this is mine' (which is a translation by Ñāṇavīra Thera) may not sound quite perfect. But accuracy in meaning is more important than readability. The same of course applies to the whole triad.}

\emph{Na etaṁ mama, na eso ahaṁ asmi, na eso me attā.}
\end{quote}

Of such a one the Form alters and becomes otherwise. But in spite of the altering and otherwise-ness of Form, sorrow and grief, woe, lamentation and despair arise not in him.

\begin{quote}
Thus, monks, there is no worry from no Grasping.

 -- \href{https://suttacentral.net/sn22.8/en/bodhi}{SN 22.8}, Agitation through Clinging (2)
\end{quote}

The same again applies to the other four Groups.

Now, it is the \authoremph{same} phenomenon, viz., altering and becoming otherwise, that is taking place with the untaught \emph{puthujjana} and the well-taught Ariyan disciple. But their attitudes towards the phenomenon, the ways in which they regard it \emph{(samanupassati)} are of opposing kinds. The first way brings up Suffering, the second way prevents Suffering coming up. With the Arahat, of course, the position is different, because, he \authoremph{having come} to the end of \authoremph{all} thoughts of `self', and of `I' and `mine', \authoremph{neither} regards things as `mine', etc., \authoremph{nor} as `not mine', etc. With the Arahat there is no Suffering whatever. No question of the arising of Suffering or not-arising of Suffering is present for him. Of the differences between the \emph{puthujjana}, the Ariyan disciple, and the Arahat, we shall speak in detail later.

When a person begins to see Impermanence, Not-Self, and therewith Suffering as the Ariyan disciple does, he begins to lose delight in thoughts of `I' and `mine'. He truly gets drawn towards the Buddha's Teaching, and he begins to see a definite purpose in his existence which makes his existence a very important matter to him. Henceforth he does not aimlessly wander `taking things as they come'. He lives with a purpose and fashions his life relentlessly to achieve that purpose.

Actually, in the end, the \emph{puthujjana} sees nothing of which he can rightly say: this and \authoremph{no other} is what has to be done. Fettered he is born, fettered he exists, fettered he dies, fettered to `self'. He finds that his existence \emph{(bhava)} is without meaning and purpose. Yet he knows not how to end his purposeless existence. But the Ariyan disciple sees a purpose to his existence, and sees that this purpose is nothing but the bringing of that existence to an end \emph{(bhavanirodha)}.

Since the Arahat has no notions of Subjectivity, no thoughts of `I' and `mine' whatever, he has come to the extinction of that which conditions Suffering. When the condition for Suffering is extinct, Suffering is also extinct. He has come to the extinction of Suffering. He has come to the cessation of Suffering -- \emph{dukkhanirodho}.

\enlargethispage{\baselineskip}

\begin{quote}
He having put aside all tendency towards attachment, having dispelled all tendency to resistance, having removed tendencies to views and conceits such as `I am', having put aside ignorance, Knowledge having arisen, he is here and now an end-maker of Suffering.

 -- \href{https://suttacentral.net/mn9/en/bodhi}{MN 9}, Right View
\end{quote}

No more can any Suffering arise in him. He certainly can have painful or unpleasant feeling, but such painful or unpleasant feeling, is \authoremph{not} Suffering. In such an eventuality he just bears the pain. He does not \authoremph{suffer} by it. Suffering is all over with him.

\begin{quote}
He feeling a pleasant feeling, feels it unbound \emph{(visaññutto)} to it; feeling an unpleasant feeling, feels it unbound to it; feeling a neutral feeling, feels it unbound to it. He, monks, is called an Aryan disciple unbound to birth, decay, death, sorrow, grief, suffering, lamentation and woe; he is unbound to Suffering, I~declare.

 -- \href{https://suttacentral.net/sn36.6/en/bodhi}{SN 36.6}, The Dart
\end{quote}

Suffering \emph{(dukkha)}, it must be noted, does not refer to bodily pain. In the Pali, the word \emph{dukkha} is used to denote Suffering, and also to denote that a feeling is painful (as in \emph{dukkhaṁ vedanaṁ}). Suffering is something mental. It refers to sorrow, woe, lamentation, grief, despair, agitation, worry, etc., all of which are \authoremph{mental}. That is why the Arahat can have bodily pain, but no Suffering.\footnote{When the Arahat's body changes to the state that the \emph{puthujjana} considers as a state of decay, the Arahat can then have bodily painful feelings. But these bodily painful feelings do not lead him to consider the body as having decayed, a consideration which is nothing but Suffering since it is always attended with grief, fear, etc.}

When the \emph{puthujjana} experiences a painful feeling, he feels a repugnance for it. This means he has a twofold feeling, i.e., a bodily painful feeling and a mental painful feeling. `He feels a twofold feeling, bodily and mental.' (\href{https://suttacentral.net/sn36.6/en/bodhi}{SN 36.6}) He knows no refuge from painful feeling other than sensual pleasure. He is thus bound to sensual pleasure. The Arahat on the other hand can also experience a painful feeling. But neither does he have a repugnance for painful feeling nor has he a delight in sensual pleasure. Whether it is a pleasant feeling, or an unpleasant or painful feeling, or a neutral feeling, the Arahat is neither worried by it nor delighted by it. For him, it is just a feeling.

\begin{quote}
Now on that occasion a certain monk was seated not far from the Exalted One in cross-legged posture, holding his body upright, enduring pain that was the fruit of former \emph{kamma}, pain racking, sharp and bitter; but he was mindful, composed and uncomplaining. And the Exalted One saw that monk so seated and so employed, and seeing the meaning of it, at that time gave utterance to this saying of uplift:

`For the monk who hath all \emph{kamma} left behind,\\
and shaken off the defilements aforetime gathered,\\
who stands fast without ``mine'' --\\
for such there is no need to talk to folk'.

-- \href{https://suttacentral.net/ud3.1/en/anandajoti}{Ud 3.1}, The Discourse about Deeds
\end{quote}

The Arahat does not need to talk to folk, entreating them to relieve him of his pain, or complaining to them about his pain, because he does not \authoremph{suffer} by it, because it gives him no grief, lamentation, etc. For grief, lamentation, etc. to be there he must think `\authoremph{I am} in pain', and such thoughts are completely extinct in him. Any pain that comes his way -- that he just bears.

\clearpage

Again:

The Buddha says that for the \emph{puthujjana} \authoremph{all} is Suffering. That is to say, with regard to feeling for instance, whether the \emph{puthujjana's} feelings are pleasant, unpleasant or neutral, they are nevertheless Suffering. It is not only unpleasant feeling, that is Suffering for him, but \authoremph{all} feeling.\footnote{`Whatever is felt, that is Suffering' -- \emph{yaṁ kiñci vedayitaṁ taṁ dukkhasmin'ti} (\href{https://suttacentral.net/sn12.32/en/bodhi}{SN 12.32}, The Kaḷara). Or again, `It is just Suffering that is produced, Suffering that persists and disappears. Nought beside Suffering is produced, nought beside Suffering ceases' -- \emph{Dukkhaṁ eva hi sambhoti, dukkhaṁ tiṭṭhati veti ca, nāññatra dukkha sambhoti, nāññatra dukkhā nirujjhati ti} (\href{https://suttacentral.net/sn5.10/en/bodhi}{SN 5.10}, Vajirā Sutta).} It is precisely \authoremph{this} that is difficult to see, and hence the difficulty of seeing the First Noble Truth.

To see this one has to turn towards the fundamental characteristic of the \emph{puthujjana}, which is but a regarding things as `mine'. The \emph{puthujjana} regards that which should be regarded as `\authoremph{not} mine' as `mine'. That means he regards the Five Grasping Groups (which constitute \authoremph{all} for him) as `mine' whilst he should regard them as `not mine'.

With regard to feeling, whether the feeling he experiences is pleasant or unpleasant or neutral, he regards it always as `mine'. This regarding the Groups as `mine' is always attended with agitation and worry to \authoremph{some} degree or other, which only means that he is \authoremph{always suffering to some degree or other}.\footnote{In the complex structure of the deliberation `this is mine' \emph{(etam mama)} there are to be found those mental concomitants such as agitation, worry, fear, doubt, etc. These mental concomitants are a necessary part of the structure of this deliberation. Likewise, the deliberation `not, this is mine' \emph{(na etaṁ mama)} is divorced from these mental concomitants. These mental concomitants are \emph{dukkha}. Thus, fundamentally, the arising and ceasing of \emph{dukkha} is to be found in these deliberations. Unless this is seen the First Noble Truth is not seen. With the Arahat, of course, no \emph{dukkha} arises at all, the thought `mine' never arising in him. Therefore, with him, there is also no \emph{dukkha} to cease.} As we shall see in the next chapter, the \emph{puthujjana} acts in this fashion because he is Ignorant of (i.e. he does not \authoremph{see}) the Four Noble Truths, viz., the Noble Truth of Suffering, the Noble Truth of the Arising of Suffering, the Noble Truth of the Ceasing of Suffering, and the Noble Truth of the Path leading to the Ceasing of Suffering. In other words, the \emph{puthujjana} continues to suffer with no prospect of reducing his Suffering, and therefore continues to be a \emph{puthujjana}, because he is ignorant of the Buddha's Teaching.

\bigskip

\begin{quote}
Now I, brahmin, lay down that a man's wealth is the Dhamma,\footnote{I.e., the Buddha's Teaching.} Ariyan, beyond the world \emph{(lokuttara)}.

 -- \href{https://suttacentral.net/mn96/en/sujato}{MN 96}, With Esukārī
\end{quote}
