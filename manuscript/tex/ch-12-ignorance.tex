\chapter{Ignorance}

We have seen that the arising of Consciousness is dependent upon conditions.

In other words, the presence of an experience (the presence being Consciousness, and the experience being Name-and-Form) is dependent upon conditions.

These conditions determine \authoremph{what} is to be present, i.e., they determine what particular Consciousness is to be. All these conditions taken as a whole are also Name-and-Form, which again is nothing but the other four Groups.

But of these conditions totalled as Name-and-Form there is one condition which plays a key role, viz., intention. Intention directs the play as it were. And the direction along which intention directs is dependent upon \emph{taṇhā}.

Now, the Buddha teaches that Consciousness is dependent upon Name-and-Form. He also teaches that Consciousness is dependent upon Determinations (\emph{saṅkhārapaccayā viññānaṁ}). That is, with Determinations as condition, arises Consciousness.

Thus we have, in the above mentioned instance, Determinations (\emph{saṅkhārā}) being synonymous with Name-and- Form. The reason is that although intention is primarily a condition for the arising of Consciousness, through intention \authoremph{alone} pure and simple, Consciousness cannot come about. All those other conditions such as In-and-Out-Breathing, Perception, Feeling, etc., must also be present. In the statement, Consciousness is dependent upon Determinations (\emph{saṅkhāra paccayā viññānaṁ}), the word Determinations (\emph{saṅkhāra}) includes all these things.

The question that arises now is: How is it that the Grasping Groups persist in the manner explained so far and in \authoremph{no other} manner? In other words, why does life (save that of the Arahat's, of course) persist in the manner it does and in no other manner? This same question can be put in other ways too. For example, it may be put thus: If \emph{taṇhā} is that which guides intentional action, and hence the play of life, what is it that keeps this all important factor called \emph{taṇhā} in existence?

The answer is: Ignorance of life. In other words, things are not \authoremph{seen} in their true nature. They are seen wrongly. And Ignorance of life means nothing but Ignorance about the Five Grasping Groups.

\emph{Taṇhā}, that is to say, wanting `self'-existence and sense-pleasure, is maintained because of Ignorance.

\begin{quote}
I declare, monks, that \emph{bhava-taṇhā} (wanting `self'-existence) is with nutriment, not without nutriment. And what is the nutriment of \emph{bhava-taṇhā}? Ignorance is to be so called.

 -- \href{https://suttacentral.net/an10.62/en/bodhi}{AN 10.62}, Craving
\end{quote}

Ignorance can be defined in more than one way. It can be defined as not seeing or not knowing the arising and ceasing of the Five Grasping Groups.

\begin{quote}
``Ignorance! Ignorance! it is said, Lord. But what, Lord, is Ignorance, and to what extent is one Ignorant?''

``Herein, monks, the uninstructed \emph{puthujjana} does not, as it really is, know Form that is of the nature of arising as Form that is of the nature of arising; does not, as it really is, know Form that is of the nature of passing away as Form that is of the nature of passing away; does not, as it really is, know Form that is subject to arising and passing away as Form that is of the nature of arising and passing away. He does not, as it really is, know Feeling \ldots\hspace{0pt} Perception \ldots\hspace{0pt} Determinations \ldots\hspace{0pt} Consciousness \ldots\hspace{0pt} This, monks, is called Ignorance, and to that extent is there Ignorance.''

 -- \href{https://suttacentral.net/sn22.126/en/sujato}{SN 22.126}, Liable To Originate
\end{quote}

Not knowing the arising and ceasing nature of the Five Grasping Groups means not knowing Impermanence. And not knowing Impermanence simultaneously involves not knowing Not-self and not knowing Suffering. Therefore Ignorance may also be defined as not knowing Impermanence, Not-self and Suffering.

It should be clearly understood that by the word `knowing' (\emph{pajānāti}) is not meant a mere conceptual and objective knowing, but a knowing as true -- a seeing, a comprehending, with \authoremph{no doubt} about it (\emph{tiṇṇa vicikiccho}).

Another way of defining Ignorance would be:

\begin{quote}
Friend, non-knowledge of Suffering, non-knowledge of the arising of Suffering, non-knowledge of the ceasing of Suffering, non-knowledge of the Path leading to the ceasing of Suffering -- this, friend, is called Ignorance.

 -- \href{https://suttacentral.net/mn9/en/bodhi}{MN 9}, Right View
\end{quote}

Suffering, arising of Suffering, ceasing of Suffering, and the Path leading to the ceasing of Suffering, are called the Four Noble Truths. Thus Ignorance is the non-knowledge of the Four Noble Truths.

Now, \emph{taṇhā} is present because Ignorance is present. In other words, since life is Ignorant of itself it has \emph{taṇhā}. What then must be present for Ignorance to be present? Or, on what condition does Ignorance depend? This is a very important question.

\begin{quote}
Monks, a first point of Ignorance is not to be discerned, so that one may say: `Before this was not Ignorance, it has come to be since.' This, however, is to be discerned: `Ignorance is dependent on this.' I say, monks, that Ignorance is with nutriment, not without nutriment.

And what is the nutriment of Ignorance? The Five Hindrances (\emph{pañca nīvaraṇā}) are to be so called. The Five Hindrances too, I declare, are with nutriment, not without nutriment. And what is the nutriment of the Five Hindrances? The three evil ways of conduct (\emph{tīṇiduccaritāni}) are to be so called. They too, I declare are with nutriment, not without nutriment.

And what is the nutriment of the three evil ways of conduct? Non-restraint over the senses (\emph{indriya asaṁvaro}) are to be so called. That too, I declare, to be with nutriment, not without nutriment.

And what is the nutriment of non-restraint over the senses? Lack of mindfulness and clear comprehension (\emph{asatāsampajaññaṁ}) are to be so called. That too, I declare to be with nutriment, not without nutriment.

And what is the nutriment of lack of mindfulness and clear comprehension? Improper attention (\emph{ayoniso manasikāro}) is to be so called. That too, I declare, is with nutriment, not without nutriment.

And what is the nutriment of improper attention? Absence of faith (\emph{asaddhiyaṁ}) is to be so called. That too, I declare, is with nutriment, not without nutriment.

And what is the nutriment of absence of faith? Hearing of not-right doctrine (\emph{asaddhamma savanaṁ}) is to be so called. That too, I declare, is with nutriment, not without nutriment.

And what is the nutriment of hearing of not-right doctrine? Association with the unworthy (\emph{asappurisasaṁsevo}) is to be so called.

Thus, indeed, monks, the fulfilment of association with the unworthy fulfils the hearing of not-right doctrine; the fulfilment of hearing not-right doctrine fulfils absence of faith; the fulfilment of absence of faith fulfils improper attention; the fulfilment of improper attention fulfils lack of mindfulness and clear comprehension; the fulfilment of lack of mindfulness and clear comprehension fulfils non-restraint over the senses; the fulfilment of non-restraint over the senses fulfils the three evil ways of conduct; the fulfilment of the three evil ways of conduct fulfils the Five Hindrances; the fulfilment of the Five Hindrances fulfils Ignorance. Such is the nourishment of Ignorance, such is the fulfilment.

 -- \href{https://suttacentral.net/an10.62/en/bodhi}{AN 10.62}, Craving
\end{quote}

Thus, according to the above, Ignorance finally depends upon not hearing the Buddha's Teaching, and of course upon not practising it. At the same time if we take all of the factors narrated in the above, beginning with the Five Hindrances, (these Hindrances being lust, ill-will, sloth and torpor, restlessness and worry, and doubt) upon which Ignorance depends, we find that each and every one of these factors \authoremph{involves the presence of Ignorance.}

For the Five Hindrances, improper attention, etc., to be present Ignorance must be present. It would then appear that Ignorance depends upon Ignorance. This is in fact directly indicated in the \emph{Sammādiṭṭhi Sutta}.\footnote{\href{https://suttacentral.net/mn9/en/bodhi}{MN 9}} In this \emph{Sutta} it is said that the condition for Ignorance is the taints; `From the arising of the taints is the arising of Ignorance. From the ceasing of the taints is the ceasing of Ignorance.' These taints are in turn defined in the same \emph{Sutta} as the taint of sense-pleasure, taint of \emph{bhava}, and the taint of \authoremph{Ignorance}: `Friend, there these three taints, viz., the taint of sense-pleasure, the taint of \emph{bhava}, the taint of Ignorance.' Thus Ignorance depends upon the taint of Ignorance.

That Ignorance depends upon Ignorance really means this: any thing upon which Ignorance depends \authoremph{involves} Ignorance. The Five Grasping Groups is a matter of \authoremph{not-knowing} things rightly. Thus all factors that make for the \authoremph{Grasping} Groups are individually also a matter of \authoremph{not-knowing} rightly. Ignorance is something negative and abstract -- \authoremph{not-knowing}. Its corresponding direct positive manifestation is the Grasping, which essentially is a considering things as `mine'.

With every attempt the uninstructed \emph{puthujjana} makes to gain right knowledge he carries Ignorance with him. When he tries to spot out Ignorance, to recognize Ignorance -- that he does with Ignorance. It is as if a man sets out to catch a thief, but the thief himself leads the man in his search, because the man does not recognize the thief as \authoremph{the} thief.

This indicates to us how firm Ignorance is and how difficult it is to get rid of it, so much so that it appears almost an impossibility to get rid of Ignorance without some external aid. It is this aid, however, that the Buddha gives. He gives it in the form of a Teaching that goes \authoremph{against} the \emph{puthujjana's} understanding of things. That is why the Buddha describes his Teaching as `going against the stream' (\emph{paṭisotagāmi}).\footnote{\href{https://suttacentral.net/mn26/en/bodhi}{MN 26}} The \emph{puthujjana} constantly thinks `This is mine; this am I; this is my self.' The Buddha points out to him that it is wrong, and teaches him to think instead `Not, this is mine; not, this am I; not, this is my self'. It is because of Ignorance that the \emph{puthujjana} thinks `This is mine \ldots\hspace{0pt}' But every time he thinks against this stream, metaphorically speaking, he injects into Ignorance a destructive poisonous dose. When this `going against the stream' is practised there come a time when \authoremph{all} notions of `self', \authoremph{all} thoughts of `I' and `mine' are completely got rid of, never to arise again.

\begin{quote}
``Lord, how knowing, how seeing, does there not come to be in this body having Consciousness, and in all external indications, the tendency to the conceit `I' and `mine'?''

``Rāhula, whatever Form \ldots\hspace{0pt} Feeling \ldots\hspace{0pt} Perception \ldots\hspace{0pt} Determinations \ldots\hspace{0pt} Consciousness, be it past, future, or present, external or internal, gross or subtle, low or high, far or near -- all Consciousness -- (is to be regarded as) `Not, this is mine; not, this am I; not, this is my self.' That is seeing things by right insight as they really are.''

``Thus knowing, Rāhula, thus Seeing, in this body having Consciousness, and in all external indications, there comes to be no tendency to the conceit `I' and `mine'.''

 -- \href{https://suttacentral.net/sn22.91/en/bodhi}{SN 22.91}, Rāhula
\end{quote}

All thoughts of `I' and `mine' are completely got rid of means that Ignorance is completely got rid of; which again means that the entire purpose of all this effort is achieved, viz., Suffering is wholly and entirely destroyed.

The Arahat has got rid of Ignorance, which means that the Arahat fully \authoremph{knows}, or that (Right) Knowledge has arisen (\emph{vijjā uppanno}) in him. And he fully knows means, he has \authoremph{ended} Grasping. With him, the `person' is extinct; `my existence is extinct; Suffering is extinct.

It should be noted that three distinct types of individuals are involved in all this. Firstly the \emph{puthujjana} who thinks `This is mine \ldots\hspace{0pt}' Secondly, the Aryian disciple who \authoremph{sees} that `This is mine \ldots\hspace{0pt}' is wrong, but still is \authoremph{not} rid of thoughts of `I' and `mine'. It is \authoremph{this} second type of individual who thinks `Not, this is mine \ldots\hspace{0pt}'. He is called a `learner' (\emph{sekha}), and he is \authoremph{on the Path} to Arahatship. Thirdly, there is the Arahat. The Arahat not only sees that `This is mine \ldots\hspace{0pt}' is wrong, but also \authoremph{has completely rid} himself of thoughts of `I' and `mine'. Therefore the Arahat does \authoremph{not} have the occasion to say `Not, this is mine \ldots\hspace{0pt}' either. He is called `learning-ender' (\emph{asekha}: literally `not-learner', but to prevent any confusion it is better translated as `learning-ender').

Thus, summarily: the \emph{puthujjana} says `This is mine \ldots\hspace{0pt}'; the Ariyan disciple on the Path says `Not, this is mine \ldots\hspace{0pt}'; the Arahat says neither.

These distinctions, particularly that between the Ariyan disciple on the Path and the Arahat, should be noted, or else confusion can arise.

We have said that it is almost impossible to overcome Ignorance without some external aid. How then did the Buddha overcome it without any such aid? The Buddha said, `For me there is no teacher.'\footnote{\href{https://suttacentral.net/mn26/en/bodhi}{MN 26}, The Noble Search} This means he overcame Ignorance by himself.

The answer is: though it is extremely difficult and appears almost impossible, it is nevertheless possible. The destruction of Ignorance \authoremph{unaided} is something so difficult that it is extremely rare. It is precisely as rare as the appearance of Buddhas.
