\chapter{Nibbāna}

By now the reader may have obtained some idea of the individual called the Arahat. Essentially, the Arahat is that individual who has no notions whatever of `self', and no thoughts whatever of `I' and `mine'. Thus he is not subject to any form of Suffering whatever. In other words, we may say that in the Arahat these things are extinct. Or, the Arahat is that individual who experiences here and now the \authoremph{extinction} of these things. In him these can \authoremph{never} arise again. He \authoremph{experiences} extinction. That is, he experiences \emph{Nibbāna}.

\emph{Nibbāna} literally means extinction (or cessation). The word by itself says nothing more. To the question `The extinction of what, is \emph{Nibbāna}?' many answers can be given. Some of the more important answers would be that it is the extinction of subjectivity, or of Suffering, or of the taints, or of \emph{taṇhā}, or of Grasping. The extinction of any one of these things implies \authoremph{also} the extinction of the others. When all subjectivity is extinct, all Suffering is also extinct, Ignorance is also extinct, the taints are also extinct, and so on. It is therefore clear that \emph{Nibbāna} can be described in many ways, and in discussion one would usually refer to it as the extinction of that thing round which the discussion revolves. If one is discussing \emph{bhava}, then it would be appropriate to describe \emph{Nibbāna} as the extinction or cessation of \emph{bhava}: `The cessation of \emph{bhava} is \emph{Nibbāna}'\footnote{\href{https://suttacentral.net/an10.7/en/bodhi}{AN 10.7}, Sāriputta} -- \emph{bhavanirodho nibbānaṁ}. If one is discussing the `person' \emph{(sakkāya)} then it would be appropriate to say that \emph{Nibbāna} is the cessation of the `person' \emph{(sakkāyanirodho)}.\footnote{\href{https://suttacentral.net/sn22.105/en/sujato}{SN 22.105}, Identity} If we reckon the purpose of the Buddha's Teaching (which should at no time be lost sight of) then we will describe \emph{Nibbāna} as the extinction of Suffering, or as the cessation of Suffering, or as the destruction of Suffering.

\emph{Nibbāna} is often defined as `the destruction of lust, the destruction of hatred, and the destruction of delusion.'\footnote{\href{https://suttacentral.net/sn38.1/en/sujato}{SN 38.1}, Extinguishment}

We saw earlier that Arahatship also has been defined by the Buddha as `the destruction of lust, the destruction of hatred, and the destruction of delusion.' It means that the Arahat experiences all this destructions. He experiences the extinction of lust, hatred and delusion. Thus he experiences \emph{Nibbāna}.

In the \emph{Saṁyutta Nikāya IV}, there are 33 descriptive words given for the Not-\emph{Determined} \emph{(asaṅkhata)} or for the destruction of lust, hatred and delusion. As terms for the destruction of lust, hatred and delusion, they also become terms for Arahatship, and so for \emph{Nibbāna} as well. In fact one term is \emph{Nibbāna} itself. It is well worth rapidly going through these terms dwelling at some length on the more important ones.

\authoremph{(1) Asaṅkhata -- The \emph{Not}-Determined}

We have seen its meaning earlier. It means Not-\emph{Determined}. There is no `person' or subject (`I') determined. To the Arahat \authoremph{only} is applicable the word `impersonal' in its \authoremph{fullest} meaning.

\authoremph{(2) Antaṁ -- The End}

Arahatship is the \authoremph{summum bonum} of all life's endeavour. It is the end. All that had to be done has been done \emph{(kataṁ karanīyaṁ)}. There is nothing more to come here from \emph{(nāparaṁ itthattāyāti)}.

`For the Arahat, friend, there is nothing further to be done.' (\href{https://suttacentral.net/sn22.122/en/suddhaso}{SN 22.122})

\authoremph{(3) Anāsavaṁ -- The Without Taints}

All the Taints, viz., the taints of sense-pleasure, of `self'-existence, and of Ignorance are extinct in Arahatship.

\authoremph{(4) Saccaṁ -- The Truth}

The experience of Arahatship is the experience of the highest truth, or of the highest actuality.

`For this, monks, is the highest Ariyan wisdom, that is to say the knowledge of the destruction of all Suffering. That deliverance of his is founded on truth, is unshakeable\ldots\hspace{0pt} For this, monks, is the highest Ariyan truth, that is to say, \emph{Nibbāna}, which is not a state unreal.' (\href{https://suttacentral.net/mn140/en/bodhi}{MN 140})

\authoremph{(5) Pāraṁ -- The Beyond}

\protect\hypertarget{beyond}{}{}Essentially it means that Arahatship is beyond all Suffering.

\authoremph{(6) Nipunaṁ -- The Subtle}

The experience of the Arahat is not plain to common understanding. It is deep, and cannot be comprehended through a process of mere conceptual thinking. It is comprehensible only to the wise man, and that too if he dwells upon it with Right Mindfulness \emph{(sammā sati)}.

\emph{Nipunaṁ} can also be taken to mean `accomplished' or `skilled'.

\authoremph{(7) Saduddasaṁ -- The Very Hard to See}

What is so very hard to see is that the Arahat has intention but has no thoughts of subjectivity, or that he has intention but no \emph{taṇhā}. His intentional action is completely unaccompanied by any thoughts of `I' and `mine'.

\authoremph{(8) Ajaraṁ -- The No-decay}

We have already seen its meaning. The Arahat does not decay simply because there is no `person' or `I' to decay. The changes that occur in his body are not decay to him.

\authoremph{(9) Dhuvaṁ -- The Stable}

Arahatship is \authoremph{the} stable simply because it is \authoremph{the only} state of life that \authoremph{does not} and \authoremph{cannot} change its character or nature. For instance, the Arahat can never go back to being a \emph{puthujjana}. Arahatship is irreversible.\footnote{It will be seen that the Buddha's Teaching is aimed at altering one's thinking, and altering it to the point where it can \authoremph{never more} be altered.}

\authoremph{(10) Apalokitaṁ -- The Taken Leave of}

The Arahat has `taken leave of' the world. `\authoremph{My} world' is extinct in him. So long as he lives he experiences feelings, etc., but he is neither attracted by them nor repelled by them.

\authoremph{(11) Anidassanaṁ -- The Non-Indicative}

This is one of the most important descriptions of Arahatship, yet one which is often misunderstood. \emph{Anidassanaṁ} is usually seen explained as `invisible' or `cannot be seen with the eyes'. Far from such, \emph{anidassanaṁ} refers to something very important and equally difficult to see.

Literally, \emph{anidassanaṁ} means `not pointing to' or `non-indicative'. What, however, does Arahatship not point to? Of what is it non-indicative?

The answer is: a subject (`I').

The non-Arahat has Grasping Consciousness. That is to say, to the non-Arahat, in varying degrees, things present themselves as `mine'. And as we have said earlier this presence of things as `mine' points to an `I' to whom they are present. A subject `I' is thus indicated. With the Arahat there is no presence of things as `mine'. His Consciousness is Not-Grasping \emph{(anupādā)}. No `mine' being present, no `I' is indicated. The Arahat's Consciousness therefore does not point to or indicate a subject `I'. Thus his Consciousness is non-indicative \emph{(viññāṇaṁ anidassanaṁ)}. That is why Arahatship is described as the `Non-Indicative'.

To the non-Arahat in varying degrees `things are mine'. To the Arahat, `things are'. When the life of the Arahat ceases `things are' also ceases. In other words: To the non-Arahat there is a `my world'; to the Arahat `my world' has ceased, and there is only a `world' left; when the Arahat's life ceases the `world' also ceases.

\authoremph{(12) Nippapaṁ -- The Without Impediment}

`I am' is called a \emph{papañcitaṁ}.\footnote{\href{https://suttacentral.net/sn35.248/en/bodhi}{SN 35.248}, The Sheaf of Barley} `I am' is also called a \emph{maññitaṁ} (supposition), a \emph{mānagataṁ} (gone to conceit).\footnote{\href{https://suttacentral.net/sn35.248/en/bodhi}{SN 35.248}, The Sheaf of Barley} \emph{Papañca} is sometimes taken to refer to `diffuseness in thinking'.\footnote{\href{https://suttacentral.net/an8.30/en/bodhi}{AN 8.30}, Anuruddha} From this it is clear that \emph{papañca} refers definitely to something that is a hindrance or impediment to progress. The Arahat has cut out all such impediments \emph{(chinnapapañca)}. Arahatship is therefore without impediment.

\authoremph{(13) Santaṁ -- The Peace}

In the \emph{puthujjana} there is no real peace, no real tranquillity. So long as thoughts of `I' and `mine' are present there \authoremph{cannot} be utter peace. These being absent in the Arahat he is really and truly at peace. Arahatship is the highest peace it is possible to experience.

\authoremph{(14) Amataṁ -- The Deathless}

We have seen earlier what is meant by the Arahat being deathless. With the Arahat there is no `person' to die.

\authoremph{(15) Panītaṁ -- The Excellent}

Arahatship is the most excellent experience possible.

\authoremph{(16) Sivaṁ -- The Fortunate}

Arahatship is the most fortunate purely because there is no Suffering whatsoever.

\authoremph{(17) Khemaṁ -- The Security}

Arahatship is the experiencing of the highest security. It is the highest form of security because there is no `person' or `I' to feel any insecurity. The `person' not existing, the experience is one that is completely free from insecurity.

\authoremph{(18) Tanhakkhayo -- The Desttuction of taṇhā}

The Arahat is free from all \emph{taṇhā}, of whatever kind it be.

\authoremph{(19) Acchariyaṁ -- The Wonderful}

Arahatship is the truly wonderful experience.

\authoremph{(20) Abbhūtaṁ -- The Astonishing}

Arahatship is the truly astonishing experience.

\authoremph{(21)Anītikaṁ -- The Freedom from Harm}

With the Arahat there is no `person' to be harmed. A painful feeling is experienced just in the same unattached or unaffected manner as a pleasant feeling would be.

\authoremph{(22) Anītikadhammaṁ -- The State of Freedom from Harm}

Arahatship is an experience that is beyond being harmed. It is the state of freedom from harm.

\authoremph{(23) Nibbānaṁ -- Extinction}

This is a word with a very broad meaning, and in its meaning it includes the extinction of all those that make for the \authoremph{Grasping} Groups. As we shall presently see it is extended to cover the extinction of the residual Not-Grasping Groups which happens when the life of the Arahat comes to an end.

\authoremph{(24) Avyāpajjho -- The Harmless}

In Arahatship there is no ill-will, no thoughts of causing harm, etc., whatever.

\authoremph{(25) Virāgo -- Non-Attachment}

Arahatship is described as non-attachment purely because there is no attachment of any kind whatever to things. With non-attachment there also comes the corresponding characteristic of non-resistance or non-repulsion. The Arahat is neither attracted by things nor repelled by them.

\authoremph{(26) Suddhi -- Purity}

\protect\hypertarget{suddhi}{}{}In the true and worthy sense of the word, it is only Arahatship that can be called Purity.

\authoremph{(27) Mutti -- The Release}

Arahatship is the release from all Suffering.

\authoremph{(28) Anālayo -- The Done Away With}

Usually in the context of done away with \emph{taṇhā}. The Arahat has completely done away with \emph{taṇhā} or any other thing that makes for Suffering.

\authoremph{(29) Dīpaṁ -- The Island}

Used in a metaphorical sense for safety -- safety from all Suffering. Arahatship is the island of safety.

\authoremph{(3O) Lena -- The Cave}

Again used in a metaphorical sense. Arahatship is compared to a cave which one gets into for safety from all harm, etc.

\authoremph{(31) Tānaṁ -- The Shelter}

Once again used in a metaphorical sense. Arahatship is the shelter from all harm, etc.

\authoremph{(32) Saranaṁ -- The Refuge}

Arahatship is the only refuge from all Suffering. It is so because it is only the Arahat who is completely free from all Suffering.

\authoremph{(33) Parāyanaṁ -- The Ultimate Goal}

A goal beyond Arahatship there is not. All other `goals' are nothing but various states involving Suffering to \authoremph{some} degree or other. Arahatship is wholly and entirely free from Suffering. Hence it is the ultimate goal.

\includegraphics{sectionbreak.png}

Apart from the above thirty three descriptions other descriptions for Arahatship are to be found, such as not-born \emph{(ajātaṁ)}, not-being \emph{(abhūtaṁ)} or not-made \emph{(akataṁ)}:

`Monks, there is the not-born, the not-being, the not-made, and the not-\emph{determined}. If, monks, there were not the not-born, the not-being, the not-made and the not-\emph{determined}, there would be discerned no escape here from the born, the being, the made and the \emph{determined}. But, monks, since there is the not-born, the not-being, the not-made and the not-determined, therefore an escape from the born, the being, the made, and the \emph{determined} is discernible.' (\href{https://suttacentral.net/ud8.3/en/anandajoti}{Ud 8.3})

\authoremph{Arahatship is referred to as not-born, not-being, not-made and not-determined because with regard to the Arahat there is no longer a `person' (who says `I' and `mine') that is born or being or made or determined.}

Another common description of Arahatship is the `ultimate happiness' \emph{(paramaṁ sukhaṁ)}. This `ultimate happiness' is defined by the Buddha as follows:

`Were there a going beyond the sense-pleasures of the world, that detachment is happiness. Were there a destruction of the conceit `I am', that indeed is the ultimate happiness.' (\href{https://suttacentral.net/ud2.1/en/anandajoti}{Ud 2.1})

A description of Arahatship which would interest the ethicist is that given in the \emph{Pāsādika Sutta} wherein the Buddha in describing the Arahat says:

\begin{quote}
Friend, the monk in whom the taints are destroyed is incapable of deliberately depriving a living being of life. The monk in whom the taints are destroyed is incapable of taking what is not given so that it constitutes theft. The monk in whom the taints are destroyed is incapable of indulging in sex \emph{(methunaṁ dhammaṁ)}. The monk is whom the taints are destroyed is incapable of mindfully uttering falsehood. The monk in whom the taints are destroyed is incapable of laying up treasure for indulging in pleasures as he did when being a house-holder. The monk in whom the taints are destroyed is incapable of taking a course of action through desire. The monk in whom the taints are destroyed is incapable of taking a course of action through hatred. The monk in whom the taints are destroyed is incapable of taking a course of action through delusion. The monk in whom the taints are destroyed is incapable of taking a course of action through fear. Friend, the monk who is Arahat, in whom the taints are destroyed, has done what was to be done, has laid down the burden, attained the highest, completely destroyed the fetter of \emph{bhava}, released through right knowledge, is incapable of these nine behaviours.

 -- \href{https://suttacentral.net/dn29/en/thanissaro}{DN 29}, The Inspiring Discourse
\end{quote}

The Arahat is incapable \emph{(abhabbo)} of doing these nine things. The nature of Arahatship is such that it is \authoremph{impossible} for these things to be done. The conditions that must be present if these things are to be done are not present in the Arahat, nor can they ever arise in him again.

Of all these descriptions of Arahatship the most common one, however, is that it is the destruction of lust, hatred and delusion.

\protect\hypertarget{remainder}{}{}Now, Arahatship as we saw, is the experience of the extinction of Grasping. The Five Grasping Groups are wholly and entirely extinct and what remains is a Not-Grasping residual Five Groups. These residual Five Groups are called the `Extinction element with residue' \emph{(saupādisesa nibbānadhātu)}. It is the `stuff remaining'. When Arahatship is over, i.e., when the life of the Arahat is over, the `residue' is also over. This is called `Extinction element without residue'. \emph{(anupādisesa nibbānadhātu)}. It is `without stuff remaining'. In the three phases we have, therefore, firstly Five Grasping Groups, secondly Five Groups, and thirdly the extinction of the Five Groups. The first refers to the non-Arahat, the second to the Arahat, and the third to the life-ending of the Arahat.

\begin{quote}
Monks, there are these two \emph{Nibbāna} elements. What two? The \emph{Nibbāna} element with residue and the \emph{Nibbāna} element without residue.

What, monks, is the \emph{Nibbāna} element with residue?

Here, monks, a monk is Arahat, has destroyed the taints, has lived the life, done what was to be done, laid down the burden, attained the highest goal, completely destroyed the fetter of \emph{bhava}, released by perfect knowledge. In him the five senses still remaining, these not destroyed, he experiences pleasant and unpleasant things, feels ease and pain. In him the destruction of lust, the destruction of hatred, and the destruction of delusion is called the \emph{Nibbāna} element with residue.

What, monks, is the \emph{Nibbāna} element without residue?

Here, monks, a monk is Arahat \ldots\hspace{0pt} released by perfect knowledge. But in him, monks, here itself all that are sensed, not delighted in, will become cool. This, monks, is called the Nibbāna element without residue.

 -- \href{https://suttacentral.net/iti44/en/ireland}{Iti 44}, The Nibbāna-element
\end{quote}

Often it is assumed that the descriptions of \emph{Nibbāna} such as not-born, not-being, not-made and not-\emph{determined} are descriptions of the \emph{Nibbāna} element without residue. This is a wrong assumption. Making such a wrong assumption, it is lamented that the Nibbāna element without residue is an incomprehensibility. But such a situation should not arise.

There is nothing incomprehensible in the Buddha's Teaching, though the Teaching is certainly difficult to \authoremph{see}. The Not-\emph{Determined} \emph{(asañkhata)} has been very clearly defined as Arahatship. And any synonym for Not-\emph{Determined} must also be a descriptive word for Arahatship or for the \emph{Nibbāna} element with residue.

Another \emph{Sutta} passage which describes the \emph{Nibbāna} element with residue, but is usually taken to describe the \emph{Nibbāna} element without residue, is as follows:

\begin{quote}
Monks, there is that sphere wherein is neither earth nor water nor fire nor air, wherein is neither the sphere of infinite space, nor of infinite consciousness, nor of nothingness, nor of neither-perception-nor-non-perception, wherein is neither this world nor a world beyond, nor both sun and moon. There, monks, there is no coming, I declare; no going, no persisting,\footnote{As shown earlier, \emph{thitiṁ} (persistence) is a characteristic of the \emph{saṅkhata}, i.e. of the Five Grasping Groups. It is not a characteristic of the \emph{asaṅkhata} which is Arahatship. Appearance \emph{(uppādo)}, disappearance \emph{(vayo)}, and \emph{thitiṁ} (persistence) are applicable only to a `person' or a `self' or a `somebody'. With the Arahat the latter are extinct; hence appearance, disappearance, and persistence are not applicable.} no passing away, no arising. Without support without being, without anything as object it is. This, indeed, is the end of Suffering.

 -- \href{https://suttacentral.net/ud8.1/en/anandajoti}{Ud 8.1}, Nibbāna
\end{quote}

Here again it is Arahatship or the \emph{Nibbāna} element \authoremph{with} residue that is being referred to. To get the full meaning of this passage, however, one must understand what is meant by the Four Primary Modes -- earth, water, fire and air -- `getting no footing'.

In the \emph{Kevaḍḍha Sutta}\footnote{\href{https://suttacentral.net/dn11/en/sujato}{DN 11}} we have Kevaḍḍha asking the question:

`Where do the Four Primary Modes -- earth, water, fire and air -- cease without remainder?'

The Buddha points out to Kevaḍḍha that it is not a proper question, and that the proper question should be:

`Where do (the Modes) earth, water, fire and air get no footing \emph{(nagādhati)}? Where do long and short, large and small, auspicious and inauspicious, and Name-and-Form cease without remainder \emph{(asesaṁ uparujjhati)}?'

It is necessary to see why Kevaḍḍha's question is not a proper question before we can see the significance of the question that the Buddha himself put in its place.

\protect\hypertarget{the-four-primary}{}{}The Four Primary Modes (i.e. the four primary modes of behaviour) \authoremph{purely by themselves} are not a matter for Consciousness. But their \authoremph{appearance} is a matter for Consciousness, and their `existence' is \authoremph{inferred} through the behaviour of this \authoremph{appearance}, i.e. through the behaviour of Name \emph{(nāma)}. In other words, since Name behaves in a certain fashion (e.g. while an object is perceived the percept behaves in a certain fashion) we \authoremph{infer} that the object, or that the set of behaviours, of which we are conscious behaves in that same fashion too. This means that we are really \authoremph{inferring} that the Four Primary Modes exist. Therefore, strictly speaking, we cannot say that the Four Primary Modes \authoremph{exist}. At the same time, since there is a behaviour of appearance we cannot also say that they do \authoremph{not} exist. Further, if we cannot say that they \authoremph{exist}, we cannot also say that they \authoremph{cease}. Thus Kevaḍḍha's question is improper.\footnote{The impropriety of Kevaḍḍha's question is fully within the scope of Science and the Philosophy of Science. But the same does not apply to the question that the Buddha put in its place and to its answer, the reason being that Arahatship is beyond the scope of any Science or Philosophy.}

What we \authoremph{can} rightly say is that there is a behaviour of appearance -- a behaviour which is not motivated by the individual's Consciousness but by something which he experiences as having \authoremph{no} connection with his Consciousness. The appearance keeps behaving as he maintains his awareness. What \authoremph{does} definitely exist for the individual is his being conscious of something and the appearance of that something whilst he is so conscious. Thus the Four Primary Modes get a \authoremph{footing} in this existence. And it gets this footing as the \authoremph{behaviour of appearance}. In other words, we can only say that the Four Primary Modes \authoremph{appear to exist as rūpa} (i.e. as Form or `matter') in \emph{nāma-rūpa} (Name-and-Form).\footnote{The Buddha states that Form or `matter' is dependent on the Four Primary Modes. See \href{ch-03-name-and-form-and-consciousness.xml\#dependent}{Chapter 3: Name-and-Form and Consciousness, `Monk, it is to be seen\ldots\hspace{0pt}'}. This statement is better understood at this stage.} Appearance gets a borrowed behaviour and behaviour gets a borrowed appearance.

As against what is the case with the Four Primary Modes the concepts of long and Short, large and small, auspicious and inauspicious are \authoremph{always} a matter for Consciousness. They are actually a part of Name, and therefore exist for so long as Consciousness exists only.

Now, for Name-and-Form to be there, Consciousness must be there. When Consciousness ceases, Name-and-Form ceases. When Name-and-Form ceases, the Four Primary Modes \authoremph{lose their footing in existence}, and those concepts like long and short, large and small, auspicious and inauspicious \authoremph{cease}. Therefore Kevaḍḍha's question should be as formulated by the Buddha.

Further, we have seen that cessation has two aspects, firstly the cessation of the Grasping, and secondly the cessation of the Not-Grasping Residue. In the same manner `getting a footing' also has two aspects.

With the Arahat, Grasping Consciousness has ceased. The Arahat's Consciousness is Not-Grasping \emph{(anupādā)}. That means, nothing is present to him as `mine'. Now, `mine' being absent, no `I' is indicated \emph{(anidassanaṁ)}. No `I' being present, his Consciousness is `not devoted' \emph{(ananuruddha)}'\footnote{\href{https://suttacentral.net/mn2/en/bodhi}{MN 2}, All the Taints and \href{https://suttacentral.net/sn35.94/en/bodhi}{SN 35.94}, Untamed, Unguarded} to anything (or is `not engaged' with anything) as for example the \emph{puthujjana}'s Consciousness is when he experiences a pleasant feeling. On the other hand it is `not in opposition' \emph{(appaṭiviruddha)} to anything either, as for example the \emph{puthujjana}'s Consciousness is when he experiences an unpleasant feeling. Therefore, with regard to the footing that the Four Primary Modes get and with regard to those concepts like long and short, large and small, auspicious and inauspicious, he is neither devoted to them nor is in opposition to them. They bear no \authoremph{significance} whatever to him as they do bear to the non-Arahat. Now, the Arahat's Consciousness being neither devoted to anything nor in opposition to anything, it is said to be ceased' \emph{(niruddha)}. `Non-Indicative' Consciousness (which is the Arahat's Consciousness) is therefore a Consciousness that is said to be `ceased' \emph{(viññāṇassa nirodhena)}. When Consciousness is said to be ceased, the Four Primary Modes are said to get no footing in existence. Further, Name-and- Form is also then said to be ceased, and therefore all concepts are also said to be ceased.

\emph{Viññāna nirodha} -- cessation of Consciousness -- is used to refer to the cessation of Grasping Consciousness (in which case it points to the Arahat's Consciousness, i.e. to \emph{anidassana viññāṇa} -- `non-indicative' Consciousness) as well as to the cessation of the Arahat's Consciousness which occurs when the Arahat's life ceases.

To the extent that the Arahat has Consciousness, to that extent the Four Primary Modes get a footing, and there is the presence of the concepts of long and short, etc. But these have nothing whatever to do with Grasping; and as a result the Arahat's Consciousness being neither devoted to them nor obstructed by them, they bear no significance whatever. When the Arahat's Consciousness ceases with the laying down of life the Four Primary Modes get no footing whatsoever, and likewise the concepts of long and short, large and small, auspicious and inauspicious, and Name-and-Form cease without any remainder whatsoever.

Therefore the answer to the question is:

\begin{quote}
The non-indicative Consciousness, the without end;\footnote{\emph{Anantaṁ} (without end) should probably be taken to mean `without aim' or `without objective'.} the all given up\footnote{\emph{Pahaṁ}, as a shortened form of \emph{pajahaṁ} so as to maintain the metre in the verse, and meaning `given up entirely', fits in here very much better than \emph{pabhaṁ}.} -- there it is where earth, water, fire and air get no footing. There it is where long and short, large and small, auspicious and inauspicious, and Name-and-Form cease without remainder; with the ceasing of Consciousness, these cease.
\end{quote}

The Arahat's Consciousness does not take anything as an object for holding \emph{(anārammanamevetaṁ)}. The holding or the Grasping is over, and so the subject ('I') is over. The subject ('I') being over, `my world' \emph{(loko)} is over, a `world beyond' is over; coming, going, birth, death are all over; Suffering is over.

\begin{quote}
For him who clings there is agitation. For him who clings not there is no agitation. Agitation not being, there is calm. Calm being, there is no inclination. Inclination not being, there is no coming, no going. Coming and going not being, there is no decease-and-birth. Decease-and-birth not being, there is no `here' nor `yonder' nor anything in between. This, indeed, is the end of Suffering.

 -- \href{https://suttacentral.net/ud8.4/en/anandajoti}{Ud 8.4}, Nibbāna
\end{quote}

Clearly this refers to Arahatship. `For him who clings not' means `for the Arahat.'

These passages from the \emph{Udāna} just quoted are misconstrued to refer to the \emph{Nibbāna} element without residue only because attempts are made to understand them \authoremph{verbally}. If seeing and understanding the Buddha's Teaching is only a matter of verbally understanding the \emph{Sutta}, then one can be an Arahat in next to no time. The \emph{Nibbāna} element without residue is also seen described by meaningless words like `Absolute', `Unconditioned', and so on, only because of a lack of understanding, which in turn is born of the attempt to understand the Teaching verbally. Further, it is sometimes thought that the \emph{Nibbāna} element without residue is some kind of metaphysical existence which has nothing to do with the Five Groups, yet, that it is an eternal existence of some sort or other. Such a view can arise owing to the presence of that very subtle form of Grasping -- `\emph{Nibbāna} is mine, he conceives' \emph{(nibbānaṁ meti maññati)} -- which the Buddha refers to in his Discourse on The Fundamentals of All Things.\footnote{\href{https://suttacentral.net/mn1/en/bodhi}{MN 1}}

\includegraphics{sectionbreak.png}

The Buddha Said: `All \emph{determinations} are Impermanent, all things are Not-self, all \emph{determinations} are Suffering' \emph{(sabbe saṅkhārā aniccā, sabbe dhammā anattā, sabbe saṅkhārā dukkhā)}. The following question can arise here: whilst saying that all \authoremph{things} are Not-self, why did the Buddha say that all \emph{determinations} are impermanent and Suffering? In other words, whilst saying that all things are Not-self, why did he say that \authoremph{all things upon which other things depend} are Impermanent and Suffering? Why did he not \authoremph{directly} say all \authoremph{things} are Impermanent and Suffering as he did with regard in the characteristic of Not-self?

The answer is that there is a distinct purpose in his Teaching. He does not say things seeking others' approval of them. Nor does he set out to \authoremph{explain} or \authoremph{analyse} things. He has just one intention underlying his Teaching. That is, purely and simply, to lead the follower towards the extinction of Suffering. And this extinction of Suffering is at one and the same time the extinction of all notions of `self' and of all thoughts of `I' and `mine'. The purpose of the Teaching is not to save `self' but to be saved \authoremph{from} `self'.

Thus the Buddha does not take one directly towards a thing's impermanence. He takes one towards it in an indirect manner, and that is more effective. He shows that a thing is impermanent by showing that the things upon which that thing depends are impermanent. Then, since the thing is impermanent, he shows that it is Not-self.

It should therefore be clear that this triad -- `All \emph{determinations} are Impermanent, all things are Not-self, all \emph{determinations} are Suffering' -- is not an exposition of things pure and simple. It includes a definite \authoremph{way} of teaching.

This fact is lost sight of, and then in a conceptual manner various reasons are adduced for its particular form. The most common of these reasons appears to be that in this triad the word `thing' \emph{(dhamma)}, unlike the word `\emph{determinations}' \emph{(saṅkhāra)}, includes \emph{Nibbāna} also. In other words it is often thought that the reason for the Buddha saying `all \emph{determinations} are Impermanent, all things are Not-self' without saying `all things are Impermanent, all things are Not-self' is that he wanted \emph{Nibbāna} too to be included as something Not-self.

But this is a wrong notion, and it is arrived at in the following manner:

To begin with, the word \emph{saṅkhāra} is taken to mean `\emph{determined}'. That is, it is taken to be the same as \emph{saṅkhata}. This, as we have seen, is wrong. \emph{Sankhāra} means something which \emph{determines} some other thing, i.e., a \emph{determination}, or a \emph{determinant} Now, \emph{Nibbāna} has been described as the Not-\emph{Determined}, i.e., as \emph{asaṅkhata}. On the face of this description of \emph{Nibbāna} it cannot be included in the word \emph{saṅkhāra} which is now wrongly taken to be the same as \emph{saṅkhata}. Therefore a word which embraces both \emph{saṅkhata} and \emph{asaṅkhata} has to be found. That would be \emph{dhamma} (thing). Since the Buddha wanted \emph{Nibbāna} also to be described as Not-self the word \emph{dhamma} was used.

Such is the wrong argument through which this wrong notion is arrived at.

But the \emph{Nibbāna} element, with or without residue, has \authoremph{nothing whatever} to do with `self' \authoremph{or} Not-self. In \emph{Nibbāna} there is no deception of a `self' whatever, which means that there is no such `self' \authoremph{to be denied}. There is no necessity whatever for Not- self. The question of Not-self arises only when the question of `self' arises. \emph{Nibbāna} is beyond both `self' and Not-self. The Arahat has no notion whatever of `Self'. Hence the Arahat has no occasion whatever to see anything as Not-self. Seeing things as Not-self is only the \authoremph{path} to Purity\footnote{Purity refers to Arahatship. See \protect\hyperlink{suddhi}{Suddhi -- Purity\ldots\hspace{0pt}}} (or to \emph{Nibbāna}). It is \authoremph{not} Purity. `All things are Not-self. When this is seen with wisdom, one wearies oneself of Suffering. This is the path to Purity.'\footnote{\href{https://suttacentral.net/dhp273-289/en/anandajoti}{Dhp 279}} The Arahat \authoremph{has arrived} at Purity and lives in Purity. He has come to the end of `Not, this is my self'.

With the Five Grasping Groups there is a deception of a `self'. Something appears as `self'. But this thing which appears as `self' is really not a self. That is to say, it is Not-self. The `self' of the Five Grasping Groups is \authoremph{not} a self, since no self of any kind whatever is to be found at all anywhere. Therefore this `self' has to be seen as Not-self.

With the residual Not-Grasping Groups of the Arahat there is no apparent `self' to be found. There \authoremph{nothing} appears as `self'. Hence no seeing anything as Not-self arises.

Again: Though no self actually is to be found, things are being seen as `self' or Not-self. And seeing things as `self' precedes seeing things as Not-self. The Arahat has come to the end of all seeings. And in \emph{Nibbāna}, which is the experience of the Arahat, there is no question of a seeing things as Not-Self, since there is no question of a `self' arising at all.

Perhaps an analogy would help to make this matter clearer. Let us imagine two deer gazing at the sun shining upon the sand. One of them is an ordinary deer, and being ordinary it sees `water' as it gazes at the said phenomenon. To this deer there is the problem of `water'. It has to be told that what it is taking for `water' is not-water, and that it is merely the sun shining upon the sand. Now let us imagine that the second deer has perfect understanding and clear penetrative vision. To this deer, its vision being so perfect, no `water' appears at all. It also understands fully well that it is gazing at the sun shining upon the sand. To this deer there is nothing to be taken as `water' or as not-water. Suppose we now tell this clear visioned deer that the phenomenon it is gazing at is not-water, it will look at us and say, `What on earth are you speaking about?'

The confusion seems to lie in assuming that when the Buddha says some \emph{dhamma} is \emph{anattā}, what the Buddha purely and simply means by it is that \authoremph{in} that \emph{dhamma} there is no \emph{attā}. Such an assumption is a very grave lapse, seriously misleading, and missing the vital point. (To indicate that there is no permanent self-existent thing anywhere, a Buddha is not necessary. A Hume would do for that. Let alone \authoremph{in} the Arahat, even \authoremph{in} the \emph{puthujjana} there is no actual self.) This type of assumption will only lead us to the conclusion that, with regard to the problem of `self', there is really no difference between the Arahat and the \emph{puthujjana}. So that it will not lead us anywhere; since the real culprit -- that is, the \authoremph{deception} of `self' (which is there for the \emph{puthujjana}, but not there for the Arahat) -- has been beautifully allowed to escape notice, and so will continue to remain as strong as it ever was. This is precisely what happens with the individual who thinks that when the Buddha says some \emph{dhamma} is \emph{anattā}, all that is meant by it is that \authoremph{in} the \emph{dhamma} there is no \emph{attā}. He further seeks confirmation of this verbal understanding by analysing the Five Groups into infinitesimal bits and pieces with the lofty equanimity of the scholar, and to his great satisfaction (since his verbal understanding is being confirmed) he sees no actual self anywhere. In fact he could well spare himself the trouble of such fine analysis and yet see that there is no self to be found anywhere. But -- and that is the vital point -- in spite of all his masterly analysis, he still \emph{looks upon the Five Grasping Groups as `self'}; more precisely, as `my self'.

In this triad -- \emph{sabbe saṅkhārā aniccā, sabbe dhammā anattā, sabbe saṅkhārā dukkhā} -- the meaning of \emph{sabbe dhammā anattā} is: All things (which are taken as `self') are Not-self. Thus it does not apply to Arahatship or \emph{Nibbāna}.

As we have said earlier the Buddha is teaching with a definite purpose. He does not have to help us remove a self that actually does not exist. He is helping us to remove the \authoremph{notion} of `self' that exists with us. And he, and \authoremph{only} he, can help us to remove this notion. His Teaching is one that is designed to lead on towards a specific goal. That is also why he says that the \emph{saṅkhārā} are \emph{aniccā}, without directly saying that the dhamma (which are \emph{saṅkhatā}. and dependent on \emph{saṅkhārā}) are \emph{aniccā}. Further, his Teaching is also one that is `well said' \emph{(svākhāto)}. But it is also necessary that we understand it well.\footnote{Note the following statement of the Buddha: `Dependent on two things, monks, is there the arising of wrong view. What two? Voice from beyond, and improper attention. Dependent on these two things, monks, is there the arising of wrong view.' `Dependent on two things, monks, is there the arising of right view. What two? Voice from beyond, and proper attention. Dependent on these two things, monks, is there the arising of right view.' (\href{https://suttacentral.net/an2.118-129/en/sujato}{AN 2.125-126}) `Voice from beyond' \emph{(Parato ghoso)} refers to the voice of an Arahat, the `beyond' referring to Arahatship. See \protect\hyperlink{beyond}{Pāraṁ -- The Beyond\ldots\hspace{0pt}}}

`What is impermanent, that is Suffering; what is Suffering, that is Not-self' \emph{(yad aniccaṁ taṁ dukkhaṁ, yaṁ dukkaṁ tad anattā)}.\footnote{\href{https://suttacentral.net/sn22.15/en/bodhi}{SN 22.15: What is Impermanent}} Here again, the Buddha is showing the person who is seeing things as `self' how and why those things are Not-self. Wherever a `self' is asserted the Buddha rejects it, and shows that there is no basis to consider anything as a self. He does not have to do that with the Arahat. These three characteristics of Impermanence, Not-self and Suffering always stand or fall together. \emph{Nibbāna}, with or without residue, is \authoremph{beyond} all these three characteristics.
