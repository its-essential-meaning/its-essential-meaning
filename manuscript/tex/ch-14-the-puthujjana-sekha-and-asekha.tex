\chapter{The Puthujjana, Sekha and Asekha}

There are three distinct classes of individuals that the Buddha's Teaching brings out, i.e. the \emph{puthujjana}, the \emph{sekha} (Learner), and the \emph{asekha} (Learning-ender). The fundamental differences between them are as follows:

\begin{enumerate}
\def\labelenumi{\arabic{enumi}.}
\item
  The \emph{puthujjana} regards things as `mine'.
\item
  The \emph{sekha} (learner) knows and sees that the notion `mine' is wrong but the notion `mine' still arises in him. Therefore he regards things as `not mine'.
\item
  The \emph{asekha} (Learning-ender) not only knows and sees fully that the notion `mine' is wrong, but also no thoughts of `mine' whatsoever arise in him. Thus he neither regards things as `mine' nor as `not mine'. The \emph{asekha} is the Arahat.
\end{enumerate}

From these fundamental differences follow all other differences that lie between them. We can therefore indicate the differences between them in other ways too. As an example, we may speak of their differences as follows; the \emph{puthujjana} does not see the Buddha's Teaching; the \emph{sekha} sees the Buddha's Teaching but has not fully experienced it; the \emph{asekha} not only sees the Buddha's Teaching but also \authoremph{fully} experiences it.

To the extent the \emph{sekha} experiences the Teaching, to that extent does he see the Teaching better. His seeing is still imperfect and his cultivation of the Four Applications of Mindfulness\footnote{See \href{ch-16-satipatthana.xml\#start}{Chapter 16: On the Four Applications of Mindfulness}} is only partial (\emph{catunnaṁ kho āvuso satipaṭṭhānānaṁ padesaṁ bhāvitattā})\footnote{\href{https://suttacentral.net/sn47.26/en/bodhi}{SN 47.26}, Partly} as against that of the \emph{asekha} which is complete (\emph{satipaṭṭhānānaṁ samattaṁ bhāvitattā}).\footnote{\href{https://suttacentral.net/sn47.27/en/bodhi}{SN 47.27}, Completely} He, the \emph{sekha}, can see a part of the Teaching without experiencing it. But seeing and understanding in the \authoremph{fullest} sense comes only with the experience of it. It is therefore to be expected that there would be various classes of \emph{sekhas} depending on the extent to which they experience the Teaching.

In the \emph{Saṁyutta Nikāya} we have the difference between the \emph{sekha} and the \emph{asekha} denoted as follows:

\begin{quote}
Again, monks, the monk who is a \emph{sekha} knows the five controlling faculties: faith, energy, mindfulness, concentration, and wisdom. Yet he neither lives experiencing with the body, nor penetratively sees with wisdom, what they lead to, their excellence, their fruit, and their end \ldots\hspace{0pt} Here, monks, a monk who is an \emph{asekha} knows the five controlling faculties: faith, energy, mindfulness, concentration, and wisdom. He lives experiencing with the body, penetratively seeing with wisdom, what they lead to, their excellence, their fruit, and their end.

 -- \href{https://suttacentral.net/sn48.53/en/sujato}{SN 48.53}, A Trainee
\end{quote}

The \emph{sekha} has faith in the fact that deathlessness can be attained by developing the five controlling faculties. He `stands knocking at the door of deathlessness' -- \emph{amara dvāraṁ āhacca tiṭṭhati.}\footnote{\href{https://suttacentral.net/sn12.49/en/bodhi}{SN 12.49}, The Noble Disciple (1)} But the \emph{asekha} having developed these faculties fully has achieved deathlessness and lives experiencing deathlessness. Arahat Sāriputta said that the latter was the case with him. Incidentally, it is wrong to think that the \emph{puthujjana} has these faculties. The \emph{puthujjana} does not possess these faculties. It is the \emph{sekha} who has acquired them, but of course he has to develop them.\footnote{\href{https://suttacentral.net/sn48.12/en/sujato}{SN 48.12} and \href{https://suttacentral.net/sn48.18/en/sujato}{SN 48.18}}

Khemaka, an \emph{Anāgami} (i.e. one of the higher classes of \emph{sekhas}) says:

\begin{quote}
Though, friends, I discern in the Five Grasping Groups no self nor aught pertaining to self, yet I am not Arahat, nor one in whom the taints are destroyed. Though, friend, in the Five Grasping groups is found `I am', yet I do not discern that I am this `I am'.

\begin{enumerate}
\def\labelenumi{\roman{enumi}.}
\item
  Friends, though an Ariyan disciple has put away the five lower fetters, yet there remains in his Five Grasping Groups the conceit `I am', the desire `I am', the tendency `I am', still not removed from him.
\end{enumerate}

 -- \href{https://suttacentral.net/sn22.89/en/bodhi}{SN 22.89}, Khemaka
\end{quote}

The \emph{puthujjana} is \authoremph{not} on the Path to \emph{Nibbāna}, whilst the \emph{sekha} is on the Path. And once an individual is on the Path, the Buddha teaches that he is assured of arriving at \emph{Nibbāna}. The \emph{sekha} is assured of being an \emph{asekha} (i.e. Arahat). The \emph{asekha} or the Arahat \authoremph{has trod the Path to completion} and has arrived at the goal of \emph{Nibbāna}. He lives experiencing \emph{Nibbāna}.

Now, all \emph{sekhas} and the \emph{asekhas} are referred to as \emph{Ariyas}. Thus, in the broadest classification we have \emph{puthujjanas} and \emph{Ariyas}. Literally, \emph{Ariya} means Noble.

We may now examine in some detail the various categories of individuals classed as \emph{puthujjanas} and \emph{Ariyas}.

Two classes of \emph{puthujjanas} can be distinguished:

\begin{enumerate}
\def\labelenumi{\arabic{enumi}.}
\item
  The \emph{assutavā} (one who has not heard). He has not heard the Buddha's Teaching, and so he holds views contrary to the Teaching.
\item
  The \emph{anulomikāya khantiyā samannāgata} (one possessed of acquiescence in agreement). He has heard the Teaching, and he possesses tacit agreement with the Teaching. But he is not one who has merely studied the Teaching and or professes to follow the Teaching. He is much more than that. He makes a genuine attempt to realize the Teaching in himself, knowing that he still does not actually see the Teaching.\footnote{\href{https://suttacentral.net/an6.101/en/sujato}{AN 6.101}, Extinguished}
\end{enumerate}

Of these two kinds of \emph{puthujjana} the \emph{assutavā} is me kind that is to be found more often.

There are eight classes of \emph{Ariyas}. In ascending order of perfection they are:

\begin{enumerate}
\def\labelenumi{\arabic{enumi}.}
\item
  \emph{Sotāpattiphalasacchikiriyāya paṭipanno} -- The one who practices for the realization of the fruit of Steam-entrance.
\item
  \emph{Sotāpanno} -- The Stream-entrant.
\item
  \emph{Sakadāgāmiphalasacchikiriyāya paṭipanno} -- The one who practises for the realization of the fruit of once-return.
\item
  \emph{Sakadāgāmi} -- The Once-returner.
\item
  \emph{Anāgāmiphalasacchikiriyāya paṭipanno} -- The one who practises for the realization of the fruit of non-return.
\item
  \emph{Anāgāmi} -- The Non-returner.
\item
  \emph{Arahattāya paṭipanno} -- The one who practises for the realization of Arahatship.
\item
  \emph{Arahā} -- The Arahat, or the Consummate One.
\end{enumerate}

Of these eight, the first seven are not yet Arahat. That is, they are still not Consummate or Perfect, and have still more work to do. Thus they are called \emph{sekhas} (Learners). But they are all on the Path, and are assured of becoming Arahats or Consummate Ones. They have crossed from the plane of the \emph{puthujjana} (\emph{puthujjana-bhūmi}) to the plane of the Noble (\emph{ariya-bhūmi}). The last and the eighth, i.e. the Arahat, has done whatever was to be done, has finished training, has achieved the goal, has laid down the burden, has attained the Consummate state, has attained \emph{Nibbāna}. He is therefore \emph{asekha} (Learning-ender).

Thus, with the \emph{puthujjana}, we have nine kinds of individuals.\footnote{\href{https://suttacentral.net/an9.9/en/sujato}{AN 9.9}, Persons} If we take into account the two types of \emph{puthujjanas} we then have ten kinds of individuals.

It will be seen that of the seven \emph{sekhas} there are fruit-attainers (\emph{phala-lābhi}), i.e. the \emph{sotāpanna}, the \emph{sakadāgāmi} and the \emph{anāgāmi}. The remaining four are practising for the realization of the corresponding fruit. Thus they are called path-attainers (\emph{magga-lābhi}). They have attained to the path which will lead them to the corresponding fruit. The \emph{asekha} is also a fruit-attainer. He has attained to the fruit of Arahatship. The notion that the attainment of the fruit is immediately followed by the attainment of the path is wrong. This notion found in certain Commentaries is not in keeping with the Suttas wherein the path-attainer is definitely said to be practising for the realization of the fruit. There is therefore a time interval between path-attainment and fruit-attainment.

\begin{quote}
Here, friends, a monk develops insight preceded by serenity. In thus developing insight preceded by serenity, the Path is born. He pursues that Path, develops and practises it. In him thus pursuing, developing, and practising that Path, the fetters are put away, and the latencies cease.

 -- \href{https://suttacentral.net/an4.170/en/thanissaro}{AN 4.170}, In Tandem
\end{quote}

However, the first class of Path-attainer \authoremph{shall always} attain the fruit before his death even if that fruit-attainment be just before death. According to the Suttas one of two thing makes this fruit attainment possible -- diligent work, or the crisis of approaching death which provide the necessary impetus to attainment.\footnote{\href{https://suttacentral.net/sn25.1/en/sujato}{SN 25.1}, The Eye} Unlike the \emph{puthujjana} who is subject to retrogression the \emph{sekha} whether he be a path-attainer or a fruit-attainer progresses towards the goal.

The Buddha teaches that there are ten fetters which bind beings to \emph{bhava}, and the \emph{sekhas} who are fruit-attainers are generally described in terms of the various fetters they have broken. These fetters are (1) `person'-view (\emph{sakkāyadiṭṭhi}), (2) Doubt (\emph{vicikicchā}), (3) Practice of rites and ritual (\emph{sīlabbata-parāmāso}), (4) Desire for sense-pleasure (\emph{kāmacchando}), (5) Ill-will (\emph{vyāpāda}), (6) Attachment to Form (\emph{rūparāgo}), (7) Attachment to no-Form (\emph{arūparāgo}), (8) Conceit (\emph{māno}), (9) Restlessness (\emph{uddhaccaṁ}), and (10) Ignorance (\emph{avijjāṁ}). The first five are described as lower fetters (\emph{orambhāgiyāni saññojanāni}) whilst the other five are described as higher fetters (\emph{uddhambhāgiyāni saññojanāni}).\footnote{\href{https://suttacentral.net/an10.13/en/bodhi}{AN 10.13}, Fetters}

The first fruit-attainer is called \emph{sotāpanno} (Stream-entrant). He has destroyed the first three fetters of `person'-view, doubt, and practice of rites and ritual. Entered the Stream (i.e. entered the \emph{sota}) means got on to the Noble Eightfold Path, the Stream (\emph{sota}) being defined as this path:

\begin{quote}
The Stream, Sāriputta, is just this Noble Eightfold Path, that is to say, right understanding, right thinking, right action, right speech, right living, right effort, right mindfulness, right concentration.

 -- \href{https://suttacentral.net/sn55.5/en/sujato}{SN 55.5}, With Sāriputta (2nd)
\end{quote}

The second fruit-attainer is called \emph{sakadāgāmi} (Once-returner). He has destroyed the first three fetters and reduced lust, hatred and delusion (\emph{tiṇṇaṁ saññojanānaṁ parikkhayā rāgadosamohānam tanutta}); Therefore he has not only destroyed the first three fetters but also has partly overcome the fourth and the fifth fetters, namely desire for sense-pleasure and ill-will. The third fruit-attainer is called \emph{anāgāmi} (Non-returner). He has destroyed the first five fetters, i.e., the lower fetters. The fourth and last fruit-attainer is of course the Arahat who has destroyed all the ten fetters.

The first path-attainers, i.e., those practising for the realization of the fruit of Stream-entrance, are of two kinds -- the \emph{dhammānusāri} (Dhamma-striver) and the \emph{saddhānusāri} (Faith-striver). These two have just crossed over from the plane of the \emph{puthujjana} to the plane of the \emph{Ariya}. The \emph{dhammānusāri} is one who through wisdom is pleased with the Dhamma to an extent, whilst the \emph{saddhānusāri} is one who through faith is firmly attached to Dhamma.\footnote{\href{https://suttacentral.net/mn70/en/bodhi}{MN 70}, At Kīṭāgiri and \href{https://suttacentral.net/sn25.1/en/sujato}{SN 25.1}, The Eye} As stated earlier they are both incapable of passing away without realizing the fruit of Stream-entrance, i.e., without becoming \emph{sotāpanna}.

The maximum number of lives left for the \emph{sotāpanna} is seven (\emph{sattakkhattuṁ paramatā}). Further, none of these seven lives will be in an unfortunate sphere. He is assured of \emph{Nibbāna} or Enlightenment within this period (\emph{niyato sambodhi-parāyano}). The \emph{sakadāgāmi} returns once more to this world and accomplishes the destruction of Suffering (\emph{sakideva imaṁ lokaṁ āgantvā dukkhassantaṁ karoti}). The \emph{anāgāmi}, when he dies here, will be reborn spontaneously in the Pure Abodes and attains to Extinction there.\footnote{\href{https://suttacentral.net/an3.88/en/sujato}{AN 3.88: Training (3rd)} and \href{https://suttacentral.net/an3.89/en/sujato}{AN 3.89: Three Trainings (1st)}}

All this means that, as a cart pushed just over the hilltop will roll down by its own weight without extra effort, so will the \emph{sotāpanna} in any case end up in \emph{Nibbāna} within a maximum of seven further lives. The Buddha however exhorts all \emph{sekhas} to act with diligence (\emph{appamādena karaṇīyan}) and try to make an end of it all in this life itself by attaining Arahatship.

\begin{quote}
Monks, just as a little bit of faeces is foul smelling, even so do I not praise \emph{bhava}, not even for so brief a time as is needed for a finger snap.

 -- \href{https://suttacentral.net/an1.316-332/en/sujato}{AN 1.328}
\end{quote}
