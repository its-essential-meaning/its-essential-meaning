\chapter{Rebirth}

A discussion on rebirth is usually a discussion on a subject concerning which there is no personal experience among those who discuss it. Also, for the task of seeking a solution to the \authoremph{present} problem of Suffering in the \authoremph{present} itself, a study of rebirth is not essential. For these reasons this chapter will not be a discussion on the subject of rebirth proper. Nevertheless there are a couple of matters regarding rebirth which are worth of thought. This chapter will therefore limit itself to a discussion of those matters.

However much one may argue and infer that it is only through the Buddha's doctrine of rebirth that the variegated inequalities of human beings could be accounted for, there yet remains a certain amount of doubt about it until one \authoremph{sees} rebirth. Until then, to some extent or other, one has trust in the Buddha with regard to the matter.

On the other hand the Buddha demands no belief in rebirth from one whose sole aim is to end Suffering, nor does he insist that one must see rebirth if one is to come to the end of Suffering. As we pointed out earlier, the Suttas speak of Arahats who saw rebirth and could recollect past lives as well as of those who could \authoremph{not}.\footnote{For example: \href{https://suttacentral.net/sn16.9/en/bodhi}{SN 16.9: Jhanas and Direct Knowledges}} Certainly belief in rebirth would be very useful in that it would act as an urge to reach the Path as fast and as diligently as possible, and so get to \emph{Ariyabhumi} (plane of the \emph{Ariyas}) at least, in this life. But it is not absolutely essential.\footnote{It is sometimes thought that the most effective way of convincing one of the validity of the Buddha's Teaching is to prove to one that there is rebirth. This is not so. We find that even ascetics who could see rebirth and could recollect their past lives did not always accept the Buddha's Teaching In fact, their very seeing rebirth and recollecting past lives made them come to wrong View. For example, see \href{https://suttacentral.net/dn1/en/bodhi}{DN 1}. Far too much time seems to be spent on the subject of rebirth by those interested in the Buddha's Teaching. If this time is spent by them in trying to see here and now itself a solution to the problem of their present existence, they are bound to be benefited much more, and in fact will also be attracted towards the Buddha's Teaching much more.}

The present problem of my existence, which is just the problem of my \authoremph{present} Suffering, is to be solved \authoremph{here} and \authoremph{now} with no reference to a past life or a future life. Whether there will be or whether there will not be a renewed existence (\emph{punabbhava}) for the \emph{puthujjana}, it is clear that for the Arahat there can be no renewed existence in the future. The Arahat has already done away with birth (\emph{jāti}) and existence (\emph{bhava}) here itself, these never to arise again. Likewise, it is clear that for the \emph{sekhas}, whatever rebirth awaits them, it cannot be in spheres of unfortunate or unpleasant experience, for the simple reason that the key factor which conditions such experience is out in them -- \emph{sakkāyadiṭṭhi}, and that thoughts of `I' and `mine' keep steadily declining in them. \emph{Taṇhā}, desire, attachment -- these factors which upbring all the Suffering are thereby greatly reduced in the \emph{Sekhas}, and hence Suffering is greatly reduced. Whatever new existence awaits them after death here, that will be an existence with a very reduced degree of Suffering. Indicating the magnitude of the \emph{sotāpanna}'s achievement with a simile, the Buddha points out that the Suffering the \emph{sotāpanna} has destroyed is as vast as the earth whilst the Suffering he will have to endure in the future during a maximum of a further seven lives is as small as the bit of soil he placed on his finger nail.\footnote{\href{https://suttacentral.net/sn13.1/en/sujato}{SN 13.1: A Fingernail}} With regard to the relative value of the \emph{sotāpanna}'s achievement it is said:

\begin{quote}
Better than sole kingship of the earth, better than going to heaven, better than supreme rulership of all the worlds, is the fruit of Stream-entrance (\emph{Sotāpattiphalaṁ}).

 -- \href{https://suttacentral.net/dhp167-178/en/sujato}{Dhp 178}
\end{quote}

There are many passages in the Suttas\footnote{For example, in \href{https://suttacentral.net/mn135/en/bodhi}{MN 135}} where we have the Buddha teaching in a rather general manner how one is reborn in accordance with one's deeds. That is, he teaches that the rebirth awaiting a person is \authoremph{in accordance with} his \emph{kamma}.\footnote{\href{https://suttacentral.net/mn136/en/thanissaro}{MN 136}}

It should also be noted that the subject of rebirth need not always remain a matter of trust in the Buddha. The Buddha has shown the course by following which one could recollect his own past lives and \authoremph{see} other beings dying here and being born there according to their deeds.\footnote{\href{https://suttacentral.net/mn77/en/bodhi}{MN 77}} To the individual who has achieved this vision the subject of rebirth no longer remains a matter of trust in the Buddha. To such an individual, rebirth is a matter of certainty. The Buddha said that he himself could recollect the past as far back as he wished.\footnote{\href{https://suttacentral.net/mn29/en/bodhi}{MN 29}}

On many an occasion the not yet enlightened monks went to the Buddha, either for inspiration or from curiosity, and inquired from him as to where some departed one had been reborn. As one reads through these passages in the Suttas, one imagines the Buddha smiling to himself at these questions and giving the answers and the reasons for them, as one who answers children, wishing to soothe them. In the \emph{Mahāparinibbāna Sutta}\footnote{\href{https://suttacentral.net/dn16/en/bodhi}{DN 16}} we have him telling Ānanda who asked these questions:

\begin{quote}
Now there is nothing strange in this, Ānanda, that a human being should die; but that as each one does you should come to me and inquire about him in this manner -- that is wearisome to me. I will, therefore, preach to you a way of Dhamma called the Mirror of Dhamma, which if the Ariyan disciple possesses, he may, if he should so desire, himself predict to himself: Torment is destroyed for me; so is the animal womb, the \emph{peta} realm. Destroyed is the falling into hells, into states of woe. I am \emph{sotāpanna}, I am of the nature not to fall away, and am assured of attaining the goal of Enlightenment.
\end{quote}

From this statement of the Buddha it would also appear that the \emph{sotāpanna} (even though he may not actually \authoremph{see} rebirth) has some sort of self-assurance that whatever rebirth awaits him, it will not be in an unfortunate sphere. It would not be idle speculation to reflect on this important characteristic of the \emph{sotāpanna} and try to see whether any adequate reason lies for its being so.

The Buddha says that the \emph{sotāpanna} will not be reborn in the animal world, the world of the \emph{petas}, and such other unfortunate worlds. Of these worlds the world best known to us is the animal world. For this reason, we may limit our discussion to the beings in this world, i.e., to the animals.

There is really no \authoremph{fundamental} or \authoremph{basic} difference between the \emph{puthujjana} and the animal in as much as both regard Form, Feeling, Perception, \emph{Determinations} and Consciousness as `mine'. The \emph{puthujjana} can however (though perhaps not in all cases) \authoremph{develop} himself so as to regard these as `\authoremph{not} mine', in which case of course he no longer remains a \emph{puthujjana}. This development is not available to the animal, so that there can be no animal which will be regarding its Form, Feeling, Perception, \emph{Determinations} and Consciousness as `not mine'. The fundamental attitude of the \emph{puthujjana} and the animal being the \authoremph{same}, and the fundamental attitude as between the \emph{sotāpanna} and the animal being \authoremph{opposite}, it would appear that whilst the \emph{puthujjana} can be reborn an animal the \emph{sotāpanna} \authoremph{cannot}. The fundamental characteristic of the \emph{sotāpanna}'s mentality being `not mine', he cannot be born in a realm where the mentality of each and every being in it has the fundamental characteristic of regarding things as `mine'. The realms of the \emph{petas}, etc., would also appear to be those in which each and every being possesses this latter mentality.

Viewed from the angle of rebirth it is rather frightening, and it indicates that the \emph{puthujjana} is in a perilously insecure position. If he cannot develop himself to the extent of becoming a \emph{sotāpanna} he must at least try his best to understand and practise the Buddha's Teaching.

One, however, comes across the individual who argues thus: however much life may be Suffering, I will make the most of it and die; for, why should I sacrifice all my sense-pleasures and make so great an endeavour as to tread the Noble Eightfold Path if I cannot get a certain proof that I shall be reborn when I die?

At first glance, this may appear a practical and sensible argument. But such an individual is no destroyer of Suffering. Leaving aside the fact that he does not actually \authoremph{see} that life is Suffering, he does not even see that he has a problem. In him one discerns only a looking for reasons to support a way of life for which he longs. And even if such a person be given a certain proof of rebirth such as he may desire, it is a matter of grave doubt whether he will choose to practise the Buddha's Teaching or to merely do deeds that he thinks will ensure for him a more fortunate life hereafter.

What can Buddhism do with such people? It can only wait for them, patiently and with compassion, until some rude shock has awakened them to the true characteristics of their existence, when they may come to it as genuine thinkers.

Rebirth or no rebirth, each and every individual (save of course the Arahat) is undergoing Suffering. But it is only a very small proportion that can see even the unsatisfactory and disquieting nature of existence. And it is only this small proportion that has the potential to become genuine followers of the Buddha.

In the final analysis, it all comes down to one's attitude towards the problem of one's own existence. Do I have a \authoremph{present} problem which I must solve in the \authoremph{present} itself, or do I not? If I \authoremph{do have} such a problem, then all discussions on past lives and future lives can certainly wait.
