\chapter{A Note on the Four Applications of Mindfulness}

\textit{(Cattāro Satipaṭṭhānā)}

\protect\hypertarget{start}{}{}The actual way of living out the Noble Eightfold Path for the development of Wisdom and therewith gaining deliverance from Suffering, or for attaining Arahatship, is the practising of the Fourfold Applications of Mindfulness (\textit{cattāro satipaṭṭhānā}). How one practises this Way of Mindfulness is given in the Discourse called the \textit{Mahā Satipaṭṭhānā Sutta}.\footnote{\href{https://suttacentral.net/dn22/en/sujato}{DN 22} and \href{https://suttacentral.net/mn10/en/sujato}{MN 10}}

It is sometimes thought that the practice of this Way of Mindfulness can be undertaken without any prior understanding of the Buddha's Teaching. This is wrong. To the one who examines the \textit{Satipaṭṭhānā Sutta} carefully it is quite clear that there must be a good understanding of the Teaching if one is to embark on the practice of the four \textit{satipaṭṭhānas} so to obtain any beneficial results. Repeatedly the \textit{Satipaṭṭhānā Sutta} says `abides seeing the nature of things in things' (\textit{dhammesu dhammānupassī viharati}), and this abiding is defined as understanding or knowing as it really is (\textit{yathābhūtaṁ pajānāti}), This means that the individual practising it is one who is \emph{seeing}.

Further, the \textit{Sutta} says that if the \textit{Satipaṭṭhānā} is practised for between seven years to seven days the individual so practising it can expect either Arahatship or \textit{anāgāmi}-ship. This therefore indicates that, if such great results are to be expected, its practice has to be a full-time pursuit which cannot in any way be taken lightly. For instance, \textit{kamesu micchācārā vāyāmo} will not be a mere avoidance of `wrongful' sex conduct as it is sometimes supposed to be, but a \emph{complete cutting away} from \emph{all} pleasures of the senses. Such a thorough practice is very difficult for a householder. Therefore it would be incorrect to expect one to become a \textit{sotāpanna}, \textit{sakadāgāmi} or \textit{anāgāmi}, or even to reach the Path, by a repetition of the \textit{Satipaṭṭhānā Sutta} however often and regularly that be.

It is also sometimes thought that the fruits mentioned in the \textit{Satipaṭṭhānā Sutta} can be achieved quickly and in a comfortable manner without sufficient renunciation. Such individuals sooner or later find themselves disillusioned. And then the worst of it all happens. Having been so disillusioned, they begin to wonder whether the Buddha has been right or wrong; and to add to the bargain they imagine that they are now in a better position to wonder.

The Buddha says that the \textit{Satipaṭṭhānā} is the `one and only way' (\textit{ekāyano maggo}) to the full comprehension of the Four Noble Truths and therefore to Arahatship. In order to see this one should examine in detail what is meant by `abides seeing the nature of things in things' (\textit{dhammesu dhammānupassī viharati}).

\textit{Viharati} means abides or lives. That means one is having living experience. In other words one is \emph{conscious} of something Categorizing broadly, one is conscious of the four Groups of Form, Feeling, Perception and \textit{Determinations}. That any of these Four Groups is \emph{present} means one is conscious of it.

Consciousness is \emph{always} entailed. That is why Consciousness is not one of the four \textit{satipaṭṭhāna} -- the four \textit{satipaṭṭhānas} being on the Body (i.e. the most important Form to one), Feeling, Mentality (\textit{citta}) and dhammas (\textit{things}). Not doubt Consciousness is included in the list of the \emph{dhammas} which are to be contemplated on under the fourth \textit{satipaṭṭhāna} called \textit{dhammesu dhammānupassī viharati}. But that is different.

Now, all living experience can be classified under two categories:

\begin{enumerate}
\def\labelenumi{\arabic{enumi}.}
\item
  Experiencing something and having \emph{right} knowledge about the experience,
\item
  Experiencing something and having \emph{wrong} knowledge about the experience.
\end{enumerate}

I can see a rope and recognize it as a rope, or I can see a rope and take it for a snake. Whilst seeing the sun shining upon the sand I can take it to be `water' or to be the sun shining upon the sand. The seeing, together with the wrong understanding, is as much a living experience as the seeing together with the right understanding is. Likewise one experiences a certain thing. One feels a feeling (\textit{vedanaṁ vediyāmī}). That is, there is \textit{vedanāsu} \ldots\hspace{0pt} \textit{viharati}. One experiences a lustful thought (\textit{sarāgaṁ cittaṁ}). That is, there is \textit{citte} \ldots\hspace{0pt} \textit{viharati}. Likewise there is the experience of the various \textit{dhammas} That is, there is \textit{dhammesu} \ldots\hspace{0pt} \textit{viharati}. But - and this is the important thing -- one can see the true nature of that which is being experienced \emph{or} not see it. Seeing the true nature of the feeling that is being experienced is the \textit{vedanānupassī}. Likewise, seeing the true nature of the thought is the \textit{cittānupassī}. Seeing the true nature of the \textit{dhamma} is the \textit{dhammānupassī}. So, together we get \textit{vedanāsu vedanānupassī viharati}, \textit{citte cittānupassī viharati} and \textit{dhammesu dhammānupassi viharati}.

The position with regard to the \textit{satipaṭṭhāna} on the body is slightly different towards the latter part, in that one does not and cannot experience in oneself all the states of the body described therein, such as the dead body in the charnel-field, though of course one sees that the same fate will befall one's own body. In this particular case one sees the phenomenon externally (i.e. as of another) but as applicable internally (i.e. to oneself) too. A matter worthy of note in this \textit{satipaṭṭhāna} concerning the body is the use of the word \textit{kāyasankhāra}. Having spoken of the in-breathing and out-breathing, the word \textit{kāyasaṅkhāra} is brought in. \textit{Kāyasankhāra}, we have seen, has been defined as in-breathing and out-breathing. Diverting the mind to \textit{kāyasaṅkhāra} is to indicate that the in-breathing and out-breathing is the \textit{saṅkhāra} upon which the body stands supported When the thing (body, in this case) is seen to depend on a \textit{saṅkhāra} (breathing, in this case) that is subject to arising and passing away, then it is seen that the thing (body) is also subject to arising and passing away, and is therefore Not-self. Therefore to translate \textit{kāyasankhāra} as `activity of the body' or as `bodily formation' is not only wrong but also misleading and misses the entire purpose.

It is quite clear that there can be \emph{no other way} for one to fully comprehend things. The \textit{dhammesu} \ldots\hspace{0pt} \textit{viharati} part is necessary for \emph{full} comprehension, since full comprehension comes only with actual experience. That is why, though the \textit{sekha} sees the cessation of Suffering, he is described as not having fully comprehended it. To fully comprehend it or \emph{penetratively} see it \emph{through and through} he must also \emph{experience} it. The Arahat is at all times experiencing the cessation of Suffering. He therefore fully comprehends it and sees it penetratively through and through.

\protect\hypertarget{truth-for-him}{}{}The \textit{Satipatthānā Sutta} assumes a prior understanding of the Buddha's Teaching. Obviously, this understanding cannot be obtained from this \textit{Sutta}. It has to be obtained from the other Suttas. Therefore, before embarking on the actual practice of the \textit{Satipatthāna} one has to go through the other Suttas and devote a great deal of time to trying to obtain sufficient understanding of the Buddha's Teaching. And the most certain way of obtaining a proper understanding of it is to build one's understanding on the very fundamentals that the Buddha has taught in the \textit{Mūlapariyāya Sutta}. But very hard work is needed. In conclusion one can only repeat what has already been said in the preface - that is, that though these fundamentals and their resultant implications are very difficult to \emph{see}, they edify him who sees them. They are truth \emph{for him}.
