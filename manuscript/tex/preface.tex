\chapter{Preface}

In a letter to the author, the late Venerable Ñāṇavīra Thera stated:


\begin{quotation}
" … unless one’s thinking is all-of-a-piece, that is, properly speaking,
no \textbf{thinking} at all. A person who simply makes a collection – however
vast – of ideas, and does not perceive that they are at variance with
one another, has actually no ideas of his own, and if one attempts to
instruct him (which is to say, to \textbf{alter} him) one finds that one is
adding to the junk-heap of assorted notions without having any other
effect whatsoever. As Kierkegaard has said, 'Only the truth that edifies
is truth \textbf{for you}.' Nothing that one can say to these collectors of
ideas is truth \textbf{for them}. What is wanted is a man who will argue a
single point, and go on arguing it until the matter is clear to him,
\textbf{because he sees that everything else depends upon it}. With such a
person communication (i.e. of truth that edifies) can take place."


\end{quotation}

More so does the above apply when it comes to the Buddha’s Teaching. In
one’s understanding of it, one must form an articulated, consistent,
whole; a whole such that no one part can be modified without affecting
the rest. At the outset it is not so important that the understanding is
\textbf{right}. That can \textbf{only} come later. Nobody, after all, who has not
reached the Path can afford to assume that he is right about the
Buddha’s Teaching.


With the Buddha’s Teaching, however, if one’s understanding of it is
wrong, one will find that one cannot form a consistent whole. It then
becomes the surest sign that some revision is necessary \textbf{right down the line}.


In the \emph{Mūlapariyāya Sutta} (\emph{Majjhima Nikāya I}) the Buddha taught
certain things as the fundamentals. Either these \textbf{are} the fundamentals
with regard to the problem of Suffering and its cessation, or the Buddha
is wrong. It cannot be both. And if they \textbf{are} the fundamentals, then
there can be no hope of understanding his Teaching unless they are
sufficiently appreciated. These fundamentals and their resultant
implications are, however, difficult to \textbf{see} though easy to state. They
are beyond the scope of scholasticism. But they edify him who sees them.
They are truth \textbf{for him}.


There is always, however, the person to whom the Buddha’s Teaching
appears easy. But it appears easy only because he takes it up
objectively and in conceptual fashion, and then passes it on; like the
man who takes up a basket of mangoes, opens the lid, gazes at the
mangoes, closes the lid, and passes the basket on. Taking up the
Teaching in scholarly fashion, he thinks: What after all is there so
difficult in understanding Impermanence, Not-self, and Suffering? As one
breaks up the chariot into its constituent parts and finds there is
nothing permanent or self-existent in it, he breaks up the personality
(i.e. the Five Grasping Groups) into bits and pieces, these into further
bits and pieces, and proclaims he cannot find any self in it anywhere.
Therefore he thinks he perceives Not-self! The result is that he has
very effectively called a halt to his own progress. In spite of all the
masterly analysis of his personality into as many constituents as
possible, and his finding no self-existent thing in it anywhere, he
still looks upon his personality as '\textbf{my self}'! He remains just where
he started from, though he thinks he has advanced.


This 'ease' of understanding only points to the shallowness of the
understanding. What a decade ago, pursuing such scholasticism, appeared
easy to the present writer – that he now finds to be by no means easy.
This however, not because his thinking powers have declined, but because
the urge in him to \textbf{see} a solution to the problem of his own existence
has disquietingly brought out into the open those very same difficulties
which in the earlier years he chose to treat rather lightly.


To what individual does the Buddha’s Teaching matter? It matters to the
individual who sees that the problem of his own existence is a \textbf{present}
problem, and wishes to have a solution to it \textbf{in the present}. It is
therefore only to such an individual that any book which endeavours to
indicate what the Buddha taught can really matter.


R.G. de S. Wettimuny


40/13, Park Road, Colombo 5. \\
14.4.69


\hypertarget{x-a-note-on-the-translation-of-the-pali}{\section*{A Note on the Translation of the Pali}}
With regard to the translation of the Pali which is the language of the
Buddhist Texts, the usual difficulty remains. That is, to produce a
version which is both readable and accurate in meaning. To some extent
readability has had to be sacrificed for the sake of accuracy in
meaning. Hence the appearance of a few rather unusual phrases.


The Pali has been given alongside in many instances. This should assist
the reader who has some knowledge of Pali. Actually one cannot come to
understand the Buddha’s Teaching without becoming familiar with the
Pali.


R.G. de S.W.



\vfill\eject

\begin{quotation}
"Formerly, and now also, Anurādha, it is just Suffering and the
cessation of Suffering that I proclaim."


\emph{pubbe cāham Anurādha etarahi ca dukkhañceva paññāpemi dukkhassa ca nirodhanti}


 — \href{https://suttacentral.net/sn44.2/en/sujato}{SN 44.2}, Anurādha Sutta


\end{quotation}

